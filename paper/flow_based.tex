\documentclass[11pt,twocolumn]{article}
\usepackage{graphicx}
\usepackage[left=1.80cm, right=1.80cm, top=2.00cm, bottom=2.00cm]{geometry}
\usepackage{amsmath,bm}
\usepackage[colorlinks]{hyperref}
\usepackage[backend=biber, bibencoding=utf8]{biblatex}
\usepackage{eurosym}
\usepackage[dvipsnames]{xcolor}
\usepackage{subcaption}
\usepackage{booktabs} % for table (toprule, bottomrule)
\usepackage{enumitem} % for alphabetical enumeration 
\usepackage{accents}
\usepackage[capitalise]{cleveref}



\addbibresource{main.bib}
\graphicspath{{../figures/}{../figures/example/}}

\crefname{relation}{Rel.}{Rels.}
\creflabelformat{relation}{(#2#1#3)}
\crefname{constraint}{Constr.}{Constrs.}
\creflabelformat{constraint}{(#2#1#3)}
\crefname{subequations}{Eqs.}{Eqs.}
\creflabelformat{subequations}{(#2#1#3)}

\setlength\parindent{8pt}

% style operators
\newcommand{\ie}{\textit{i.e.} }
\newcommand{\eg}{\textit{e.g.} }
\newcommand{\ubar}[1]{\underaccent{\bar}{#1}}
\newcommand{\note}[1]{\textcolor{Orange}{#1}}
\newcommand{\vpad}{\vspace{1mm}}
\newcommand{\hpad}{\hspace{15pt}}
\newcommand{\resultsin}[1]{\hspace{6pt} \bot  \hspace{6pt} #1}
\newcommand{\Forall}[1]{\hspace{10pt} \forall \,\, #1 }
\newcommand{\pdv}[2]{\frac{\partial #1}{\partial #2}}



% general symbols
\newcommand{\state}{s_{i,t}}
\newcommand{\capacity}{S_{i}}
\newcommand{\costfactor}{\gamma_{i,t}}
\newcommand{\capacityupper}{\bar{S}}
\newcommand{\capacitylower}{\ubar{S}}
\newcommand{\muuppernom}{\bar{\mu}^\text{nom}_{i}}
\newcommand{\mulowernom}{\ubar{\mu}^\text{nom}_{i}}

\newcommand{\kk}{k~\euro\,}

%generation
\newcommand{\generation}{g_{s,t}}
\newcommand{\generationpotential}{\bar{g}_{s,t}}
\newcommand{\generationshare}[1][n]{\omega_{#1,s,t}}
\newcommand{\nodalgeneration}[1][n]{g_{#1,t}}
\newcommand{\capacitygeneration}{G_{s}}
\newcommand{\capacitygenerationupper}{\bar{G}_{s}}
\newcommand{\capacitygenerationlower}{\ubar{G}_{s}}
\newcommand{\operationalpricegeneration}{o_{s}}
\newcommand{\capitalpricegeneration}{c_{s}}
\newcommand{\mulowergeneration}{\ubar{\mu}_{s,t}}
\newcommand{\muuppergeneration}{\bar{\mu}_{s,t}}
\newcommand{\muuppergenerationnom}{\bar{\mu}^\text{nom}_{s}}
\newcommand{\mulowergenerationnom}{\ubar{\mu}^\text{nom}_{s}}


% flow
\newcommand{\flow}{f_{\ell,t}}
\newcommand{\capacityflow}{F_{\ell}}
\newcommand{\capacityflowUpper}{\bar{F}_{\ell}}
\newcommand{\capacityflowLower}{\ubar{F}_{\ell}}
\newcommand{\operationalpriceflow}{o_\ell}
\newcommand{\capitalpriceflow}{c_{\ell}}
\newcommand{\mulowerflow}{\ubar{\mu}_{\ell,t}}
\newcommand{\muupperflow}{\bar{\mu}_{\ell,t}}
\newcommand{\muupperflownom}{\bar{\mu}^\text{nom}_{\ell}}
\newcommand{\mulowerflownom}{\ubar{\mu}^\text{nom}_{\ell}}

% storage
\newcommand{\storage}{g_{r,t}}
\newcommand{\storagedispatch}{\storage^\text{dis}}
\newcommand{\storagecharge}{\storage^\text{sto}}
\newcommand{\storagesoc}{\storage^\text{ene}}
\newcommand{\storageprevioussoc}{g_{r,t-1}^\text{ene}}

\newcommand{\efficiency}{\eta_{r}}
\newcommand{\efficiencydispatch}{\efficiency^\text{dis}}
\newcommand{\efficiencycharge}{\efficiency^\text{sto}}
\newcommand{\efficiencysoc}{\efficiency^\text{ene}}

\newcommand{\operationalpricestorage}{o_r}
\newcommand{\capitalpricestorage}{c_r}
\newcommand{\capacitystorage}{G_r}
\newcommand{\capacitystorageupper}{\bar{G}_r}
\newcommand{\capacitystoragelower}{\ubar{G}_r}
\newcommand{\mulowerstorage}{\ubar{\mu}_{r,t}}
\newcommand{\muupperstorage}{\bar{\mu}_{r,t}}
\newcommand{\mulowerstoragedispatch}{\ubar{\mu}_{r,t}^\text{dis}}
\newcommand{\muupperstoragedispatch}{\bar{\mu}_{r,t}^\text{dis}}
\newcommand{\mulowerstoragecharge}{\ubar{\mu}_{r,t}^\text{sto}}
\newcommand{\muupperstoragecharge}{\bar{\mu}_{r,t}^\text{sto}}
\newcommand{\mulowerstoragesoc}{\ubar{\mu}_{r,t}^\text{ene}}
\newcommand{\muupperstoragesoc}{\bar{\mu}_{r,t}^\text{ene}}
\newcommand{\muupperstoragenom}{\bar{\mu}^\text{nom}_r}
\newcommand{\mulowerstoragenom}{\ubar{\mu}^\text{nom}_r}
\newcommand{\mustateofcharge}{\lambda^\text{ene}_{r,t}}
\newcommand{\munextstateofcharge}{\lambda^\text{ene}_{r,t+1}}


% other
\newcommand{\lagrangian}{\mathcal{L}}
\newcommand{\lmp}[1][n]{\lambda_{#1,t}}
\newcommand{\averagelmp}{\left<\lmp\right>_t}
\newcommand{\demand}[1][n]{d_{#1,t}}
\newcommand{\nodaldemand}[1][n]{d_{#1,t}}
\newcommand{\demandshare}[1][n]{\omega_{#1,a,t}}
\newcommand{\injection}{p_{n,t}}
\newcommand{\netconsumption}[1][n]{p^{-}_{#1,t}}
\newcommand{\netproduction}[1][n]{p^{+}_{#1,t}}
\newcommand{\selfconsumption}[1][n]{p^{\circ}_{#1,t}}
\newcommand{\totalnetconsumption}{p^{-}_{t}}
\newcommand{\totalnetproduction}{p^{+}_{t}}
\newcommand{\totalselfconsumption}{p^{\circ}_{t}}

% network
\newcommand{\incidence}[1][n]{K_{#1,\ell}}
\newcommand{\incidencegenerator}[1][n]{K_{#1,s}}
\newcommand{\incidencestorage}[1][n]{K_{#1,r}}
\newcommand{\incidenceasset}[1][n]{K_{#1,i}}
\newcommand{\ptdf}[1][n]{H_{\ell,#1}}
\newcommand{\cycle}{C_{\ell,c}}
\newcommand{\impedance}{x_\ell}
\newcommand{\cycleprice}{\lambda_{c,t}}



% Cost Terms
\newcommand{\emission}{e_{s}}
\newcommand{\emissionprice}{\mu_{\text{CO2}}}
\newcommand{\megawatthour}{MWh$_\text{el}$}
\newcommand{\totalcost}{\mathcal{TC}}
\newcommand{\cost}{\mathcal{C}}
\newcommand{\opex}{\mathcal{O}}
\newcommand{\opexgeneration}{\mathcal{O}^G}
\newcommand{\opexflow}{\mathcal{O}^F}
\newcommand{\opexstorage}{\mathcal{O}^E}
\newcommand{\capexgeneration}{\mathcal{C}^G}
\newcommand{\capexflow}{\mathcal{C}^F}
\newcommand{\capexstorage}{\mathcal{C}^E}
\newcommand{\emissioncost}{\mathcal{E}}
\newcommand{\remainingcost}{\mathcal{R}}
\newcommand{\scarcitycost}{\remainingcost^\text{scarcity}}
\newcommand{\subsidycost}{\remainingcost^\text{subsidy}}


% allocation quantities
\newcommand{\allocategeneration}[1][s, n]{A_{#1,t}}
\newcommand{\allocatestoragedispatch}[1][r, n]{A_{#1,t}}
\newcommand{\allocatepeer}[1][m \rightarrow n]{A_{#1,t}}
\newcommand{\allocateflow}[1][n]{A_{\ell,#1,t}}
\newcommand{\allocatetransaction}[1][s \rightarrow n]{A_{#1,\ell,t}}
\newcommand{\allocatestate}[1][i, n]{A_{#1,t}}

\newcommand{\allocatecost}[1][n \rightarrow i]{\cost_{#1, t}}
\newcommand{\allocatecapexgeneration}[1][n \rightarrow s]{\capexgeneration_{#1,t}}
\newcommand{\allocatecapexflow}[1][n \rightarrow \ell]{\capexflow_{#1,t}}
\newcommand{\allocatecapexstorage}[1][n \rightarrow r]{\capexstorage_{#1,t}}
\newcommand{\allocateopex}[1][n \rightarrow s]{\opex_{#1,t}}
\newcommand{\allocateemissioncost}[1][n \rightarrow s]{\emissioncost_{#1,t}}
\newcommand{\allocatescarcitycost}[1][n \rightarrow i]{\scarcitycost_{#1,t}}
\newcommand{\allocatesubsidycost}[1][n \rightarrow i]{\subsidycost_{#1,t}}



\begin{document}


\title{A dispatch-based approach to fully allocate costs in optimal power systems}
\author{F. Hofmann, Brown T.}

\maketitle

\begin{abstract}
% Maximizing the welfare of all market participants within a power system is a common approach in energy system modelling. It leads to perfectly scheduled operations of generators and ... 
\end{abstract}


\subsection*{Highlights}
\begin{itemize}
 \item The LMP always breaks down into contributions of different cost terms (OPEX, CAPEX, etc.) of the system.
 \item Breaking the LMP further down into contributions of components (generator, transmission line, etc.) requires assumptions on power flow assignments, where prominent methods as Average Participation or Flow Based Market Coupling can be applied.
 \item ...
 \item ...
\end{itemize}


% \section{Introduction}



\subsubsection*{Nomenclature}

\begin{table}[h]
    \centering
    \begin{tabular}{ll}
        $\lmp$ & Locational Market Price at bus $n$ and  \\ & time step $t$ in \euro/MW \vpad \\
        $\demand$ & Electric demand per bus $n$, demand type $a$, \\ & time step $t$ in MW  \vpad \\
        $\generation$ & Electric generation of generator $s$, \\ & time step $t$  in MW \vpad \\
        $\flow$ & Active power flow on line $\ell$, \\ & time step $t$ in MW   \vpad \\
        $\operationalpricegeneration$ & Operational price in \euro/MW \vpad \\
        $\capitalpricegeneration$ & Capital Price in \euro/MW \vpad \\
        $\capitalpriceflow$  & Capital Price in \euro/MW for transmission \\ & capacity on line $\ell$  \vpad \\
        $\capacitygeneration$ & Generation capacity in MW \vpad \\
        $\capacityflow$ & Transmission capacity in MW \vpad \\
        $\incidence$ & Incidence matrix \vpad 
    \end{tabular}
\end{table}

\section{Economic Context}

In long-term operation and investment planning models, the total costs $\totalcost$ of a power system is the sum of multiple cost terms. Typically, these include operational expenditures (OPEX) $\opexgeneration$ for generators, expenditures for emissions $\emissioncost$, capital expenditures (CAPEX) for generators $\capexgeneration$, CAPEX for the transmission system $\capexflow$ and possible other terms, \ie
\begin{align}
\totalcost &= \opexgeneration + \emissioncost +  \capexgeneration +  \capexflow + ...
\label{eq:total_cost}
\end{align}
In turn, each of these terms $\mathcal{C} = \{\opexgeneration, \emissioncost, \capexgeneration, \capexflow, ...\}$ consists of cost associated to the asset $i$ in the system, 
\begin{align}
    \cost = \sum_{i} \cost_{i}
    \label{eq:asset_cost}
\end{align} 
where an asset describes any operating part of the network, such as a generator, line, energy storage etc. %For example the total OPEX, $\opexgeneration = \sum_{s} \opexgeneration_{s}$, is the combined OPEX of all generators $s$.


In a cost-optimal setup with minimized $\totalcost$, the Locational Marginal Price (LMP) describes the price for an incremental increase of electricity demand $\demand$ at node $n$. It is given by the derivative of the total system cost $\totalcost$ with respect to the local demand $\demand$
\begin{align}
\lmp = \pdv{\,\totalcost}{\,\demand}
\label{eq:lmp}
\end{align}
% 
\begin{figure}[h]
\centering
\includegraphics[width=.8\linewidth]{price_decomposition.png}
\caption{Schematic decomposition of the Locational Market Price $\lmp$. In power system model with optimal long-term operation and planning, the total system costs $\totalcost$ split into different cost terms, \ie OPEX and CAPEX for production and transmission and possibly other expenditures. }
\label{fig:price_decomposition}
\end{figure}
% 
This leads to a nodal pricing where over the span of optimized time steps $t$, the system costs are partially or totally payed back by the consumers 
\begin{align}
\totalcost - \remainingcost &=  \sum_{n,t} \lmp \, \demand
\label{eq:total_revenue}
\end{align}
depending on the costs $\remainingcost$ which are independent of the nodal demand  
\begin{align}
 \pdv{\remainingcost}{\demand} = 0
\end{align}
% 
Generally speaking, the cost term $\remainingcost$, not covered by the consumers, results from additional demands on the network design, such as capacity expansion limits or minimum share of one technology in the power mix. However, in most cases, where $\remainingcost \ll  \totalcost$, these play a minor role. 

From feeding \cref{eq:total_cost} into \cref{eq:lmp} it follows naturally that the LMP splits into contribution to the above mentioned cost terms. This relation, which we schematically show in \cref{fig:price_decomposition}, was already shown in extensive investigations of the LMP \cite{schweppe_spot_1988}. However the question of how the LMP can be decomposed into contributions of single cost terms $\cost_{i}$ associated with asset $i$ remains unanswered. This work aims at presenting and illustrating a an intuitive, peer-to-peer cost allocation including all network assets. 



\section{Dispatch-Based Cost Allocation}
\label{sec:theory}
\begin{subequations}\label[subequations]{eq:general_scheme}

Let $\cost_{i}$ denote a general cost term associated with asset $i$. Consider a long-term equilibrium in a power system with perfect competition, then, according to the zero-profit condition, each cost term $\cost_{i}$ can be considered as a cost-weighted sum of the operational state $s_{i,t}$ of asset $i$, \ie
\begin{align}
    \cost_{i} = \sum_t  \costfactor \, \state
    \label{eq:cost_decomposition}
\end{align}
where $\costfactor$ denotes a cost factor in \euro/MW. 
If $\cost_i$ describes the OPEX occasioned by asset $i$, the cost factor $\costfactor$ is simply given by the marginal operational price $o_i$. However, as we will show later, if it describes the CAPEX of asset $i$, $\costfactor$ is a composition of shadow prices $\mu_{i,t}$ given at the optimum. %As we will show later the composition of $\mu_{i,t}$ must be determined for each type of asset individually.


% where $\costfactor$ is either a fixed operational price or a composition of shadow prices
Following the implications of \cref{eq:lmp,eq:total_revenue},  we define the cost $\allocatecost$ that consumers at bus $n$ have pay to asset $i$ at time $t$, in order to compensate for $\cost_i$. This leads us to 
\begin{align}
    \allocatecost = \costfactor \,\pdv{\state}{\demand} \demand
    \label{eq:cost_allocation}
\end{align}
The derivative on the right hand side is defined through the sensitivity of the operational variable $\state$ at the optimum against changes in the demand. 
\begin{align}
    \allocatestate \rightarrow\pdv{\state}{\demand} \demand
    \label{eq:state_allocation}
\end{align}    
... may be interpreted as the amount of power that asset $i$ supplies demand $\demand$ with. It heavily relies on the derivative of the operational state with respect to the nodal demand $\partial \state / \partial \demand$. 
From the natural fact that the sum of all contributions must return the cost term, 
\begin{align}
    \cost_i = \sum_{n,t} \allocatecost
    \label{eq:cost_payback}
\end{align}
it follows that $\allocatestate$ must fulfill
\begin{align}
    \state = \sum_n \allocatestate
    \label{eq:state_allocation_constraint}
\end{align}
The last equation states that all power produced or processed by asset $i$ must be totally consumed by the network demand $\demand$.  
Finally, the total contribution from node $n$ at time $t$ to the cost term $\cost$ amounts 
\begin{align}
    \cost_{n,t} = \sum_i \allocatecost
\end{align}
\end{subequations}

\begin{table*}[t]
    \begin{center}
        \begin{tabular}{l|c|c|c|c|c}
        & $i$ & $\cost$ & $\cost_i$  & $\costfactor$ & $\state$  \\
        \toprule 
        OPEX Production & $s$ & $\opexgeneration$ & $\sum_{t} \operationalpricegeneration \, \generation$   & $\operationalpricegeneration$ & $\generation$ \\  
        OPEX Transmission  & $\ell$ & $\opexflow$ & $\sum_{t} \operationalpriceflow \, |\flow|  $ & $\operationalpriceflow$ & $|\flow|$ \\  
        OPEX Storage & $r$  & $\opexstorage$ & $\sum_{t} \operationalpricestorage \, \storagedispatch$ &  $\operationalpricestorage$ & $\storage$ \\
        \midrule   
        CAPEX Production & $s$ & $\capexgeneration$ & $ \capitalpricegeneration \capacitygeneration$ & $\muuppergeneration$ & $\generation$ \\
        CAPEX Transmission & $\ell$ & $\capexflow$ & $ \capitalpriceflow \capacityflow$ & $\left(\muupperflow - \mulowerflow \right)$ & $\flow$ \\
        CAPEX Storage & $r$ & $\capexstorage$ & $ \capitalpricestorage \capacitystorage$ & $ \muupperstoragedispatch - \mulowerstoragedispatch  + (\efficiencydispatch )^{-1} \mustateofcharge $ & $\storage$ \\
        \midrule
        Emission Cost & $s$ & $\emissioncost$ & $ \emissionprice \, \emission \, \generation$ & $\emissionprice \,\emission$ & $\generation$ \\
        % \bottomrule   
    \end{tabular}
    \end{center}
    \caption{Mapping of different cost terms to the cost allocation scheme given in \cref{eq:general_scheme}. These include OPEX \& CAPEX for production, transmission and storage assets in the network, as well as a cost term for the total Green House Gas (GHG) emissions.}
    \label{tab:cost_allocation_map}
\end{table*}
    
Now, assume a network with generators $s$, transmission lines $\ell$ and storage units $r$. Each asset $i = \{s, \ell, r\}$ adds an term for OPEX and a term for CAPEX to the total system cost $\totalcost$.

\subsection{OPEX Allocation}

Let the operational price for an asset $i$ be given by $o_i$. Then, for example the OPEX occasioned by generator $s$ is given by 
\begin{align}
    \opexgeneration_s = \sum_t \operationalpricegeneration \, \generation 
    \label{eq:opexgeneration}
\end{align}
where $\generation$ denotes its power generation at time $t$. As \cref{eq:opexgeneration} matches the form of \cref{eq:cost_decomposition} which allows us to use the above presented scheme in \cref{eq:general_scheme}. As a result we obtain 
\begin{align}
    \allocateopex[n \rightarrow s] &= 
   \operationalpricegeneration \,  \allocategeneration
\label{eq:allocate_opexGeneration_detailed}
\end{align}
which is the contribution of $\demand$ to the OPEX at generator $s$.
The quantity  
\begin{align}
 \allocategeneration = \pdv{\generation}{\demand} \, \demand
 \label{eq:allocate_peer}
\end{align}
can be considered as the power that is produced by generator $s$ and consumed at node $n$ at time $t$. \\
In the same manner, we can follow the scheme allocate OPEX for flow $\flow$ and storage unit dispatch $\storagedispatch$. As we assume a bidirectional flow on line $\ell$, the OPEX is set proportional to the absolute value of the flow. The upper section in \cref{tab:cost_allocation_map} shows the mapping of variables to \cref{eq:general_scheme} in order to define the full OPEX allocation. \\

The scheme works for all other cost attached to the operational state of an asset $i$. Given for example a fix price for emissions $\emissionprice$ in \euro\, per tonne-CO$_2$ equivalents, the cost term for emission adds up to 
\begin{align}
 \emissioncost = \emissionprice \, \sum_s  \emission \, \generation
\end{align}
where $\emission$ denotes the emission factor in tonne-CO$_2$ per \megawatthour\, of generator $s$.
The allocated payment for consumers at bus $n$ at time $t$ assigned to generator $s$ is then given by 
\begin{align}
 \allocateemissioncost = \emissionprice \, \emission \, \allocategeneration
\end{align}


\subsection{CAPEX Allocation}
For the CAPEX allocation, it becomes crucial to look at the individual relations between operational state $\state$ and the capacity limit. For all assets, let the capital price for one unit capacity expansion be denoted by $c_i$.  All quantities for the CAPEX allocation, which we now discuss in detail, are summarized in the middle section of \cref{tab:cost_allocation_map}.  

\subsubsection{Generators}

The nominal capacity $\capacitygeneration$ constrains the generation $\generation$ in the form of 
\begin{align}
\generation - \generationpotential \capacitygeneration  &\le 0 \resultsin{\muuppergeneration} \Forall{s,t} 
\label[constraint]{eq:upper_generation_capacity_constraint}\\ 
- \generation &\le 0 \resultsin{\mulowergeneration} \Forall{s,t} 
\label[constraint]{eq:lower_generation_capacity_constraint}
\end{align}
where $\generationpotential \in \left[ 0,1\right]$ is the capacity factor for renewable generators. At a cost-optimum, these two constraints yield the shadow prices $\muuppergeneration$ and $\mulowergeneration$.  As shown in \cite{brown_decreasing_2020} and in detailed in \cref{sec:zero_profit_generation}, over the whole time span, the CAPEX for generator $s$ is payed back by the production $\generation$ times the shadow price $\muuppergeneration$, 
\begin{align}
 \capexgeneration_s = \capitalpricegeneration \capacitygeneration = \sum_t \muuppergeneration \,  \generation 
 \label{eq:no_profit_capex_generation}
\end{align}
This representation connects the CAPEX with the operational state of generator $s$, \ie matches the form in \cref{eq:cost_decomposition} allows for using the cost allocation scheme. The resulting the CAPEX allocation is given by
\begin{align}
 \allocatecapexgeneration = \muuppergeneration \, \allocategeneration
 \label{eq:allocate_capexGeneration_detailed}
\end{align}
How does this allocation behave? According to the polluter pays principle, it differentiates between consumers who are `responsible` for investments and those who are not. If $\muuppergeneration$ (in literature often denoted as the Quality of Supply) is bigger than zero, the upper Capacity \cref{eq:upper_generation_capacity_constraint} is binding. Thus it is these times steps which push investments in $\capacitygeneration$. If $\muuppergeneration = 0$, the generation $\generation$ is not bound and investments are not necessary. 
When summing over all CAPEX payments to generator $s$ in \cref{eq:allocate_capexGeneration_detailed} , we can use \cref{eq:state_allocation_constraint,eq:no_profit_capex_generation} to see that each generator retrieves exactly the cost that were spent to build the capacity $\capacitygeneration$.
 

\subsubsection{Transmission Lines}

The transmission capacity $\capacityflow$ limits the flow $\flow$ in both directions,
\begin{align}
\flow - \capacityflow &\le 0 \resultsin{\muupperflow} \Forall{\ell,t} 
\label[constraint]{eq:upper_flow_capacity_constraint} \\
- \flow - \capacityflow &\le 0 \resultsin{\mulowerflow} \Forall{\ell,t} 
\label[constraint]{eq:lower_flow_capacity_constraint}
\end{align}
which yield the shadow prices $\muupperflow$ and $\mulowerflow$. Again, we use the result of \cite{brown_decreasing_2020} (for details see \cref{sec:zero_profit_flow}) which derives that over the whole time span, the investment in line $\ell$ is payed back by the shadow prices times the flow 
\begin{align}
\capexflow_\ell = \capitalpriceflow \capacityflow = \sum_{t} \left( \muupperflow - \mulowerflow \right)  \flow 
\label{eq:no_profit_capex_flow}
\end{align}
From here, we follow the scheme in \cref{eq:general_scheme} which finally defines the CAPEX allocation as 
\begin{align}
    \allocatecapexflow &=  
   \left( \muupperflow - \mulowerflow\right) \, \allocateflow
   \label{eq:allocate_capexFlow_detailed}
\end{align}
The quantity 
\begin{align}
 \allocateflow =  \pdv{\flow}{\demand}\, \demand
\end{align}
can be interpreted as the flow that the demand at node $n$ and time $t$ causes on line $\ell$.
% According to \cref{eq:state_allocation_constraint} it has to fulfill 
% \begin{align}
%  \flow = \sum_{n} \allocateflow
% \label{eq:allocate_flow_constraint}
% \end{align}
The  shadow prices $\muupperflow$ and $\mulowerflow$ again can be seen as a measure for necessity of transmission investments at $\ell$ at time $t$. Hence, the definition of $\allocatecapexflow$ states that consumers, which retrieve power flowing on congested lines, yielding a bound \cref{eq:upper_flow_capacity_constraint} or \eqref{eq:lower_flow_capacity_constraint}, pay compensations for the resulting investments at $\ell$. Again the sum of all CAPEX payments to line $\ell$ equals the total CAPEX spent. This is seen when summing \cref{eq:allocate_capexFlow_detailed} over all buses and time steps and using \cref{eq:state_allocation_constraint,eq:no_profit_capex_flow}


\subsubsection{Storages}
\label{sec:storages}

% The complementary slackness of the above constraints,  
% \begin{align}
%     \capexstorage_r = \capitalpricestorage \, \capacitystorage = \sum_t \muupperstoragedispatch \, \storagedispatch + \muupperstoragecharge \, \storagecharge + \muupperstoragesoc \, \storagesoc
% \end{align}

In a simplified storage model, $\capacitystorage$ limits the storage dispatch $\storagedispatch$ and charging $\storagecharge$. Further it limits the maximal storage capacity $\storagesoc$ by a fix ratio $h_r$, denoting the maximum hours at full discharge. The storage $r$ dispatches power with efficiency $\efficiencydispatch$, charges power with efficiency $\efficiencycharge$ and preserves power from one time step $t$ to the next, $t+1$, with an efficiency of $\efficiencysoc$. In \cref{sec:zero_profit_storage_units} we formulate the mathematical details. As already shown in \cite{brown_decreasing_2020}, the total expenditures at $r$ are fully paid back by differences of the LMP at which the storage ``buys`` and ``sells`` power. Taking only the CAPEX into account the zero profit condition reduces to
\begin{align}
    \notag
    \capexstorage =& \capitalpricestorage \, \capacitystorage \\
    \notag
    =& \sum_t \left(\muupperstoragedispatch - \mulowerstoragedispatch  + (\efficiencydispatch )^{-1} \mustateofcharge \right) \storagedispatch \\
    &- \sum_t \lmp \incidencestorage  \storagecharge \Forall{r} 
    \label{eq:no_profit_capex_storage}
\end{align}
where $\muupperstoragedispatch$ and $\mulowerstoragedispatch$ are the shadow prices of the upper and lower dispatch capacity bound and $\mustateofcharge$ is the shadow price of the energy balance constraint. When applying the cost allocation scheme \cref{eq:general_scheme}, it stands to reason to assume that $\partial \storagecharge / \partial \demand \cdot \demand = 0$, implying that the when a storage charges power, it does not supply any demand. Rather it stands with the demand on the consumer side, retrieving power from producing assets. 
This leaves us with 
\begin{align}
     \allocatecapexstorage = \left(\muupperstoragedispatch - \mulowerstoragedispatch  + (\efficiencydispatch )^{-1} \mustateofcharge \right) \allocatestoragedispatch
\end{align}
and the power allocation 
\begin{align}
    \allocatestoragedispatch = \pdv{\storagedispatch}{\demand} \demand
\end{align}
The latter only allocates dispatched power of storage $r$. Note that this will break \cref{eq:cost_payback} as the payments to $r$ surpass the CAPEX by an amount $\remainingcost^E_r$. 
% \begin{align}
%     \capexstorage_r + \remainingcost^E_r = \sum_n \allocatecapexstorage
% \end{align} 
It is crucial to note that like this, storage units perform a redistribution of money, and therefore distort the cost allocation. So, certain share of what is allocated to the CAPEX of a storage is in another time step spent by the storage in order to buy power from other assets. This effect scales with the amount of installed capacity.

It is possible to incorporate this redistribution effect into the cost allocation, by replacing the demand $\demand$ with the power charge $\storagecharge$ in \cref{eq:general_scheme}. Then, the derived payments that a storage unit $r$ has to pay to asset $i$ is given by $\cost_{r \rightarrow i}$. The sum of those payments due to $r$ will the sum up to $\remainingcost^E_r$. 

% In a final step we can still allocate all cost to the demand $\demand$. However, this would require knowledge about when the power $\storagedispatch$ was initially charged: Given the amount of power $A_{r, t_{in} \rightarrow t_{out}}$ that is charged by storage $r$ at time step $t_{in}$ and discharged at $t_{out}$ ... 



\subsection{Design Constraints}
\label{sec:design_constraints}
Power system modelling does rarely follow a pure Greenfield approach with unlimited capacity expansion. Rather, today's models are setting various constraints defining socio-political or  technical requirements. As mentioned before this will alter the equality of total cost and total revenue, \ie leads to $\remainingcost \ne 0$ in \cref{eq:total_revenue}. More precisely, each constraint $h_j$ (other then the nodal balance constraint) of the form 
\begin{align}
    h_j \left(\state, \capacity \right) - K < 0
    \label{eq:design_constraint}
\end{align}
where $K$ is any non-zero constant, will result in a cost term contributing to $\remainingcost$ and in some cases alter \cref{eq:cost_decomposition} to 
\begin{align}
    \cost_i - \remainingcost_i = \sum_t \costfactor \, \state
\end{align}
The effective revenue per asset $i$ associated with cost term $\cost$ can then be defined as 
\begin{align}
    \cost' = \sum_i  \cost_i - \remainingcost_i
    \label{altered_cost}
\end{align}
Note that $\cost'$ is the amount which is allocated to consumers which, according to the quality of \cref{eq:design_constraint} and the corresponding $\remainingcost_i$, can either be larger, equal or lower than $\cost$.  
In the following we highlight two often used classes of designs constraints in the form of \cref{eq:design_constraint} and show how to incorporate them into the cost allocation.  

\subsubsection{Capacity Expansion Limit}

In more realistic setups, generators, lines or other assets can only be built up to a certain limit. This might be due to land use restrictions or social acceptance problems. %In the following give a general solution how the cost allocation may handle this.  
However, when constraining the capacity $\capacity$  for a subset $I$ of assets to an upper limit $\capacityupper$, in the form of 
\begin{align}
    \capacity - \capacityupper \le 0 \resultsin{\muuppernom} \Forall{i \in I}
\label{eq:capacityexpansionmaximum},
\end{align}
the zero profit condition alters as soon as the constraint becomes binding. Then, the revenue of asset $i$ exceeds its total expenditures (OPEX + CAPEX). More precisely, the allocated CAPEX in \cref{tab:cost_allocation_map} will surpass the actual CAPEX of asset $i$ by the cost it has to pay for the scarcity, given by the absolute value of 
\begin{align}
    \scarcitycost_i = - \muuppernom \capacity \Forall{i \in I}
    \label{eq:scarcitycost}
\end{align}
% \begin{align}
%  \cost_i - \remainingcost_i =  \sum_t \costfactor \, \state \Forall{i}
%  \label{eq:capex_generation_duality_bf2}
% \end{align}

% In order to ensure that the allocated CAPEX sum up to the actual CAPEX at asset $i$, meaning that \cref{eq:cost_payback} holds, we adjust the CAPEX allocations to 
% \begin{align}
%     \allocatecost = \left(\dfrac{c_i}{c_i + \muuppernom}\right) \, \costfactor \, \allocatestate \Forall{i \in I}
% \end{align}
% with the cost factors $\costfactor$ for CAPEX given in \cref{tab:cost_allocation_map}. 
% The costs which consumers at $n$ have to pay for the scarcity impacting asset $i$ are given   
% \begin{align}
%     \allocatescarcitycost = \left(\dfrac{\muuppernom}{c_i + \muuppernom}\right) \, \costfactor \, \allocatestate \Forall{i \in I}
% \end{align}

The costs which consumers at $n$ have to pay for the scarcity impacting asset $i$ are given by 
\begin{align}
    \allocatescarcitycost = \dfrac{\muuppernom}{c_i + \muuppernom} \, \allocatecost^I \Forall{i \in I}
\end{align}
where $\allocatecost^I$ denotes the CAPEX allocation presented above. 

% have to be adjusted, as otherwise consumers would pay too much. As we show in ... updating the prices by a fix ratio 
% \begin{align}
%     \costfactor \rightarrow \dfrac{c_i}{c_i + \muuppernom}\, \costfactor
%     \label{eq:adjustment_lowernom}
% \end{align}
% will . Note that $\muuppernom$ is a positive number, so the adjustment cannot increase the paid price. 


\subsubsection{Brownfield Constraints}

In order to take already built infrastructure into account, the capacity $\capacity$ can be constrained to a minimum required capacity $\capacitylower$. Mathematically this translates to 
\begin{align}
    \capacitylower - \capacity  \le 0 \resultsin{\mulowernom} \hpad \forall{i \in I}
\label{eq:capacityexpansionminimum}
\end{align}
Again, such a setup alters the zero profit condition of asset $i$, as soon as the constraint becomes binding. 
In that case, asset $i$ does not collect enough revenue in order to match the CAPEX. The difference, given by 
\begin{align}
    \subsidycost_i = \mulowernom \capacity
    \Forall{i}
\end{align}
has to be subsidized by governments or communities. It is rather futile wanting to allocate these cost to consumers as assets may not gain any revenue for their operational state, \ie where $\cost^I = \subsidycost_i $. 


% Again, we can balance this effect out by updating the cost factor to 
% \begin{align}
%     \costfactor \rightarrow \costfactor +  \dfrac{\mulowernom \capacity}{\sum_t \state}
% \end{align}
% In contrast to \cref{eq:adjustment_lowernom} the price is not weighted by a factor but shifted for all time steps. This is required as 
% As $\mulowernom$ is a positive number, this will increase the price and therefore the revenue for asset $i$. 

% In order to take this effect into account for the cost allocation, we update the cost factors in \cref{tab:cost_allocation_map} to  

% \begin{align}
%     \costfactor \rightarrow \dfrac{c_i}{c_i + \muuppergenerationnom}\, \costfactor
% \end{align}
% This ensures that the allocated CAPEX sum up to the actual CAPEX at asset $i$ (and \cref{eq:cost_payback} holds).



\section{Assumptions on Power Allocations}
% \section{\texorpdfstring{Extended Solution Space of $\boldsymbol{\slackk[m]}$}{Solution Space of the Slack}}
\label{sec:localizing_allocations}

The presented cost allocation suits for any type of topology and network setup. But so far, the question of how $\allocatestate$ for generators $s$, lines $\ell$ and storages $r$ are defined was left open. We recap that all rely on derivatives of $\generation$, $\flow$ and $\storagedispatch$  with respect to the nodal demand $\demand$ (see \cref{eq:state_allocation_constraint}). \\

Let $\nodalgeneration[m]$ denote the nodal power generation which combines the power production of all producing assets, in this case generators $S$ and storages $R$, at node $n$ and time $t$. It is given by 
\begin{align}
    \nodalgeneration[m] = \sum_{i \in \{S, R\}} \incidenceasset[m] \, \state
\end{align}
with $\incidenceasset[m]$ being 1 if asset $i$ is attached to bus $m$ and zero otherwise. Further, let $\allocatepeer$ collect the power produced by assets at node $m$ and consumed at $n$, given by 
\begin{align}
    \allocatepeer = \sum_{i \in \{S,R\}} \incidenceasset[m] \, \allocatestate  
    % \nodalgeneration[m] = \sum_s \incidencegenerator[m] \, \generation + \sum_r \incidencestorage[m] \storagedispatch \Forall{n}
\end{align}
Now, let $\ptdf$ denote Power Transfer Distribution Factors (PTDF) giving the changes in the flow on line $\ell$ for one unit (typically one MW) of net power production at bus $n$. The linear power flow equation can be written as 
\begin{align}
 \flow  = \sum_m \ptdf[m] \left( \nodalgeneration[m] - \demand[m] \right)
 \label{eq:power_flow_equation}  
\end{align}
Note that for transport models or mixed AC-DC networks, $\ptdf$ can be artificially calculated using the formulation presented in \cite{hofmann_flow_2020-1}.
Taking the derivative with respect to the demand, 
\begin{align}
 \allocateflow = \pdv{\flow}{\demand} \demand = \sum_m \ptdf[m] \left( \allocatepeer  - \delta_{n,m} \demand \right) 
 \label{eq:allocate_peer_to_allocate_flow},
\end{align}
shows that $\allocateflow$ is fully determined through the peer-to-peer allocation $\allocatepeer$. In other words, we only need to know how much power produced at node $m$ is consumed at node $n$ in order to derive the allocated flow $\allocateflow$. Further we can breakdown $\allocatepeer$ to $\allocategeneration$ for generators and $\allocatestoragedispatch$ for storages proportionally to their contribution to the nodal generation $\nodalgeneration[m]$. Unfortunately, the solution for $\allocatepeer$ is non-unique and requires further assumptions. Established flow allocation schemes approach this problem from different directions. Principally two options exist \textit{what} is allocated 
% 
\begin{enumerate}
\item gross power injections \label{gross}
\item net power injections \label{net}
\end{enumerate}
% 
Further it is important \textit{what assumptions} define the allocation, \ie what method is used to define the pairs of sources and sinks. The three suitable approaches we present here are
% 
\begin{enumerate}[label=\alph*., ref=\alph*]
\item Equivalent Bilateral Exchanges (EBE) \cite{galiana_transmission_2003} which assumes
that every producer supplies every consumer proportional to its share in the total consumption. \label{ebe} 
\item Average Participation (AP) \cite{bialek_tracing_1996,achayuthakan_electricity_2010} which traces the flow from producer to consumer following the law of proportional sharing. \label{ap}
\item Flow Based Market Coupling (FBMC) which uses zonal PTDF for allocating power within predefined regions. The interregional exchange is only allocating net power deficit or excess of the regions. \label{fbmc}
\end{enumerate}
% 
We show the mathematical formulation for all combinations \ref{ebe}\ref{gross} - \ref{fbmc}\ref{net} in \cref{sec:gross_ebe,sec:net_ebe,sec:gross_ap,sec:net_ap}.
% 
Principally, type \ref{net} leads to less P2P trades then type \ref{gross} as power from a bus $m$ with $\nodalgeneration[m] \le \nodaldemand[m]$ is not assigned to other buses, only to $m$. 
Further, as literature has often pointed out, the EBE principal \ref{ebe} does not suit for large networks where remote buses would interconnect in the same way as buses in close vicinity \cite{gil_multiarea_2005}. The AP based type \ref{ap} tackles this problem by restricting P2P trades to those which are traceable when applying the proportional sharing principal. Therefore $\allocategeneration$ denotes that part of power produced by bus $m$ which, when only following in the direction of $\flow$, ends up at bus $n$. Type \ref{fbmc} further allows to control the regions or market zones which are netted out in a first step. If in a region $R$ the generation undercuts the demand, $\sum_{n \in R} \nodalgeneration \le \sum_{n \in R} \demand$, none of the inner-regional generation is assigned to other regions. However, it relies on further assumptions such as the Generation Shift Keys determining the production which are deployed for the inter-regional exchange. 

From this point of view, we restrict the focus of this research to \textbf{net} injection allocation on the basis of \textbf{AP}, type \ref{net}\ref{ap}, as locality and no further dependency on external decisions are strong arguments for a transparent cost allocation. We emphasize that the AP based allocation does not imply traceability of the resulting flows, as only the \textit{source}-\textit{sink} relations $\allocatepeer$ are determined by it, the flow allocation still follows the power plow equation given as shown in \cref{eq:allocate_peer_to_allocate_flow}.
In the following example, we illustrate the functionality of the cost allocation by an example of a small artificial network.

\subsection{Numerical Example}
\label{sec:numerical_example}

\begin{figure*}[t]
    \centering
    \includegraphics[width=\linewidth]{example_network.png}
    \caption{Illustrative example of a 2-bus network with one optimized time step. Fixed prices and constraining values are given in the left box for each bus and the transmission line. Optimized values are given in the right boxes. Bus~1 has a cheaper operational price $o$, capital prices are the same for both. As both generator capacities are constraint to 100~MW, the optimization also deploys the generator at bus~2. The resulting electricity prices $\lambda$ are then a composition of all prices for operation and capital investments.}
    \label{fig:example_network}
    \end{figure*}
    % 
% 
% 
Consider a two bus system, as shown in \cref{fig:example_network}, with one transmission line and one generator per bus. Whereas generator~1 (at bus~1) has a cheap operational price of 50 \euro/\megawatthour, an expensive operational price of 200~\euro/\megawatthour is set to  generator~2 (at bus~2). For both the capital investments amount 500~\euro/MW and the maximal capacity is limited to $\capacitygenerationupper$~=~100~MW. The transmission line has a capital price of 100~\euro/MW and no upper capacity limit. With a demand of 60~MW at bus~1 and 90~MW at bus~2, the optimization expands the cheaper generator at bus~1 to its full limit of 100~MW. The 40~MW excess power, not consumed at bus~1, flows to bus~2 where the generator is built with only 50~MW. \\
%  
\begin{figure*}[h!]
    \begin{subfigure}[c]{.495\linewidth}
    % NOTE: Net EBE and Net AP lead to the same result here. Due to time limits just take the ebe figures
    \includegraphics[width=\linewidth]{example_allocation_bus1_net_ebe.png}
    \vspace{-40pt}
    \subcaption{}
    \label{fig:example_allocation_bus1}
    \end{subfigure}
    \begin{subfigure}[c]{.495\linewidth}
    \includegraphics[width=\linewidth]{example_allocation_bus2_net_ebe.png}
    \vspace{-40pt}
    \subcaption{}
    \label{fig:example_allocation_bus2}
    \end{subfigure}
    \caption{Power allocations of net power injections using the AP scheme, type \ref{net}\ref{ap}, in the example network, \cref{fig:example_network}. Consumers at for bus~1, figure (a), are, with 60~MW, totally supplied by the local generator and do not consume any imported power from bus 2. In contrast consumer at bus~2, figure (b), retrieve 40~MW from generator 1, induce a flow of 40~MW on the transmission line and consume 50~MW from the local generator.}
    \label{fig:example_allocation}
\end{figure*}
% 
\begin{figure}[h]
    \centering
    \includegraphics[width=\linewidth]{example_payoff_net_ebe.png}
    \caption{Full P2P cost allocation on the basis of power allocations shown in \cref{fig:example_allocation}. Consumers compensate OPEX and CAPEX of the generators they retrieve from, see \cref{fig:example_allocation}. As bus~1 is totally self-supplying, all it payment is assigned to the local generator.  As bus~2 imports power from bus~1 and thus induces a flow on line~1, it not only compensates local expenditures but also OPEX and CAPEX at bus~1 and CAPEX for the transmission.}
    \label{fig:example_payoff}
\end{figure}    
% 
% 
% 
\Cref{fig:example_allocation} shows the allocated transactions on the basis of allocation \ref{net}\ref{ap} for both buses 1~\&~2 separately. Note, the ``sum'' of the two figures give to the actual dispatch. The corresponding P2P payments are given in \cref{fig:example_payoff}.  

The left graph \cref{fig:example_allocation_bus1} shows that $d_1$ is with 60~MW totally supplied by the local production. Consequently consumers at bus~1 pay 3k~\euro~OPEX, which is the operational price of 50~\euro\, times the 60~MW. Further they pay 33k~\euro~ for the CAPEX at generator~1. Note that 3k~\euro~ of these account for the scarcity imposed buy the upper expansion limit $\capacitygenerationupper$. The rest makes up 60\% of the CAPEX spent at generator~1, exactly the share of power allocated to $d_1$. Consumers at bus~1 don't pay any transmission CAPEX as no flow is associated with there demand. \\
The right graph \cref{fig:example_allocation_bus2} shows the power allocations to $d_2$. We see that 50~MW are self-supplied whereas the remaining 40~MW are imported from bus~1. Thus, consumers have the pay for the local OPEX and CAPEX as well as the expenditures at bus~1 and the transmission system. As the capacity at generator~2 does not hit the expansion limit $\capacitygenerationupper$, no scarcity cost are assigned to it. The allocated CAPEX exactly matches the actual CAPEX of generator~2. However in the case of generator~1, 2\kk of the 22\kk CAPEX allocation are associated to the scarcity cost. The payed congestion revenue of 4\kk is exactly the CAPEX of the transmission line (zero-profit condition).   
% 
The sum of all values in the payoff matrix in \cref{fig:example_payoff} yield $\totalcost - \scarcitycost$, the total system cost minus the scarcity cost (which in turn is negative).
The sum of a column  yields the total revenue per the asset $i$. These values match their overall spending plus the cost of scarcity. The sum a row returns the total payment of consumers at bus $n$. For example the sum of payments of consumers at bus~1 is 36k~\euro. This is exactly the electricity price of 600~\euro/MW times the consumption of 60~MW, $\lambda_1 d_1$. \\
% 

The fact that OPEX and CAPEX allocations are proportional to each other results from the fact of optimizing one time step only. This coherence breaks for for larger optimization problems with multiple time steps. Then CAPEX allocation takes effect only for time steps in which one or more of the capacity constraints  \cref{eq:upper_generation_capacity_constraint,eq:lower_flow_capacity_constraint,eq:upper_flow_capacity_constraint} become binding.  \\
The example shows that allocating net power injection leads to intuitive and transparent  cost allocations. 

% In contrast the to equivalent allocation of gross power production, netting out injections for each bus leads to less P2P payments. The resulting payment given in \cref{fig:example_payoff_net_ebe} builds on the allocated power flow shown \cref{fig:example_allocation_net_ebe} in \cref{sec:example_plots}. As bus~2 does not produce excess power, none of its power production is assigned to bus~1 and thus no payment of bus~1 to bus~2 allocated. Neither has bus~1 to pay fee to the transmission system as it only exports power. So, consumers at bus~1 pay to its local generator. Bus~2 in contrast bear all CAPEX for the transmission system as well as CAPEX and OPEX for generators at bus~1. 
% % 
% Again the cumulative payments per bus meet the nodal spending $\lmp \, \demand$. The cumulative revenues per generator and transmission line meet the all CAPEX and OPEX. Note this gives the same result as when allocating net injection with the Average Participation \ref{net}\ref{ap}.



\section{Application Case}

We showcase the behavior of the cost allocation in a more complex system, by applying it to an cost-optimized German power system model with 50 nodes and one year time span with hourly resolution. The model builds on the PyPSA-EUR workflow \cite{horsch_jonas_pypsa-eur_2020} with technical details and assumptions reported in \cite{horsch_pypsa-eur_2018}. 

We follow a brownfield approach where transmission lines can be expanded starting from today's capacity values, originally retrieved from the ENTSO-E Transmission System Map \cite{entso-e_entso-e_nodate}. Pre-installed wind and solar generation capacity totals of the year 2017 were distributed in proportion to the average power potential at each site excluding those with an average capacity factor of 10\%. Further, wind and solar capacity expansion are limited by land use restriction. These consider agriculture, urban, forested and protected areas based on the CORINE and NATURA2000 database \cite{corine2012,natura2000}. Pumped Hydro Storages (PHS) and Run-of-River power plants are fixed to today's capacities with no more expansion allowed. Additionally, unlimited expansion of batteries and H$_{2}$-storages and Open-Cycle Gas Turbines (OCGT) are allowed at each node. 
% 
\begin{figure*}[t]
    \centering
    \includegraphics[width=\linewidth]{de50bf/network}
    \caption{Brownfield optimization of the German power system. The left side shows existent renewable capacities, matching the total capacity for the year 2017, which serve as lower capacity limits for the optimization. The right side shows the capacity expansion of renewables as well as installation of backup gas power plants. The effective CO$_2$ price is set to 120 \euro per tonne CO$_2$ emission.}
    \label{fig:network}
\end{figure*}
% 
\begin{figure}
    \centering
    \includegraphics[width=\linewidth]{de50bf/average_price}
    \caption{Average electricity price $\averagelmp$ per region as a result from the optimization of the German power system. Regions in the middle and south of Germany have high prices whereas electricity in the North with a strong wind, transmission and OCGT infrastructure is cheaper.}
    \label{fig:average_price}
\end{figure}
We impose a effective carbon price of 120 \euro\, per tonne-CO$_{2}$ which, for OCGT, adds an effective price of 55 \euro/\megawatthour (assuming a gross emission of 180~kg/MWh and an efficiency of 39\%). All cost assumptions on operational costs $o_i$ and annualized capital cost $c_i$ are summarized in detail in \cref{tab:cost_assumptions}. 

The optimized network is shown in  \cref{fig:network}. On the left we find the lower capacity bounds for renewable generators and transmission infrastructure, on the right the capacity expansion for generation, storage and transmission. The optimization expands solar capacities in the south, onshore and offshore wind in the upper north and most west. Open-Cycle Gas Turbines (OCGT) are build within the broad middle of the network. Transmission lines are amplified in along the North-South axis, including one large DC link, associated with the German S\"ud-Link, leading from the coastal region to the South-West. 
The total annualized cost of the power system roughly sums up to 42 billion \euro.
% 

\Cref{fig:average_price} displays the average electricity price $\lmp$ per region. We observe a relatively strong gradient from south (at roughly 92 \euro/MWh) to north (80 \euro/MWh). Regions with with little pre-installed capacity and capacity expansion, especially for renewables, tend to have higher prices. The node with the lowest LMP in the upper North-West, stands out through high pre-installed offshore capacities. \\
 
The costs are allocated according to type \ref{net}\ref{ap}, DC-lines are incorporated into the PTDF in \cref{eq:power_flow_equation} using the method described in \cite{hofmann_flow_2020-1}.
In \cref{fig:total_cost} we show the total of all ``allocatable'' costs $\cost'$. Due to the design constraints, \cref{sec:design_constraints}, and the role of storage units, \cref{sec:storages}, these are different from the original cost terms $\cost$. The figure also includes costs for scarcity $\scarcitycost$ and subsidy  $\subsidycost$. Note that the sum of all contributions in \cref{fig:total_cost} equals the total cost $\totalcost$.  


The largest proportion of the payments is associated with CAPEX for generators, transmission system and storage units in decreasing order. 
It turns out that  mostly areas with high CAPEX allocation are the ones with low average LMP. In particular, this counts for the coastal regions where strong offshore and onshore investments are taken, see \cref{fig:capex_price}. Several regions inside the country also pay comparably high share to CAPEX, however these are allocated to non-local assets. We further observe that CAPEX for OCGT is allocated very evenly among all consumers. Again, these payments benefit primarily regions nearby strong wind capacities in the North.  

Together with the emission cost $\emissioncost$, the total OPEX $\opex$ amount around 16 billion \euro. As to expect, 99.97\% are dedicated to OCGT alone. For a detailed regional distribution of payments per MWh and the resulting revenues for generator, see \cref{fig:opex_price,fig:emission_cost}. Compared to the CAPEX allocation for OCGT, the OPEX distributes in a more local manner. This results from the fact that during ``standard'' demand peaks, it is rather local OCGT generators which serve as backup generators. In the average power mix per region, see \cref{fig:power_mix}, we observe that regions with strong OCGT capacities predominately have high shares of OCGT power. The OPEX allocations for renewables play with 0.2 \euro/MWh in average an inferior role. 

The negative segment in \cref{fig:total_cost} is associated with the scarcity costs $\scarcitycost$, caused by land use constraints for renewables and the transmission expansion limit. These sum up approximately to 7.5 b\euro. Note again, that according to \cref{eq:scarcitycost} this term is negative and is part of the allocated CAPEX payments. It translates to the cost that consumers pay ``too much'' for assets limited in their capacity expansion. In the real world this money would be used to pay for augmented land costs or civic participation in the dedicated areas. In \cref{fig:scarcity_price} we give a detailed insight of how the scarcity cost manifest in the average price $\left<\lmp\right>_t$. The scarcity for wind and solar are relatively low and impact the average price by around 2\euro/MWh. They mostly affect regions with strong investments and those in close vicinity. Remarkably, the scarcity cost per MWh for run-of-river power plants amounts up to 16\euro. This high impact is due to the steady power potential from run-off water and the strong limitation of capacity expansion. However, as these power plants in particular are already amortized and scarcity cost should be reconsidered and removed from a final cost allocation.    
The scarcity cost for the transmission system, resulting from the limited transmission volume expansion of 25\% mostly influence regions in the middle of Germany, where in average 8\euro/MWh account for transmission scarcity, see \cref{fig:branch_scarcity_price}. This illustrates the strong impact that the line volume constraint has on the optimization. Regions in the North are not touched by the constraint. 

The last cost term in \cref{fig:total_cost}, is  caused by lower capacity constraints for pre-existing assets. These violate the optimal design and are not compensated through the nodal payments $\demand \, \lmp$. 

\begin{figure}
    \centering
    \includegraphics[width=0.75\linewidth]{de50bf/total_costs}
    \caption{Total allocated payments of the system. }
    \label{fig:total_cost}
\end{figure}




\begin{figure*}
    \centering
    \begin{subfigure}[c]{.6\linewidth}
    \includegraphics[width=\linewidth]{de50bf/ptpf_net_to_lowest-lmp}
    \end{subfigure}
    \hspace{-.2151\linewidth}
    \begin{subfigure}[c]{.6\linewidth}
    \includegraphics[width=\linewidth]{de50bf/ptpf_net_to_highest-lmp}
    \end{subfigure}
    \caption{Comparison of payments of the node with the \textbf{lowest LMP (left)} and the node with the \textbf{highest LMP (right)}. The region of the paying bus is colored in dark blue. The circles indicate where to which bus and technology combined OPEX and CAPEX payments. Further the thickness of the lines indicates the dedicated amount of payments. The cheap prices in the North...  }
    \label{fig:direct-allocation}
\end{figure*}
 
\begin{figure}
    \centering
    \includegraphics[width=\linewidth]{de50bf/locality_ptpf_net}
    \caption{Average distance between payer and receiver for different technologies and shares of the total production.}
    \label{fig:locality}
\end{figure}


\section{Conclusion}

In this work we presented a new cost-allocation scheme based on peer-to-peer dispatch allocations from asset to consumer. Starting from an optimized long-term investment model, we were able to decompose single cost terms into contributions of single consumers. For three typical classes of assets, namely generators, transmission lines and storage units, we showed how operational prices and shadow prices must be weighted with the dispatch allocation in order to allocate all system costs. Further we highlighted the impact of additional design constraints, that is constraints with a non-zero constant on the right hand side, such as lower and upper capacity expansion limits. These alter the locational marginal price and therefore distort the cost allocations. In the case of upper capacity expansion limits, this leads to an additional charge for consumers which have to compensate for \eg land use restrictions. For lower capacity investment bounds, the cost have to be subsidized by the overall system operator or the government, as those cannot be allocated.

\clearpage
\appendix

\section{Network Optimization}

\renewcommand\theequation{\thesection.\arabic{equation}}
\setcounter{equation}{0}

\renewcommand\thefigure{\thesection.\arabic{figure}}    
\setcounter{figure}{0}    

\subsection{LMP from Optimization}
The nodal balance constraint ensures that the amount of power that flows into a bus equals the power that flows out of a bus, thus reflects the Kirchhoff Current Law (KCL). Alternatively, we can the demand $\demand$ has to be supplied by the attached assets,  
\begin{align}
    \sum_i \incidenceasset \state  &=  \demand 
    % \sum_l \incidence \, \flow  - \nodalgeneration + \nodaldemand &= 0
     \resultsin{\lmp} \Forall{n,t}
    \label[constraint]{eq:nodal_balance_lin}
\end{align}
where $\incidenceasset$ is +1 if $i$ is attached to $n$ and a positive operation $\state$ delivers power to $n$, -1 if is attached to $n$ and a positive operation retrieves power from $n$ and zero else (note that for lines this results in the negative of the conventional Incidence Matrix).  
The shadow price of the nodal balance constraint mirrors the Locational Marginal Prizes (LMP) $\lmp$ per bus and time step. In a power market this is the \euro/\megawatthour-price which a consumer has to pay.\\

\subsection{Full Lagrangian}
\label{sec:full_lagrangian}
The Lagrangian for the investment model can be condensed to the following expression

\begin{align}
\notag
 \lagrangian\left(\state, \capacity, \lmp, \mu_j \right) =& 
 \;\;\; \sum_{i,t} o_i \state + \sum_{i} c_i \capacity  \\
\notag
&+  \sum_{n,t} \lmp \left( \demand - \sum_i \incidenceasset \, \state \right) \\
&+ \sum_j \mu_j \, h_j \left(\state, \capacity \right)
\label{eq:full_lagrangian}
\end{align}
where $h_j \left(\state, \capacity \right)$ denotes all inequality constraints attached to $\state$ and $\capacity$.
% \begin{align}
% \notag
% & \lagrangian\left(\generation, \flow, \capacitygeneration, \capacityflow, \boldsymbol{\lambda}, \boldsymbol{\mu} \right)   =   \\  
% \notag
% & \;\;\; \sum_{n,s} \capitalpricegeneration \capacitygeneration + \sum_{n, s, t} \operationalpricegeneration \generation + \sum_{\ell} \capitalpriceflow \, \capacityflow  \\
% \notag
% &+ \sum_{n,t} \lmp \left(\sum_\ell \incidence \, \flow  - \sum_s \incidencegenerator\, \generation +  \demand  \right)  \\ 
% \notag
% &+ \sum_{\ell,c,t} \cycleprice \, \cycle \, \impedance \, \flow  \\
% \notag
% &+ \sum_{n,s,t} \muuppergeneration \left( \generation - \generationpotential \capacitygeneration \right)  - \sum_{n,s,t} \mulowergeneration \generation  \\
% &+ \sum_{\ell,t} \muupperflow \left( \flow - \capacityflow \right) - \sum_{\ell,t} \mulowerflow \left( \flow + \capacityflow \right)     
% \label{eq:full_lagrangian}
% \end{align}
% 
% where $\boldsymbol{\lambda} = \left\lbrace \lmp, \cycleprice \right\rbrace $ and $\boldsymbol{\mu} = \left\lbrace \muuppergeneration, \mulowergeneration, \muupperflow, \mulowerflow \right\rbrace $ denote the set of related KKT variables. 
% 
In order to impose the Kirchhoff Voltage Law (KVL) for the linearized AC flow, the term 
\begin{align}
    \sum_{\ell,c,t} \cycleprice \, \cycle \, \impedance \, \flow 
\end{align}
can be added to $\lagrangian$, with $\impedance$ denoting the line's impedance and $\cycle$ being 1 if $\ell$ is part of the cycle $c$ and zero otherwise.

The global maximum of the Lagrangian requires stationarity with respect to all variables:
\begin{align}
    \pdv{\lagrangian}{\state} = \pdv{\lagrangian}{\capacity} = 0    
\end{align} 



\subsection{Zero Profit Generation}
\label{sec:zero_profit_generation}
\Cref{eq:upper_generation_capacity_constraint,eq:lower_generation_capacity_constraint}, which yield the KKT variables $\muuppergeneration$ and $\mulowergeneration$, imply the complementary slackness,
\begin{align}
\muuppergeneration \left( \generation - \generationpotential \, \capacitygeneration \right)  &= 0  \Forall{n,s,t} 
\label{eq:complementary_slackness_upper_generation} \\
\mulowergeneration  \, \generation &= 0 \Forall{n,s,t}
\label{eq:complementary_slackness_lower_generation} 
\end{align}


The stationarity of the generation capacity variable leads to 
\begin{align}
\pdv{\lagrangian}{\capacitygeneration}  = 0 \,\, \rightarrow \,\, 
\capitalpricegeneration =  \sum_t \muuppergeneration \, \generationpotential  \Forall{n,s}
\label{eq:capex_generation_duality}
\end{align}
and the stationarity of the generation to 
\begin{align}
\pdv{\lagrangian}{\generation} &= 0 \,\, \rightarrow \,\,  
\operationalpricegeneration =  \incidencegenerator \, \lmp - \muuppergeneration + \mulowergeneration \Forall{n,s} \label{eq:opex_duality}
\end{align}


Multiplying both sides of \cref{eq:capex_generation_duality} with $\capacitygeneration$ and using \cref{eq:complementary_slackness_upper_generation} leads to 
\begin{align}
 \capitalpricegeneration \, \capacitygeneration  = \sum_t \muuppergeneration \, \generation 
 \label{eq:capital_price_generation_sum}
\end{align}
The zero-profit rule for generators is obtained by multiplying \cref{eq:opex_duality} with $\generation$ and using \cref{eq:complementary_slackness_lower_generation,eq:capital_price_generation_sum} which results in 
\begin{align}
  \capitalpricegeneration \, \capacitygeneration + \sum_t \operationalpricegeneration \generation = \sum_t \lmp \incidencegenerator \generation
\end{align}
It states that over the whole time span, all OPEX and CAPEX for generator $s$ (left hand side) are payed back by its revenue (right hand side).

\subsection{Zero Profit Transmission System}
\label{sec:zero_profit_flow}

The yielding KKT variables $\muupperflow$ and $\mulowerflow$ are only non-zero if $\flow$ is limited by the transmission capacity in positive or negative direction, i.e. \cref{eq:upper_flow_capacity_constraint} or \cref{eq:lower_flow_capacity_constraint} are binding. For flows below the thermal limit, the complementary slackness 
\begin{align}
\muupperflow \left( \flow - \capacityflow \right)  &= 0 \Forall{\ell,t}
\label{eq:complementary_slackness_upper_flow} \\
\mulowerflow \left( \flow - \capacityflow \right) &=  0 \Forall{\ell,t}
\label{eq:complementary_slackness_lower_flow} 
\end{align}
sets the respective KKT to zero. 

The stationarity of the transmission capacity to
\begin{align}
\pdv{\lagrangian}{\capacityflow} = 0 \,\, \rightarrow \,\, 
\capitalpriceflow =  \sum_t \left( \muupperflow - \mulowerflow \right) \Forall{\ell}
\label{eq:capexFlow_duality}
\end{align}
and the stationarity with respect to the flow to
\begin{align}
    0 &= \pdv{\lagrangian}{\flow}  \\ 
    0 &= - \sum_n \incidence \lmp  + \cycleprice \cycle \impedance  - \muupperflow + \mulowerflow \Forall{n,s} \label{eq:opex_flow_duality}
\end{align}
    
    
When multiplying \cref{eq:capexFlow_duality} with $\capacityflow$ and using the complementary slackness \cref{eq:complementary_slackness_upper_flow,eq:complementary_slackness_lower_flow} we obtain 
\begin{align}
 \capitalpriceflow \, \capacityflow = \sum_t \left( \muupperflow - \mulowerflow \right)  \, \flow
 \label{eq:capital_price_flow_sum}
\end{align}
Again we can use this to formulate the zero-profit rule for transmission lines. We multiply \cref{eq:opex_flow_duality} with $\flow$, which finally leads us to 
\begin{align}
\capitalpriceflow \, \capacityflow = - \sum_n \incidence\, \lmp\, \flow + \cycleprice\, \cycle\, \impedance\, \flow 
\end{align}
% Watch out Incidence = - ordinary Incidence !!
It states that the congestion revenue of a line (first term right hand side) reduced by the cost for cycle constraint exactly matches its CAPEX. 


\subsection{Zero Profit Storage Units}
\label{sec:zero_profit_storage_units}

For an simplified storage model, the upper capacity $\capacitystorage$ limits the discharging dispatch $\storagedispatch$, the storing power $\storagecharge$ and state of charge $\storagesoc$ of a storage unit $r$ by 
\begin{align}
    \storagedispatch - \capacitystorage &\le 0 \Forall{r,t} \resultsin{\muupperstoragedispatch} \\
    \storagecharge - \capacitystorage &\le 0 \Forall{r,t} \resultsin{\muupperstoragecharge} \\
    \storagesoc - h_r \, \capacitystorage &\le 0 \Forall{r,t} \resultsin{\muupperstoragesoc}
\end{align}
where we assume a fixed ratio between dispatch and storage capacity of $h_r$. 
% From stationarity we obtain 
% \begin{align}
%     \notag
%     \pdv{\lagrangian}{\capacitystorage} &= 0 \\
%     \capitalpricestorage &= \sum_t \left( \muupperstoragedispatch + \muupperstoragecharge + h_r \muupperstoragesoc  \right)
% \end{align}
The state of charge must be consistent throughout every time step according to what is dispatched and stored, 
\begin{align}
    \notag
    \storagesoc - \efficiencysoc \storageprevioussoc - \efficiencycharge \storagecharge &+ (\efficiencydispatch)^{-1} \storagedispatch = 0 \\
    &\resultsin{\mustateofcharge} \Forall{r,t}
\end{align}



We use the result of Appendix B.3 in \cite{brown_decreasing_2020} which shows that a storage recovers its capital (and operational) costs from aligning dispatch and charging to the LMP, thus 
\begin{align}
    \sum_t \operationalpricestorage \, \storagedispatch + \capitalpricestorage \, \capacitystorage = \sum_t \lmp \incidencestorage \left(\storagedispatch - \storagecharge \right) 
\end{align}
The stationarity of the dispatched power leads us to  
\begin{align}
    \notag
    \pdv{\lagrangian}{\storagedispatch} &= 0 \\
    \operationalpricestorage -  \lmp \, \incidencestorage - \mulowerstoragedispatch + \muupperstoragedispatch + (\efficiencydispatch )^{-1} \mustateofcharge &= 0  
    \label{eq:stationarity_storagedispatch}
\end{align}
which we  can use to define the revenue which compensates the CAPEX at $r$, 
\begin{align}
    \notag
    \capitalpricestorage \, \capacitystorage = \sum_t \left(\muupperstoragedispatch - \mulowerstoragedispatch  + (\efficiencydispatch )^{-1} \mustateofcharge \right) \storagedispatch \\
    - \sum_t \lmp \incidencestorage  \storagecharge \Forall{r} 
\end{align}

% With this constraint we can derive the following stationarities:
% \begin{align}
%     \notag
%     \pdv{\lagrangian}{\storagedispatch} &= 0 \\
%     \operationalpricestorage + \lmp + \mulowerstoragedispatch - \muupperstoragedispatch - (\efficiencydispatch )^{-1} \mustateofcharge &= 0  
%     \label{eq:stationarity_storagedispatch}
% \end{align}
% \begin{align}
%     \notag
%     \pdv{\lagrangian}{\storagecharge} &= 0 \\
%     - \lmp + \mulowerstoragecharge - \muupperstoragecharge + \efficiencycharge \mustateofcharge &= 0 
%     \label{eq:stationarity_storagecharge}
% \end{align}
% \begin{align}
%     \notag
%     \pdv{\lagrangian}{\storagesoc} &= 0 \\
%     \mulowerstoragesoc - \muupperstoragesoc - \mustateofcharge + \efficiencysoc \munextstateofcharge &= 0 
%     \label{eq:stationarity_storagesoc}
% \end{align}



\subsection{Emission Constraint}

Imposing an additional CO$_2$ constraint limiting the total emission to K,  
\begin{align}
\sum_{n,s,t} \emission \, \generation \le \text{K} \resultsin{\emissionprice} 
\label[constraint]{eq:co2_constraint}
\end{align}
with $\emission$ being the emission factor in tonne-CO$_2$ per \megawatthour, returns an effective CO$_2$ price $\emissionprice$ in \euro/tonne-CO$_2$. 
% The CO$_2$ price shifts the right hand side of the non-profit relation for generators \cref{eq:non_profit_generator} to
% 
% \begin{align}
% \capitalpricegeneration \, \capacitygeneration + \sum_{t} \operationalpricegeneration \, \generation &= \sum_{t} \left( \lmp - \emission \, \emissionprice \right)  \, \generation \Forall{n,s} 
% \label{eq:non_profit_generator_emission}
% \end{align}
% This shows nicely the duality for exchanging the CO$_2$ \cref{eq:co2_constraint} for a shifted OPEX which includes the CO$_2$ costs
As shown in ... the constraint can be translated in a dual price which shift the operational price per generator
\begin{align}
\operationalpricegeneration \rightarrow \operationalpricegeneration + \emission \, \emissionprice \label[relation]{eq:shift_opex_by_emission_cost}
\end{align}



% \subsection{Brownfield Optimization and Capacity Restrictions}

% Constraining the capacities $\capacitygeneration$  for a subset $S$ of generators to lower or upper limits in the form of
% \begin{align}
%     \capacitygenerationlower - \capacitygeneration \le 0 \resultsin{\mulowergenerationnom} \hpad \forall{s \in S} \label{eq:capacityexpansionminimum}\\
%     \capacitygeneration - \capacitygenerationupper \le 0 \resultsin{\muuppergenerationnom} \hpad \forall{s \in S}
% \label{eq:capacityexpansionmaximum}
% \end{align}
% alters the objective value as soon as one of those become bounding. 
% The complementary slackness for both are 
% \begin{align}
%     \muuppergenerationnom \left( \capacitygeneration - \capacitygenerationupper \right) =0 \Forall{s \in S} \\
%     \mulowergenerationnom \left( \capacitygeneration - \capacitygenerationlower \right) =0 \Forall{s \in S}
% \end{align}


% The CAPEX paybacks for generators and transmission lines
% \cref{eq:capital_price_generation_sum,eq:capital_price_flow_sum} change to 
% \begin{align}
% \pdv{\lagrangian}{\capacitygeneration}  = 0 \,\, \rightarrow \,\, \\
% \capitalpricegeneration =  \sum_t \muuppergeneration \, \generationpotential + \mulowergenerationnom - \muuppergenerationnom \Forall{s \in S}
% \label{eq:capex_generation_duality_bf}
% \end{align}
% for generators. Multiplying \cref{eq:capex_generation_duality_bf} by $ \capacitygeneration$ leads us to 
% \begin{align}
%  \capexgeneration_s - \remainingcost_s=  \sum_t \muuppergeneration \, \generation \Forall{n,s \in S}
%  \label{eq:capex_generation_duality_bf2}
% \end{align}
% where we define the cost resulting from the capacity expansion limits as
% \begin{align}
%     \remainingcost_s = \left(\mulowergenerationnom - \muuppergenerationnom \right) \capacitygeneration
%     \Forall{s}
% \end{align}
% As $\mulowergenerationnom \ge 0$ and $\muuppergenerationnom \ge 0$, the latter can be a net positive or net negative cost term.

% Multiplying \cref{eq:opex_duality}, which is still valid, with $\generation$ and inserting \cref{eq:capex_generation_duality_bf2} will bring us to the zero profit rule for generators with capacity expansion limits, 
% \begin{align}
%     \capexgeneration_s + \opexgeneration_s  - \remainingcost_s = \sum_t \lmp \incidencegenerator \generation \Forall{s \in S}
% \end{align}
% We see that the revenue of generator $s$ will not exactly pay back its full CAPEX and OPEX. The exogenous constraint shifts the zero-profit equations such that some of the expenditures for $s \in S$ cannot directly be allocated to consumers. In order to take this effect into account for the cost allocation, we update the shadow price $\muuppergeneration$ in \cref{tab:cost_allocation_map}, to 

% \begin{align}
%     \muuppergeneration \rightarrow \dfrac{\capitalpricegeneration}{\capitalpricegeneration - \mulowergenerationnom + \muuppergenerationnom}\, \muuppergeneration
% \end{align}
% By doing so we ensure that the allocated CAPEX sum up to the actual CAPEX at generator $s$, thus 
% \begin{align}
%     \capexgeneration_s = \sum_{n,t} \allocatecapexgeneration \Forall{s \in S}
% \end{align}



% Doing likewise with a subset $L$ of transmission lines,
% \begin{align}
% \capacityflow \ge \capacityflowLower \resultsin{\mulowerflownom} \hpad \forall{\ell \in \textit{L}} \\
% \capacityflow \le \capacityflowUpper \resultsin{\muupperflownom} \hpad \forall{\ell \in \textit{L}}
% \end{align}
% results in a shifted zero-profit rule for transmission line in the form 
% \begin{align}
%     \capitalpriceflow \, \capacityflow - \remainingcost_\ell = \sum_{n,t}\allocatecapexflow \Forall{\ell \in L}
% \end{align}
% where $\remainingcost_\ell$ is given by 
% \begin{align}
%     \remainingcost_\ell = \left(\mulowerflownom - \muupperflownom \right) \, \capacityflow
%     \Forall{\ell \in L}
% \end{align}

% % \begin{align}
% %     \muupperflownom \left( \capacityflow - \capacityflowUpper \right) =0 \\
% %     \mulowerflownom \left( \capacityflow - \capacityflowLower \right) =0 \\
% % \end{align}

% % \begin{align}
% %     \pdv{\lagrangian}{\capacityflow}  = 0 \,\, \rightarrow \,\, \\
% %     \capitalpriceflow =  \sum_t \left( \muupperflow - \mulowerflow \right)  + \mulowerflownom - \muupperflownom \Forall{n,s}
% %     \label{eq:capex_flow_duality_bf}
% %     \end{align}
% %     for transmission lines.
% % Multiplying \cref{eq:capex_generation_duality_bf} by the amount of expansion $\left( \capacitygeneration - \capacitygenerationlower \right) $
% % \begin{align}
% %   \capitalpricegeneration \, \capacitygeneration^{exp} =  \sum_t \muuppergeneration \, \left( \generation - \generationpotential \, \capacitygenerationlower \right) + \muuppergenerationnom \left( \capacitygenerationupper - \capacitygenerationlower \right) \Forall{n,s}
% % \end{align}






\section{Allocation Schemes}
\subsection{Allocating Gross Injections with EBE}
\label{sec:gross_ebe}

The allocation of gross generation to demands $\demand$ is straightforwardly obtained by a proportional distribution of the generation, \ie

\begin{align}
    \allocategeneration = \dfrac{\generation}{\sum_s \generation} \demand 
\end{align}


\subsection{Allocating Net Injections with EBE}
\label{sec:net_ebe}

Allocating net power injections using the EBE methods leads to the same result as the Marginal Participation (MP) \cite{rudnick_marginal_1995}  algorithm when allocating to consumers only, see \cite{hofmann_flow_2020-1} for further insight. We calculate it by setting 
\begin{align}
\allocatepeer &=  \delta_{m,n}\,\selfconsumption[m] + \gamma_t \, \netconsumption  \, \netproduction[m]
\label{eq:mp_slack}
\end{align}
where 
\begin{itemize}
%  \item $\injection = \left( \nodalgeneration - \nodaldemand \right) $ denotes the nodal injection
\item $\netproduction = \text{min}\left( \nodalgeneration - \nodaldemand , 0 \right) $ denotes the nodal net production 
\item $\netconsumption = \text{min}\left( \nodaldemand  - \nodalgeneration, 0 \right)$ denotes the nodal net consumption
\item $\selfconsumption = \text{min}\left( \netproduction, \netconsumption \right)$ the denotes  nodal self-consumption. That is the power generated and at the same time consumed at node $n$ and 
\item $\gamma_t = \left( \sum_n \netproduction\right) ^{-1} = \left( \sum_n \netconsumption\right) ^{-1}$ is the inverse of the total injected/extracted power at time $t$.
\end{itemize}

The allocation $\allocategeneration$ from generator $s$ to $n$, is given by multiplying $\allocatepeer$ with the nodal share $\generation / \nodalgeneration$.


\subsection{Allocating Net Power using AP}
\label{sec:net_ap}

\newcommand{\incidenceM}{K}
\newcommand{\flowM}{f}
\newcommand{\injectionM}{p}
\newcommand{\slackM}{k}
\newcommand{\DirectedIncidence}{\mathcal{K}}
\newcommand{\InverseAPInjection}{\mathcal{J}}
\newcommand\diag[1]{\operatorname{diag}\left(#1\right)}


Allocating net injections using the AP method is derived from \cite{achayuthakan_electricity_2010}. In a lossless network the downstream and upstream formulations result in the same P2P allocation which is why we restrict ourselves to the downstream formulation only. In a first step we define a time-dependent auxiliary matrix $\InverseAPInjection_t$ which is the inverse of the $N\times N$ with directed power flow $m \rightarrow n$ at entry $(m, n)$ for $m \ne n$ and the total flow passing node $m$ at entry $\left( m, m\right)$ at time step $t$. Mathematically this translates to


\begin{align}
\InverseAPInjection_t = \left( \diag{\injectionM^+} + \DirectedIncidence^- \diag{\flowM} \, \incidenceM \right)_t^{-1} 
\end{align}
where $\DirectedIncidence^-$ is the negative part of the directed Incidence matrix $\DirectedIncidence_{n,\ell} = \text{sign}\left( \flow \right)  \incidence$. Then the distributed slack for time step $t$ is given by
\begin{align}
\allocatepeer = \InverseAPInjection_{m,n,t} \, \netproduction[m] \, \netconsumption
\end{align}

\subsection{Allocating Gross Power using AP}
\label{sec:gross_ap}

We use the same allocation as in \cref{sec:net_ap} but replace the net nodal production $\netproduction$ by the gross nodal production $\nodalgeneration$ which leads to  
\begin{align}
\InverseAPInjection_t = \left( \diag{g} + \DirectedIncidence^- \diag{\flowM} \, \incidenceM \right)_t^{-1} 
\end{align}
The distributed slack is for time step $t$ is then given by
\begin{align}
\allocategeneration[s \rightarrow m] = \InverseAPInjection_{m,n} \, \generation \, \demand
\end{align}

% 
\section{Working Example}
The following figures contain more detailed information about the peer-to-peer cost allocation. The cost or prices payed by consumers are indicated by the region color. The dedicated revenue is displayed in proportion to the size of cycles (for assets attached to buses) or to the thickness of transmission branches.    
\begin{figure*}
    \centering
    \begin{subfigure}[c]{.49\linewidth}
        \includegraphics[width=\linewidth]{de50bf/maps_price_ptpf_net/one_port_investment_cost_total}
        \subcaption{All production and storage technologies}
        \label{fig:total_capex}
    \end{subfigure}
    \begin{subfigure}[c]{.49\linewidth}
        \includegraphics[width=\linewidth]{de50bf/maps_price_ptpf_net/by_carrier/onwind_one_port_investment_cost}
        \subcaption{Onshore Wind}
        \label{fig:onshore_capex}
    \end{subfigure}
    \begin{subfigure}[c]{.49\linewidth}
        \includegraphics[width=\linewidth]{de50bf/maps_price_ptpf_net/by_carrier/solar_one_port_investment_cost}
        \subcaption{Solar}
        \label{fig:solar_capex}
    \end{subfigure}
    \begin{subfigure}[c]{.49\linewidth}
        \includegraphics[width=\linewidth]{de50bf/maps_price_ptpf_net/by_carrier/OCGT_one_port_investment_cost}
        \subcaption{OCGT}
        \label{fig:ocgt_capex}
    \end{subfigure}
    \caption{Average \textbf{CAPEX allocation} per MWh, $\sum_t \allocatecapexgeneration / \sum_t \demand$ and $\sum_t \allocatecapexstorage / \sum_t \demand$, for all production and storage assets (a), onshore wind (b), solar (c) and OCGT (d). Average Allocated CAPEX per MWh within the regions are indicated by the color, the revenue per production asset is given by the size of the circles at the corresponding bus.}
    \label{fig:capex_price}
\end{figure*}

\begin{table*}[h]
    \centering
    \begin{tabular}{lllr}
\toprule
     &    & o [\euro/MWh] &  c [k\,\euro/MW]$^*$ \\
{} & carrier &               &                      \\
\midrule
Generator & Open-Cycle Gas &       120.718 &               47.235 \\
     & Offshore Wind (AC) &         0.015 &              204.689 \\
     & Offshore Wind (DC) &         0.015 &              230.532 \\
     & Onshore Wind &         0.015 &              109.296 \\
     & Run of river &               &              270.941 \\
     & Solar &          0.01 &               55.064 \\
Storage & Hydrogen Storage &               &              224.739 \\
     & Pumped Hydro &               &              160.627 \\
     & Battery Storage &               &              133.775 \\
Line & AC &               &                0.038 \\
     & DC &               &                0.070 \\
\bottomrule
\end{tabular}
    
    \caption{Operational and capital price assumptions for all type of assets used in the working example. The capital price for transmission lines are given in [k\,\euro/MW/km]. The cost assumptions are retrieved from the PyPSA-EUR model \cite{horsch_jonas_pypsa-eur_2020}.}
    \label{tab:cost_assumptions}
\end{table*} 

\begin{figure}
    \includegraphics[width=\linewidth]{de50bf/maps_price_ptpf_net/one_port_operational_cost_total}
    \caption{Average \textbf{OPEX allocation} per consumed MWh, $\sum_t \allocateopex/\sum_t \demand$.}
    \label{fig:opex_price}
\end{figure}


\begin{figure}
    \includegraphics[width=\linewidth]{de50bf/maps_price_ptpf_net/co2_cost_total}
    \caption{Average \textbf{allocated emission cost}, $\sum_t \allocateemissioncost$, per consumed MWh.}
    \label{fig:emission_cost}
\end{figure}


\begin{figure}
    \includegraphics[width=\linewidth]{de50bf/power_mix_ptpf_net}
    \caption{Average power mix per region.}
    \label{fig:power_mix}
\end{figure}

\begin{figure}
    \includegraphics[width=\linewidth]{de50bf/maps_scarcity_price_ptpf_net/branch_scarcity_cost_total}
    \caption{Average \textbf{allocated transmission scarcity cost} per consumed MWh. This scarcity cost results from the upper transmission expansion limit of 25\%.}
    \label{fig:branch_scarcity_price}
\end{figure}


\begin{figure}
    \includegraphics[width=\linewidth]{de50bf/subsidy}
    \caption{Total costs for subsidy $\subsidycost$ resulting from lower capacity expansion bounds (brownfield constraints).}
    \label{fig:subsidy}
\end{figure}



\begin{figure*}
    \centering
    \begin{subfigure}[c]{.49\linewidth}
        \includegraphics[width=\linewidth]{de50bf/maps_scarcity_price_ptpf_net/by_carrier/offwind-dc_generator_scarcity_cost}
        \subcaption{Offshore Wind}
        \label{fig:offwind-dc_generator_scarcity_cost}
    \end{subfigure}
    \begin{subfigure}[c]{.49\linewidth}
        \includegraphics[width=\linewidth]{de50bf/maps_scarcity_price_ptpf_net/by_carrier/onwind_generator_scarcity_cost}
        \subcaption{Onshore Wind}
        \label{fig:onwind_generator_scarcity_cost}
    \end{subfigure}
    \begin{subfigure}[c]{.49\linewidth}
        \includegraphics[width=\linewidth]{de50bf/maps_scarcity_price_ptpf_net/by_carrier/solar_generator_scarcity_cost}
        \subcaption{Solar}
        \label{fig:solar_generator_scarcity_cost}
    \end{subfigure}
    \begin{subfigure}[c]{.49\linewidth}
        \includegraphics[width=\linewidth]{de50bf/maps_scarcity_price_ptpf_net/by_carrier/ror_generator_scarcity_cost}
        \subcaption{Run-of-River}
        \label{fig:ror_generator_scarcity_cost}
    \end{subfigure}
    \caption{Average electricity price for \textbf{allocated scarcity cost}, $\sum_t \allocatescarcitycost$, due to technology resource limits, for offshore wind, onshore wind,  solar, run-of-river. Average Allocated Scarcity Cost per MWh within the regions are indicated by the color, the revenue per production asset is given by the size of the circles at the corresponding bus.}
    \label{fig:scarcity_price}
\end{figure*}





\clearpage
\printbibliography




\end{document}
