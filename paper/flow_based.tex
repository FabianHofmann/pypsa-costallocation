\documentclass[11pt,twocolumn]{article}
\usepackage{graphicx}
\usepackage[left=1.80cm, right=1.80cm, top=2.00cm, bottom=2.00cm]{geometry}
\usepackage{amsmath,bm}
\usepackage[colorlinks]{hyperref}
\usepackage[backend=biber]{biblatex}
\usepackage{eurosym}
\usepackage[dvipsnames]{xcolor}
\usepackage{subcaption}
\usepackage{enumitem} % for alphabetical enumeration 
\usepackage{accents}
\usepackage[capitalise]{cleveref}



\addbibresource{main.bib}
\graphicspath{{../figures/}{../figures/example/}}

\crefname{relation}{Rel.}{Rels.}
\creflabelformat{relation}{(#2#1#3)}
\crefname{constraint}{Constr.}{Constrs.}
\creflabelformat{constraint}{(#2#1#3)}

\setlength\parindent{8pt}

% style operators
\newcommand{\ie}{\textit{i.e.} }
\newcommand{\eg}{\textit{e.g.} }
\newcommand{\ubar}[1]{\underaccent{\bar}{#1}}
\newcommand{\note}[1]{\textcolor{Orange}{#1}}
\newcommand{\vpad}{\vspace{1mm}}
\newcommand{\hpad}{\hspace{15pt}}
\newcommand{\resultsin}[1]{\hspace{6pt} \bot  \hspace{6pt} #1}
\newcommand{\Forall}[1]{\hspace{10pt} \forall \,\, #1 }
\newcommand{\pdv}[2]{\frac{\partial #1}{\partial #2}}


% general symbols
\newcommand{\generation}{g_{s,t}}
\newcommand{\generationpotential}{\bar{g}_{s,t}}
\newcommand{\generationshare}[1][n]{\omega_{#1,s,t}}
\newcommand{\nodalgeneration}[1][n]{g_{#1,t}}
\newcommand{\capacitygeneration}{G_{s}}
\newcommand{\capacitygenerationupper}{\bar{G}_{s}}
\newcommand{\capacitygenerationlower}{\ubar{G}_{s}}
\newcommand{\capacityflow}{F_{\ell}}
\newcommand{\capacityflowUpper}{\bar{F}_{\ell}}
\newcommand{\capacityflowLower}{\ubar{F}_{\ell}}
\newcommand{\capitalpricegeneration}{c_{s}}
\newcommand{\capitalpriceflow}{c_{\ell}}
\newcommand{\operationalpricegeneration}{o_{s}}
\newcommand{\demand}[1][n]{d_{#1,t}}
\newcommand{\nodaldemand}[1][n]{d_{#1,t}}
\newcommand{\demandshare}[1][n]{\omega_{#1,a,t}}
\newcommand{\utility}{U_{n,a,t}}
\newcommand{\incidence}[1][n]{K_{#1,\ell}}
\newcommand{\incidencegenerator}[1][n]{K_{#1,s}}
\newcommand{\ptdf}[1][n]{H_{\ell,#1}}

\newcommand{\mulowergeneration}{\ubar{\mu}_{s,t}}
\newcommand{\muuppergeneration}{\bar{\mu}_{s,t}}
\newcommand{\muuppergenerationnom}{\bar{\mu}^\text{nom}_{s}}
\newcommand{\mulowergenerationnom}{\ubar{\mu}^\text{nom}_{s}}
\newcommand{\mulowerflow}{\ubar{\mu}_{\ell,t}}
\newcommand{\muupperflow}{\bar{\mu}_{\ell,t}}
\newcommand{\muupperflownom}{\bar{\mu}^\text{nom}_{\ell}}
\newcommand{\mulowerflownom}{\ubar{\mu}^\text{nom}_{\ell}}
\newcommand{\lmp}[1][n]{\lambda_{#1,t}}
\newcommand{\flow}{f_{\ell,t}}
\newcommand{\cycle}{C_{\ell,c}}
\newcommand{\impedance}{x_\ell}
\newcommand{\cycleprice}{\lambda_{c,t}}
\newcommand{\injection}{p_{n,t}}
\newcommand{\netconsumption}[1][n]{p^{-}_{#1,t}}
\newcommand{\netproduction}[1][n]{p^{+}_{#1,t}}
\newcommand{\selfconsumption}[1][n]{p^{\circ}_{#1,t}}

\newcommand{\totalnetconsumption}{p^{-}_{t}}
\newcommand{\totalnetproduction}{p^{+}_{t}}
\newcommand{\totalselfconsumption}{p^{\circ}_{t}}
\newcommand{\lagrangian}{\mathcal{L}}

% allocation quantities
\newcommand{\allocatepeer}[1][s \rightarrow n]{A_{#1,t}}
\newcommand{\allocateflow}[1][n]{A_{\ell,#1,t}}
\newcommand{\allocatetransaction}[1][s \rightarrow n]{A_{#1,\ell,t}}
\newcommand{\allocatecapexgeneration}[1][n]{\mathcal{C}^{G}_{#1,t}}
\newcommand{\allocatecapexflow}[1][n]{\mathcal{C}^{F}_{#1,t}}
\newcommand{\allocateopex}[1][n]{\mathcal{O}_{#1,t}}
\newcommand{\allocateemissioncost}[1][n]{\mathcal{E}_{#1,t}}

\newcommand{\emission}{e_{s}}
\newcommand{\emissionprice}{\mu_{\text{CO2}}}
\newcommand{\megawatthour}{MWh$_\text{el}$}
\newcommand{\totalcost}{\mathcal{TC}}
\newcommand{\cost}{\mathcal{C}}
\newcommand{\opexgeneration}{\mathcal{O}}
\newcommand{\opexflow}{\mathcal{O}^F}
\newcommand{\capexgeneration}{\mathcal{C}^G}
\newcommand{\capexflow}{\mathcal{C}^F}
\newcommand{\emissioncost}{\mathcal{E}}
\newcommand{\remainingcost}{\mathcal{R}}



\begin{document}


\title{Flow-Based Cost Allocation in Linear Optimized Power Systems are non-unique but localizable}
\author{F. Hofmann}

\maketitle

\begin{abstract}
% Maximizing the welfare of all market participants within a power system is a common approach in energy system modelling. It leads to perfectly scheduled operations of generators and ... 
\end{abstract}


\subsection*{Highlights}
\begin{itemize}
 \item The LMP always breaks down into contributions of different cost terms (OPEX, CAPEX, etc.) of the system.
 \item Breaking the LMP further down into contributions of components (generator, transmission line, etc.) requires assumptions on power flow assignments.
 \item Methods like Average Participation or Flow Based Market Coupling allow for localizing the P2P cost assignments and prioritize assignments within one market zone.
\end{itemize}


% \section{Introduction}



\subsubsection*{Nomenclature}

\begin{table}[h]
    \centering
    \begin{tabular}{ll}
        $\lmp$ & Locational Market Price at bus $n$ and  \\ & time step $t$ in \euro/MW \vpad \\
        $\demand$ & Electric demand per bus $n$, demand type $a$, \\ & time step $t$ in MW  \vpad \\
        $\generation$ & Electric generation of generator $s$, \\ & time step $t$  in MW \vpad \\
        $\flow$ & Active power flow on line $\ell$, \\ & time step $t$ in MW   \vpad \\
        $\operationalpricegeneration$ & Operational price in \euro/MW \vpad \\
        $\capitalpricegeneration$ & Capital Price in \euro/MW \vpad \\
        $\capitalpriceflow$  & Capital Price in \euro/MW for transmission \\ & capacity on line $\ell$  \vpad \\
        $\capacitygeneration$ & Generation capacity in MW \vpad \\
        $\capacityflow$ & Transmission capacity in MW \vpad \\
        $\incidence$ & Incidence matrix \vpad 
    \end{tabular}
\end{table}

\section{Economic Context}

In long-term operation and investment planning models, the total costs $\totalcost$ of a power system is the sum of mutliple cost terms. Typically, these include operational expenditures (OPEX) $\opexgeneration$ for generators, expenditures for emissions $\emissioncost$, capital expenditures (CAPEX) for generators $\capexgeneration$, CAPEX for the transmission system $\capexflow$ and possible other terms, \ie
\begin{align}
\totalcost &= \opexgeneration + \emissioncost +  \capexgeneration +  \capexflow + ...
\label{eq:total_cost_terms}
\end{align}
In turn, each single cost term on the right side $\mathcal{C} = \{\opexgeneration, \emissioncost, \capexgeneration, \capexflow, ...\}$ consists of cost associated to the asset $i$ in the system, 
\begin{align}
    \cost = \sum_i \cost_i
    \label{eq:asset_costs}
\end{align} 
For example the total OPEX, $\opexgeneration = \sum_s \opexgeneration_s$, is the combined OPEX of all generators $s$.


Minimizing the total cost $\totalcost$  of a network, which translates to maximizing the total welfare, will determine the single expenditure $\cost_i$ dedicated to asset $i$.  with maximized welfare. Then, the Locational Marginal Price (LMP) describes the price for an incremental increase of electricity demand $\demand$ at node $n$. It is given by the derivative of the total system cost $\totalcost$ with respect to the local demand $\demand$
\begin{align}
\lmp = \pdv{\,\totalcost}{\,\demand}
\label{eq:total_cost_derivative}
\end{align}
% 
\begin{figure}[h]
\centering
\includegraphics[width=.8\linewidth]{price_decomposition.png}
\caption{Schematic decomposition of the Locational Market Price $\lmp$. In power system model with optimal long-term operation and planning, the total system costs $\totalcost$ split into different cost terms, \ie OPEX and CAPEX for production and transmission and possibly other expenditures. }
\label{fig:price_decomposition}
\end{figure}
% 
This leads to a nodal pricing where over the span of optimized timesteps $t$, the system costs are partially or totally payed back by the consumers 
\begin{align}
\totalcost - \remainingcost &=  \sum_{n,a,t} \lmp \, \demand
\label{eq:total_cost_sum}
\end{align}
depending on the costs $\remainingcost$ which are independent of the nodal demand  
\begin{align}
 \pdv{\remainingcost}{\demand} = 0
\end{align}
% 
Generally speaking, the cost term $\remainingcost$, not covered by the consumers, results from additional demands on the network design, such as capacity expansion limits in certain regions of the network, or minimum share of one technology in the power mix. However, in most cases, where $\remainingcost \ll  \totalcost$, these play a minor role. Feeding \cref{eq:total_cost_terms} into \cref{eq:total_cost_derivative} illustrates that the LMP splits into different cost terms assigned to OPEX and CAPEX for production and transmission (and other possible expenditures), schematically shown in \cref{fig:price_decomposition}. 
Extensive investigations of the LMP as done in \cite{schweppe_spot_1988} already showed this connection. However the question of how the LMP translates to the single cost terms $\cost_i$ associated to asset $i$ in the network remains unanswered. This work aims at presenting and illustrating a an intuitive scheme how to cost breakdown the cost decomposition to a full peer-to-peer (P2P) cost allocation including all network assets. 



\section{Nodal Cost Allocation}
\label{sec:theory}

In following we show how to disaggregate the cost terms $\opexgeneration$, $\emissioncost$, $\capexgeneration$ and $\capexflow$ for asset $i$ into contributions of demand $\demand$ at bus~$n$ and time~$t$. That is what amount consumers at $\left(n, t\right)$ pay for the operation, investment and other cost-intensive, but welfare contributing, actions at asset $i$.  
We follow the general procedure: For each cost term $\cost$ we break down the contributions of asset $i$, \ie define \cref{eq:asset_costs}. According to \cref{eq:total_cost_derivative,eq:total_cost_sum}, we define its derivative with respect to $\demand$ and multiply the result with $\demand$. This returns $\cost_{n\rightarrow i,t}$, the contribution of $\demand$ to the cost term $\cost_i$. Follows naturally that when summing up all contributions the cost term $\cost_i$ must be covered, \ie
\begin{align}
    \cost_i = \sum_{n,t} \cost_{n \rightarrow i, t}
\end{align}
and the total money spent at $\left(n,t\right)$ for the cost term $\cost$ amounts 
\begin{align}
    \cost_{n,t} = \sum_i \cost_{n \rightarrow i, t}
\end{align}
Note that for allocating CAPEX, we will make use of the shadowprices associated to $\cost_i$, given by the dual values at cost optimum $\bm{\mu}$. 


\subsection{Allocate Operation Expenditures}

The total OPEX is given by the operational price $\operationalpricegeneration$ times the generation $\generation$ of all generators $s$, 
\begin{align}
 \opexgeneration& = \sum_{s,t} \operationalpricegeneration\,  \generation 
\label{eq:opex_generation}
 \end{align}

Hence, the contribution of $\demand$ to the OPEX at generator $s$ is straightforward given by
\begin{align}
    \allocateopex[n \rightarrow s] &= 
   \operationalpricegeneration \,  \pdv{\generation}{\demand} \, \demand
\label{eq:allocate_opexGeneration_detailed}
\end{align}

We define the derivative on the right hand side times the nodal demand as 
\begin{align}
 \allocatepeer = \pdv{\generation}{\demand} \, \demand
 \label{eq:allocate_peer}
\end{align}
% 
The quantity $\allocatepeer$ can be considered as a power allocation, that is the power produced by generator $s$ and consumed at node $n$ and time $t$.
By feeding back \cref{eq:allocate_opexGeneration_detailed} into \cref{eq:opex_generation}, we see that the power allocation has to fulfill
\begin{align}
\generation = \sum_n \allocatepeer
\label{eq:allocate_peer_constraint}
\end{align}
% 
% 
The definition of $\allocateopex[n \rightarrow s]$ in \cref{eq:allocate_opexGeneration_detailed} can be interpreted as the following. Consumers at bus $n$ which retrieve power from a generators $s$, directly  compensate its operational costs. The cost allocation behaves like P2P tradings between producers and consumers for fixed production prices. In this way,  generator $s$ retrieves the exact amount of money from consumers that it spends on its operation at time $t$, thus 
\begin{align}
\sum_{n} \allocateopex[n \rightarrow s] = \operationalpricegeneration \, \generation
\label{eq:no_profit_opex}
\end{align}

\subsection{Allocate Emission Expenditures}

Given a fix price for emissions in CO$_2$ equivalents $\emissionprice$, the cost term for emission adds up to 
\begin{align}
 \emissioncost = \emissionprice \, \sum_s  \emission \, \generation
\end{align}
where $\emission$ denotes the emission factor in tonne-CO$_2$ per \megawatthour for generator $s$.
Similar to the allocated OPEX, the emission cost for consumers at bus $n$ at time $t$ for operation at generator $s$ is allocated through
\begin{align}
 \allocateemissioncost[n \rightarrow s] = \emissionprice \, \emission \, \allocatepeer
\end{align}


\subsection{Allocate Capital Expenditures for Generators}

The optimization of the generation capacity $\capacitygeneration$ adds the cost term 
\begin{align}
 \capexgeneration& = \sum_{s} \capitalpricegeneration \capacitygeneration 
\end{align}
to $\totalcost$. It combines all capacities times their capital cost $\capitalpricegeneration$. The nominal capacity constraints the generation $\generation$ in the form of 
\begin{align}
\generation - \generationpotential \capacitygeneration  &\le 0 \resultsin{\muuppergeneration} \Forall{n,s,t} 
\label[constraint]{eq:upper_generation_capacity_constraint}\\ 
- \generation &\le 0 \resultsin{\mulowergeneration} \Forall{n,s,t} 
\label[constraint]{eq:lower_generation_capacity_constraint}
\end{align}
where $\generationpotential \in \left[ 0,1\right]$ is the capacity factor for renewable generators. As shown in \cite{brown_decreasing_2020} and in detailed in \cref{sec:zero_profit_generation},  the capital cost  for generator $s$ is payed back by the generation times the shadow price for the upper capacity constraint, 
\begin{align}
 \capitalpricegeneration \capacitygeneration = \sum_t \muuppergeneration \,  \generation 
 \label{eq:no_profit_capex_generation}
\end{align}
With our definition in \cref{eq:allocate_peer} the allocated CAPEX can then be defined as 
\begin{align}
 \allocatecapexgeneration[n \rightarrow s] = \muuppergeneration \, \allocatepeer
 \label{eq:allocate_capexGeneration_detailed}
\end{align}

So, how does this allocation behave? According to the polluter pays principle, it differentiates between consumers who are `responsible` for investments and those who are not. If $\muuppergeneration$ (in literature often denoted as the Quality of Supply) is bigger than zero, the upper Capacity \cref{eq:upper_generation_capacity_constraint} is binding. Thus it is these times steps which push investments in $\capacitygeneration$. If $\muuppergeneration = 0$, the generation $\generation$ is not bound and investments are not necessary. 
When summing over all CAPEX payments to generator $s$ in \cref{eq:allocate_capexGeneration_detailed} and using \cref{eq:allocate_peer_constraint,eq:no_profit_capex_generation}, we see that each generator retrieves exactly the cost that were spent to build the capacity $\capacitygeneration$.
 

\subsection{Allocate Capital Expenditures for Transmission Lines}


The CAPEX for the transmission system is given by 
\begin{align}
 \capexflow & = \sum_\ell \capitalpriceflow \capacityflow
 \label{eq:capex_flow}
\end{align}
The transmission capacity variable $\capacityflow$ limits the flow $\flow$ in both directions, such that 
\begin{align}
\flow - \capacityflow &\le 0 \resultsin{\muupperflow} \Forall{\ell,t} 
\label[constraint]{eq:upper_flow_capacity_constraint} \\
- \flow - \capacityflow &\le 0 \resultsin{\mulowerflow} \Forall{\ell,t} 
\label[constraint]{eq:lower_flow_capacity_constraint}
\end{align}
We use the result of \cite{brown_decreasing_2020} (for details see \cref{sec:zero_profit_flow}) which derives that the investment in line $\ell$ is payed back by the shadowprices times the flow 
\begin{align}
\capitalpriceflow \capacityflow = \sum_{t} \left( \muupperflow - \mulowerflow \right)  \flow 
\label{eq:no_profit_capex_flow}
\end{align}
We insert this into \cref{eq:capex_flow} and define the individual cost terms  as 
\begin{align}
    \allocatecapexflow[n \rightarrow \ell] &=  
   \left( \muupperflow - \mulowerflow\right) \,  \pdv{\flow}{\demand}\, \demand
   \label{eq:allocate_capexFlow_detailed}
\end{align}
Again, we define the derivative on the right side times the demand as 
\begin{align}
 \allocateflow =  \pdv{\flow}{\demand}\, \demand
\end{align}
which from inserting into \cref{eq:no_profit_capex_flow} has to fulfill 
\begin{align}
 \flow = \sum_{n} \allocateflow
\label{eq:allocate_transaction_constraint}
\end{align}
$\allocateflow$ can be interpreted as the flow that the demand at $\left(n,t\right)$ induces on line $\ell$. The  shadowprices $\muupperflow$ and $\mulowerflow$ again can be seen as a measure for necessity of transmission investments at $\ell$ at time $t$. Hence, the definition of $\allocatecapexflow[n \rightarrow \ell]$ states that consumers, which retrieve power flowing on congested lines, yielding a bound \cref{eq:upper_flow_capacity_constraint} or \eqref{eq:lower_flow_capacity_constraint}, pay compensations for the resulting investments at $\ell$. Again the sum of all CAPEX payments to line $\ell$ equals the total CAPEX spent. This is seen when summing \cref{eq:allocate_capexFlow_detailed} over all buses and time steps and using \cref{eq:allocate_transaction_constraint,eq:no_profit_capex_flow}

\section{Assumptions on Power Allocations}
% \section{\texorpdfstring{Extended Solution Space of $\boldsymbol{\slackk[m]}$}{Solution Space of the Slack}}
\label{sec:localizing_allocations}

The presented cost allocation suits for any type of topology and network setup. But so far, the question of how $\allocatepeer$ and $\allocateflow$ are defined was left open. We recap that both rely on derivatives of $\generation$ and $\flow$  with respect to the nodal demand $\demand$ fulfilling \cref{eq:allocate_peer_constraint,eq:allocate_transaction_constraint}. \\
% 
In fact, it turns out that ones, the peer-to-peer allocations $\allocatepeer$ are defined, the power flow allocations $\allocateflow$ are determined by the linear power flow equations.
In order to show this, let $\ptdf$ denote Power Transfer Distribition Factors (PTDF) giving the changes in the flow on line $\ell$ for one unit (typically one MW) of net power production at bus $n$.  Further let $\incidencegenerator$ be 1 if generator $s$ is attached to bus $n$ and zero otherwise. The linear power flow equals the 
\begin{align}
 \flow  = \sum_m \ptdf[m] \left( \nodalgeneration[m] - \demand[m] \right)  
\end{align}
where the nodal generation $\nodalgeneration[m] = \sum_s \incidencegenerator[m] \, \generation $ combines the production of all generators attached to bus $m$. Taking the derivative with respect to the demand, leads to 
\begin{align}
 \allocateflow = \pdv{\flow}{\demand} \demand = \sum_m \ptdf[m] \left( \allocatepeer[m \rightarrow n]  - \delta_{n,m} \demand \right) 
 \label{eq:allocate_peer_to_allocate_flow}
\end{align}
where $\allocatepeer[m \rightarrow n] = \sum_s \incidencegenerator \, \allocatepeer$ combines all power allocations of the source $m$. \Cref{eq:allocate_peer_to_allocate_flow} shows that the peer-to-peer allocation $\allocatepeer$ directly determines the flow allocation on line $\ell$.

 
This leaves us with only solving \cref{eq:allocate_peer_constraint} for $\allocatepeer$. In fact, we have to answer the question of how much power supply does generator $s$ deliver to bus $n$. Unfortunately, the solution is non-unique and requires further assumptions. Established flow allocation schemes approach this problem from different directions. Principally two options exist \textit{what} is allocated 
% 
\begin{enumerate}
\item gross power injections \label{gross}
\item net power injections \label{net}
\end{enumerate}
% 
Further it is important \textit{what assumptions} define the allocation, \ie what method is used to define the pairs of sources and sinks. The three suitable approaches we present here are
% 
\begin{enumerate}[label=\alph*., ref=\alph*]
\item Equivalent Bilateral Exchanges (EBE) \cite{galiana_transmission_2003} which assumes
that every producer supplies every consumer proportional to its share in the total consumption. \label{ebe} 
\item Average Participation (AP) \cite{bialek_tracing_1996,achayuthakan_electricity_2010-1} which traces the flow from producer to consumer following the law of proportional sharing. \label{ap}
\item Flow Based Market Coupling (FBMC) which uses zonal PTDF for allocating power within predefined regions. The interregional exchange is only allocating net power decifit or excess of the regions. \label{fbmc}
\end{enumerate}
% 
We show the mathematical formulation for all combinations \ref{ebe}\ref{gross} - \ref{fbmc}\ref{net} in \cref{sec:gross_ebe,sec:net_ebe,sec:gross_ap,sec:net_ap}.
% 
Principally, type \ref{net} leads to less P2P trades then type \ref{gross} as power from a bus $n$ with $\nodalgeneration \le \nodaldemand$ is not assigned to other buses, only to $n$. 
Further, as literature has often pointed out, the EBE principal \ref{ebe} does not suit for large networks where remote buses would interconnect in the same way as buses in close vicinity \cite{gil_multiarea_2005}. The AP based type \ref{ap} tackles this problem by restricting P2P trades to those which are traceable when applying the proportional sharing principal. Therefore $\allocatepeer$ denotes that part of power produced by bus $m$ which, when only following in the direction of $\flow$, ends up at bus $n$. Type \ref{fbmc} further allows to control the regions or market zones which are netted out in a first calculation. If in a region $R$ the generation undercuts the demand, $\sum_{n \in R} \generation \le \sum_{n \in R} \demand$, none of the inner-regional generation is assigned to other regions. 



\subsection{Numerical Example}
\label{sec:numerical_example}


In the following we showcase a two bus system with one optimized time step and its resulting allocated payments, illustrated in \cref{fig:example_network}. \\
% 
% 
\begin{figure*}[t]
\centering
\includegraphics[width=\linewidth]{example_network.png}
\caption{Illustrative example of a 2-bus network with one optimized time step. Fixed prices and constraining values are given in the left box for each bus and the transmission line. Optimized values are given in the right boxes. Bus~1 has a cheaper operational price $o$, capital prices are the same for both. As both generator capacities are constraint to 100~MW, the optimization also deploys the generator at bus~2. The resulting electricity prices $\lambda$ are then a composition of all prices for operation and capital investments. \vspace{-10pt}}
\label{fig:example_network}
\end{figure*}
% 
% 
The two buses are connected via one transmission line, each has one generator. Whereas the generator at bus~1 has an operational price of 50 \euro/\megawatthour, the generator at bus~2 has a higher operational price of 200~\euro/\megawatthour. For both the CAPEX rate is set to 500~\euro/MW and the maxmimal capacity is limited to $\capacitygenerationupper$~=~100~MW. The transmission line has a CAPEX rate of 100~\euro/MW and no upper capacity limit. With a demand of 60~MW at bus~1 and 90~MW at bus~2, the optimization expands the cheaper generator at bus~1 to its full limit of 100~MW. The 40~MW excess power not consumed at bus~1, flow to bus~2 where the generator is built with only 50~MW.  
% 
\begin{figure}[h!]
    \begin{subfigure}[c]{\linewidth}
    \includegraphics[width=\linewidth]{example_allocation_bus1.png}
    \vspace{-40pt}
    \subcaption{}
    \label{fig:example_allocation_bus1}
    \end{subfigure}
    \begin{subfigure}[c]{\linewidth}
    \includegraphics[width=\linewidth]{example_allocation_bus2.png}
    \vspace{-40pt}
    \subcaption{}
    \label{fig:example_allocation_bus2}
    \end{subfigure}
    \caption{Power allocations for bus~1 (a) and bus~2 (b) of the example network in \cref{fig:example_network}. Bus~1 retrieves 40~MW from itself and 20~MW from bus~2. The latter in turn retrieves 60~MW from bus~1 and self supplies 30~MW.
    The sum of both net flows equals the resulting flow of $f=40$~MW. \vspace{-20pt}}
    \label{fig:example_allocation}
\end{figure}
% 
% 
% 
\begin{figure}[h]
\centering
\includegraphics[width=\linewidth]{example_payoff.png}
\caption{Full P2P cost allocation for the example setup shown in \cref{fig:example_network}. The payments are derived on the basis of \cref{eq:allocate_opexGeneration_detailed,eq:allocate_capexGeneration_detailed,eq:allocate_capexFlow_detailed}. Consumers at bus $n$ have to pay each generator proportional to their consumption. As we only consider one time step the proportionality applies for OPEX $\mathcal{O}_n$ and CAPEX $\mathcal{C}_n$. As bus~1 induces a relieving flow an line~1 and therefore ``prevents`` further tranmission expansion, it is rewarded proportional to the relief.}
\label{fig:example_payoff}
\end{figure}
% 
% 
\subsubsection*{Allocating Gross Power using EBE (\ref{gross}\ref{ebe}) }

\Cref{fig:example_allocation} shows the allocated transactions on basis of equvalently traded gross power (type \ref{gross}\ref{ebe}) for both buses 1~\&~2 separately. The resulting P2P payments are given in \cref{fig:example_payoff}.
The upper graph \cref{fig:example_allocation_bus1} shows that $A_{1,1}=40$~MW at bus~1 are self-sustained. Consumers at bus~1 consequently pay 2k~\euro~OPEX and 22k~\euro~CAPEX to the generator at bus~1. The remaining 20~MW come from bus~2 and induce a subflow on line 1 of $A_{2 \rightarrow 1, 1} = -20$. As this flow is in contrary direction to the total flow, it is relieving the transmission system. This translates to a congestion reward for consumers at bus~1 of $c_{\ell=1} A_{2 \rightarrow 1, 1}$~=~2k~\euro\, which is exactly the cost that had to be spent on the transmission system if bus~1 didn't induce a relieving flow, see again \cref{fig:example_payoff}. 



The lower graph \cref{fig:example_allocation_bus2} illustrates the impact of consumption at bus~2. As $d_2$ is higher than $d_1$ the retrievements from both generators are proportionately increased as well as the OPEX and CAPEX allocations to the generators. But instead of a relieving flow, consumers at bus~2 drive the burdening flow in direction of congestion. Hence the payoff to the transmission system is positive and much higher than for bus~1.

The sum of all rows in the payoff matrix in \cref{fig:example_payoff} yields the revenues of the assets $m, \ell$. These values match their overall spendings, \textit{e.g.} the total revenue of the transmission line is 4k~\euro\, which equals the cost for investments $c_{1}\,F_{1}$. The sum of all columns yields the total payment of consumers at bus $n$. For example the sum of payments of consumers at bus~1 is 36k~\euro. This is exactly the electricity price of 600~\euro/MW times the consumption of 60~MW, $\lambda_1 d_1$. \\

The fact that OPEX and CAPEX allocations are proportional to the total consumption at a bus results from optimizing one time step only. In larger optimization problems with multiple time steps the CAPEX allocation takes effect only for time steps in which one or more of the capacity contraints  \cref{eq:upper_generation_capacity_constraint,eq:lower_flow_capacity_constraint,eq:upper_flow_capacity_constraint} become binding.  


\subsubsection*{Allocating Net Power using EBE (\ref{net}\ref{ebe}) }

In contrast the to equivalent allocation of gross power production, netting out injections for each bus leads to less P2P payments. The resulting payment given in \cref{fig:example_payoff_net_ebe} builds on the allocated power flow shown \cref{fig:example_allocation_net_ebe} in \cref{sec:example_plots}. As bus~2 does not produce excess power, none of its power production is assigned to bus~1 and thus no payment of bus~1 to bus~2 allocated. Neither has bus~1 to pay fee to the transmission system as it only exports power. So all its consumers pay to its local generators. Bus~2 in constrast bear all CAPEX for the transmission system as well as CAPEX and OPEX for generators at bus~1.
\begin{figure}[h]
\centering
\includegraphics[width=\linewidth]{example_payoff_net_ebe.png}
\caption{Full P2P cost allocation for the example setup shown in \cref{fig:example_network} when equivalently allocation net power injection (scheme \ref{ebe}\ref{net}).}
\label{fig:example_payoff_net_ebe}
\end{figure}

Again the cummulative payments per bus meet the nodal spendings $\lmp \, \demand$. The cummulative revenues per generator and transmission line meet the all CAPEX and OPEX.  

As the other schemes \ref{ap}\ref{gross} - \ref{fbmc}\ref{net} apart from the EBE scheme don't distinguishably modify allocations within the presented example, we move on to a more realistic setup where systemic and local effects show up.


\section{Application Case}

\clearpage
\appendix

\section{Appendix}

\renewcommand\theequation{\thesection.\arabic{equation}}
\setcounter{equation}{0}

\renewcommand\thefigure{\thesection.\arabic{figure}}    
\setcounter{figure}{0}    

\subsection{LMP from Optimization}
The nodal balance constraint ensures that the amount of power that flows into a bus equals the power that flows out of a bus, thus reflects the Kirchhoff Current Law (KCL)
\begin{align}
    \sum_l \incidence \, \flow  - \nodalgeneration + \nodaldemand &= 0 \resultsin{\lmp} \Forall{n,t}
    \label[constraint]{eq:nodal_balance_lin}
\end{align}
Its shadow price mirrors the Locational Marginal Prizes (LMP) $\lmp$ per bus and time step. In a power market this is the \euro/\megawatthour-price which a consumer has to pay.\\

\subsection{Full Lagrangian}
\label{sec:full_lagrangian}
\begin{align}
\notag
& \lagrangian\left(\generation, \flow, \capacitygeneration, \capacityflow, \boldsymbol{\lambda}, \boldsymbol{\mu} \right)   =   \\  
\notag
& \;\;\; \sum_{n,s} \capitalpricegeneration \capacitygeneration + \sum_{n, s, t} \operationalpricegeneration \generation + \sum_{\ell} \capitalpriceflow \, \capacityflow  \\
\notag
&+ \sum_{n,t} \lmp \left(\sum_\ell \incidence \, \flow  - \sum_s \incidencegenerator\, \generation +  \demand  \right)  \\ 
\notag
&+ \sum_{\ell,c,t} \cycleprice \, \cycle \, \impedance \, \flow  \\
\notag
&+ \sum_{n,s,t} \muuppergeneration \left( \generation - \generationpotential \capacitygeneration \right)  - \sum_{n,s,t} \mulowergeneration \generation  \\
&+ \sum_{\ell,t} \muupperflow \left( \flow - \capacityflow \right) - \sum_{\ell,t} \mulowerflow \left( \flow + \capacityflow \right)     
\label{eq:full_lagrangian}
\end{align}
% 
where $\boldsymbol{\lambda} = \left\lbrace \lmp, \cycleprice \right\rbrace $ and $\boldsymbol{\mu} = \left\lbrace \muuppergeneration, \mulowergeneration, \muupperflow, \mulowerflow \right\rbrace $ denote the set of related KKT variables. The global maximum of the Lagrangian requires stationarity with respect to all variables. The stationarity of the generation capacity variable leads to 
\begin{align}
\pdv{\lagrangian}{\capacitygeneration}  = 0 \,\, \rightarrow \,\, 
\capitalpricegeneration =  \sum_t \muuppergeneration \, \generationpotential  \Forall{n,s}
\label{eq:capex_generation_duality}
\end{align}

the stationarity of the transmission capacity to
\begin{align}
\pdv{\lagrangian}{\capacityflow} = 0 \,\, \rightarrow \,\, 
\capitalpriceflow =  \sum_t \left( \muupperflow - \mulowerflow \right) \Forall{\ell}
\label{eq:capexFlow_duality}
\end{align}


and the stationarity of the generation to 
\begin{align}
\pdv{\lagrangian}{\generation} &= 0 \,\, \rightarrow \,\,  
\operationalpricegeneration =  \incidencegenerator \, \lmp - \muuppergeneration + \mulowergeneration \Forall{n,s} \label{eq:opex_duality}
\end{align}

\begin{align}
0 &= \pdv{\lagrangian}{\flow}  \\ 
0 &= \sum_n \incidence \lmp  + \cycleprice \cycle \impedance  - \muupperflow + \mulowerflow \Forall{n,s} \label{eq:opex_flow_duality}
\end{align}



\subsection{Proof Generation Capacity Payback}
\label{sec:zero_profit_generation}

As stated in \cite{brown_decreasing_2020} the capical cost of a generator $s$ is equal to the shadowprice $\muuppergeneration$ times the upper availibility $\generationpotential$ of all time steps 
\begin{align}
 \capitalpricegeneration  = \sum_t \muuppergeneration \, \generationpotential 
 \label{eq:capital_price_generation_sum}
\end{align}
\Cref{eq:upper_generation_capacity_constraint,eq:lower_generation_capacity_constraint} which yield the KKT variables $\muuppergeneration$ and $\mulowergeneration$ imply the complementary slackness,
\begin{align}
\muuppergeneration \left( \generation - \generationpotential \, \capacitygeneration \right)  &= 0  \Forall{n,s,t} 
\label{eq:complementary_slackness_upper_generation} \\
\mulowergeneration  \, \generation &= 0 \Forall{n,s,t}
\label{eq:complementary_slackness_lower_generation} 
\end{align}
Multiplying both sides of \cref{eq:capex_generation_duality} with $\capacitygeneration$ and using \cref{eq:complementary_slackness_upper_generation} leads to 
\begin{align}
 \capitalpricegeneration \, \capacitygeneration  = \sum_t \muuppergeneration \, \generation 
 \label{eq:capital_price_generation_sum}
\end{align}
The zero-profit rule for generators is obtained by multiplying \cref{eq:opex_duality} with $\generation$ and using \cref{eq:complementary_slackness_lower_generation,eq:capital_price_generation_sum} which results in 
\begin{align}
  \capitalpricegeneration \, \capacitygeneration + \sum_t \operationalpricegeneration \generation = \sum_t \lmp \incidencegenerator \generation
\end{align}
It states that over the whole time span, all OPEX and CAPEX for generator $s$ (left hand side) are payed back by its revenue (right hand side).

\subsection{Proof Flow Capacity Payback}
\label{sec:zero_profit_flow}

The yielding KKT variables $\muupperflow$ and $\mulowerflow$ are only non-zero if $\flow$ is limited by the trasmission capacity in positive or negative direction, i.e. \cref{eq:upper_flow_capacity_constraint} or \cref{eq:lower_flow_capacity_constraint} are binding. For flows below the thermal limit, the complementary slackness 
\begin{align}
\muupperflow \left( \flow - \capacityflow \right)  &= 0 \Forall{\ell,t}
\label{eq:complementary_slackness_upper_flow} \\
\mulowerflow \left( \flow - \capacityflow \right) &=  0 \Forall{\ell,t}
\label{eq:complementary_slackness_lower_flow} 
\end{align}
sets the respective KKT to zero. 
When multipying \cref{eq:capexFlow_duality} with $\capacityflow$ and using the complementary slackness \cref{eq:complementary_slackness_upper_flow,eq:complementary_slackness_lower_flow} we obtain 
\begin{align}
 \capitalpriceflow \, \capacityflow = \sum_t \left( \muupperflow - \mulowerflow \right)  \, \flow
 \label{eq:capital_price_flow_sum}
\end{align}
Againg we can use this to formulate the zero-profit rule for transmission lines. We mulitply \cref{eq:opex_flow_duality} with $\flow$, which finally leads us to 
\begin{align}
\capitalpriceflow \, \capacityflow = \sum_n \incidence\, \lmp\, \flow + \cycleprice\, \cycle\, \impedance\, \flow 
\end{align}
It states that the congestion revenue of a line (first term right hand side) substracted by the cost for cycle constraint exactly matches its CAPEX. 





\subsection{CO$_2$ constraint}

Imposing an additional CO$_2$ constraint limiting the total emission to K,  
\begin{align}
\sum_{n,s,t} \emission \, \generation \le \text{K} \resultsin{\emissionprice} 
\label[constraint]{eq:co2_constraint}
\end{align}
with $\emission$ being the emission factor in tonne-CO$_2$ per \megawatthour, returns an effective CO$_2$ price $\emissionprice$ in \euro/tonne-CO$_2$. 
% The CO$_2$ price shifts the right hand side of the non-profit relation for generators \cref{eq:non_profit_generator} to
% 
% \begin{align}
% \capitalpricegeneration \, \capacitygeneration + \sum_{t} \operationalpricegeneration \, \generation &= \sum_{t} \left( \lmp - \emission \, \emissionprice \right)  \, \generation \Forall{n,s} 
% \label{eq:non_profit_generator_emission}
% \end{align}
% This shows nicely the duality for exchanging the CO$_2$ \cref{eq:co2_constraint} for a shifted OPEX which includes the CO$_2$ costs
As shown in ... the constraint can be translated in a dual price which shift the operational price per generator
\begin{align}
\operationalpricegeneration \rightarrow \operationalpricegeneration + \emission \, \emissionprice \label[relation]{eq:shift_opex_by_emission_cost}
\end{align}

\subsection{Storage Units}



\subsection{Brownfield Optimization and Capacity Restricitions}

Constraining the capacities $\capacitygeneration$  for a subset $S$ of generators to lower or upper limits in the form of
\begin{align}
\capacitygeneration \ge \capacitygenerationlower \resultsin{\mulowergenerationnom} \hpad \forall{n,s \in S} \label{eq:capacityexpansionminimum}\\
\capacitygeneration \le \capacitygenerationupper \resultsin{\muuppergenerationnom} \hpad \forall{n,s \in S}
\label{eq:capacityexpansionmaximum}
\end{align}
alters the objective value as soon as one of those become bounding. 
The complementary slackness for both are 
\begin{align}
    \muuppergenerationnom \left( \capacitygeneration - \capacitygenerationupper \right) =0 \Forall{n,s \in S} \\
    \mulowergenerationnom \left( \capacitygeneration - \capacitygenerationlower \right) =0 \Forall{n,s \in S}
\end{align}


The CAPEX paybacks for generators and transmission lines
\cref{eq:capital_price_generation_sum,eq:capital_price_flow_sum} change to 
\begin{align}
\pdv{\lagrangian}{\capacitygeneration}  = 0 \,\, \rightarrow \,\, \\
\capitalpricegeneration =  \sum_t \muuppergeneration \, \generationpotential + \mulowergenerationnom - \muuppergenerationnom \Forall{n,s \in S}
\label{eq:capex_generation_duality_bf}
\end{align}
for generators. Multiplying \cref{eq:capex_generation_duality_bf} by $ \capacitygeneration$ leads us to 
\begin{align}
 \capitalpricegeneration \, \capacitygeneration - \remainingcost_s=  \sum_t \muuppergeneration \, \generation \Forall{n,s \in S}
 \label{eq:capex_generation_duality_bf2}
\end{align}
where we define the cost resulting from the capacity expansion limits as
\begin{align}
    \remainingcost_s = \left(\mulowergenerationnom - \muuppergenerationnom \right) \capacitygeneration
    \Forall{s}
\end{align}
As $\mulowergenerationnom \ge 0$ and $\muuppergenerationnom \le 0$ the latter represents a net positive cost term.

Multiplying \cref{eq:opex_duality}, which is still valid, with $\generation$ and inserting \cref{eq:capex_generation_duality_bf2} will bring us to the zero profit rule for generators with capacity expansion limits, 
\begin{align}
    \capitalpricegeneration \capacitygeneration + \sum_t \operationalpricegeneration \generation  - \remainingcost_s = \sum_t \lmp \incidencegenerator \generation \Forall{s \in S}
\end{align}
We see that the revenue of generator $s$ will not fully pay back its full CAPEX and OPEX. The exogenous constraint shifts the zero-profit equations such that some of the expenditures for $s \in S$ cannot directly be allocated to consumers. The collected payments of all consumers will not cover the costs induced by \cref{eq:capacityexpansionminimum,eq:capacityexpansionmaximum},

\begin{align}
    \capitalpricegeneration \capacitygeneration - \remainingcost_s = \sum_{n,t} \allocatecapexgeneration[n \rightarrow s] \Forall{s \in S}
\end{align}



Doing likewise with a subset $L$ of transmission lines,
\begin{align}
\capacityflow \ge \capacityflowLower \resultsin{\mulowerflownom} \hpad \forall{\ell \in \textit{L}} \\
\capacityflow \le \capacityflowUpper \resultsin{\muupperflownom} \hpad \forall{\ell \in \textit{L}}
\end{align}
results in a shifted zero-profit rule for transmission line in the form 
\begin{align}
    \capitalpriceflow \, \capacityflow - \remainingcost_\ell = \sum_{n,t}\allocatecapexflow[n \rightarrow \ell] \Forall{\ell \in L}
\end{align}
where $\remainingcost_\ell$ is given by 
\begin{align}
    \remainingcost_\ell = \left(\mulowerflownom - \muupperflownom \right) \, \capacityflow
    \Forall{\ell \in L}
\end{align}

% \begin{align}
%     \muupperflownom \left( \capacityflow - \capacityflowUpper \right) =0 \\
%     \mulowerflownom \left( \capacityflow - \capacityflowLower \right) =0 \\
% \end{align}

% \begin{align}
%     \pdv{\lagrangian}{\capacityflow}  = 0 \,\, \rightarrow \,\, \\
%     \capitalpriceflow =  \sum_t \left( \muupperflow - \mulowerflow \right)  + \mulowerflownom - \muupperflownom \Forall{n,s}
%     \label{eq:capex_flow_duality_bf}
%     \end{align}
%     for transmission lines.
% Multiplying \cref{eq:capex_generation_duality_bf} by the amount of expansion $\left( \capacitygeneration - \capacitygenerationlower \right) $
% \begin{align}
%   \capitalpricegeneration \, \capacitygeneration^{exp} =  \sum_t \muuppergeneration \, \left( \generation - \generationpotential \, \capacitygenerationlower \right) + \muuppergenerationnom \left( \capacitygenerationupper - \capacitygenerationlower \right) \Forall{n,s}
% \end{align}


\subsection{Allocating Gross Injections with EBE}
\label{sec:gross_ebe}

The allocation of gross generation to demands $\demand$ is straigthforwardly obained by a proportional distribution of the generation, \ie

\begin{align}
    \allocatepeer[s \rightarrow n] = \dfrac{\generation}{\sum_s \generation} \demand 
\end{align}


\subsection{Allocating Net Injections with EBE}
\label{sec:net_ebe}

Allocating net power injections using the EBE methods leads to the same result as the Marginal Participation (MP) \cite{rudnick_marginal_1995}  algorithm when allocating to consumers only, see \cite{hofmann_flow_2020-1} for further insight. We calculate it by setting 
\begin{align}
\allocatepeer[m \rightarrow n] &=  \delta_{m,n}\,\selfconsumption[m] + \gamma_t \, \netconsumption  \, \netproduction[m]
\label{eq:mp_slack}
\end{align}
where 
\begin{itemize}
%  \item $\injection = \left( \nodalgeneration - \nodaldemand \right) $ denotes the nodal injection
\item $\netproduction = \text{min}\left( \nodalgeneration - \nodaldemand , 0 \right) $ denotes the nodal net production 
\item $\netconsumption = \text{min}\left( \nodaldemand  - \nodalgeneration, 0 \right)$ denotes the nodal net consumption
\item $\selfconsumption = \text{min}\left( \netproduction, \netconsumption \right)$ the denotes  nodal self-consumption. That is the power generated and at the same time consumed at node $n$ and 
\item $\gamma_t = \left( \sum_n \netproduction\right) ^{-1} = \left( \sum_n \netconsumption\right) ^{-1}$ is the inverse of the total injected/extracted power at time $t$.
\end{itemize}

The allocation $\allocatepeer$ from generator $s$ to $n$, is given by multiplying $\allocatepeer[m \rightarrow n]$ with the nodal share $\generation / \nodalgeneration$.


\subsection{Allocating Net Power using AP}
\label{sec:net_ap}

\newcommand{\incidenceM}{K}
\newcommand{\flowM}{f}
\newcommand{\injectionM}{p}
\newcommand{\slackM}{k}
\newcommand{\DirectedIncidence}{\mathcal{K}}
\newcommand{\InverseAPInjection}{\mathcal{J}}
\newcommand\diag[1]{\operatorname{diag}\left(#1\right)}


Allocating net injections using the AP method is derived from \cite{achayuthakan_electricity_2010-1}. In a lossless network the downstream and upstream formulations result in the same P2P allocation which is why we restrict ourselves to the downstream formulation only. In a first step we define a time-dependent auxiliary matrix $\InverseAPInjection_t$ which is the inverse of the $N\times N$ with directed power flow $m \rightarrow n$ at entry $(m, n)$ for $m \ne n$ and the total flow passing node $m$ at entry $\left( m, m\right)$ at time step $t$. Mathematically this translates to


\begin{align}
\InverseAPInjection_t = \left( \diag{\injectionM^+} + \DirectedIncidence^- \diag{\flowM} \, \incidenceM \right)_t^{-1} 
\end{align}
where $\DirectedIncidence^-$ is the negative part of the directed Incidence matrix $\DirectedIncidence_{n,\ell} = \text{sign}\left( \flow \right)  \incidence$. Then the distributed slack for time step $t$ is given by
\begin{align}
\allocatepeer[m \rightarrow n] = \InverseAPInjection_{m,n,t} \, \netproduction[m] \, \netconsumption
\end{align}

\subsection{Allocating Gross Power using AP}
\label{sec:gross_ap}

We use the same allocation as in \cref{sec:net_ap} but replace the net nodal production $\netproduction$ by the gross nodal production $\nodalgeneration$ which leads to  
\begin{align}
\InverseAPInjection_t = \left( \diag{g} + \DirectedIncidence^- \diag{\flowM} \, \incidenceM \right)_t^{-1} 
\end{align}
The distributed slack is for time step $t$ is then given by
\begin{align}
\allocatepeer[s \rightarrow m] = \InverseAPInjection_{m,n} \, \generation \, \demand
\end{align}


\subsection{Example: Power Flow Allocations of different Types}
\label{sec:example_plots}
\begin{figure}[h!]
    \begin{subfigure}[c]{\linewidth}
    \includegraphics[width=\linewidth]{example_allocation_bus1_net_ebe.png}
    \vspace{-40pt}
    \subcaption{}
    \label{fig:example_allocation_bus1_net_ebe}
    \end{subfigure}
    \begin{subfigure}[c]{\linewidth}
    \includegraphics[width=\linewidth]{example_allocation_bus2_net_ebe.png}
    \vspace{-40pt}
    \subcaption{}
    \label{fig:example_allocation_bus2_net_ebe}
    \end{subfigure}
    \caption{Power allocations for bus~1 (a) and bus~2 (b) of the example network in \cref{fig:example_network} using equivalently allocated net power injections (scheme \ref{net}\ref{ebe}). Bus~1 retrieves 60~MW from itself and nothing from bus~2. The latter in turn retrieves 40~MW from bus~1 and self supplies 50~MW. The P2P trades are less in number and more intuitive then with allocating gross flow.}
    \label{fig:example_allocation_net_ebe}
\end{figure}



\clearpage
\printbibliography




\end{document}
