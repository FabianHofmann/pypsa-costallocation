\documentclass[a4paper,10pt]{article}
\usepackage{graphicx}
\usepackage[left=1.80cm, right=1.80cm, top=2.00cm, bottom=2.00cm]{geometry}
\usepackage{amsmath}
\usepackage[colorlinks]{hyperref}
\usepackage[backend=biber,style=draft]{biblatex}
\usepackage{eurosym}
\usepackage[dvipsnames]{xcolor}
\usepackage{subcaption}
\usepackage{enumitem} % for alphabetical enumeration 
\usepackage{accents}
\usepackage[capitalise]{cleveref}


%opening
\title{}
\author{Fabian Hofmann}

\begin{document}

\maketitle

\begin{abstract}

\end{abstract}

% style operators
\newcommand{\ie}{\textit{i.e.} }
\newcommand{\eg}{\textit{e.g.} }
\newcommand{\ubar}[1]{\underaccent{\bar}{#1}}
\newcommand{\note}[1]{\textcolor{Orange}{#1}}
\newcommand{\vpad}{\vspace{1mm}}
\newcommand{\hpad}{\hspace{15pt}}

% general symbols
\newcommand{\generation}{g_{s,t}}
\newcommand{\generationpotential}{\bar{g}_{s,t}}
\newcommand{\generationshare}{\omega_{s,t}}
\newcommand{\incidenceGenerator}{K_{n,s}}
\newcommand{\nodalgeneration}{\incidenceGenerator \, \generation}
\newcommand{\capacityGeneration}{G_{s}}
\newcommand{\capacityGenerationUpper}{\bar{G}_{s}}
\newcommand{\capacityGenerationLower}{\ubar{G}_{s}}
\newcommand{\capacityFlow}{F_{\ell}}
\newcommand{\capacityFlowUpper}{\bar{F}_{\ell}}
\newcommand{\capacityFlowLower}{\ubar{F}_{\ell}}
\newcommand{\capexGeneration}{c_{s}}
\newcommand{\capexFlow}{c_{\ell}}
\newcommand{\opexGeneration}{o_{s}}
\newcommand{\demand}{d_{n,t}}
\newcommand{\demandshare}{\omega_{a,t}}
\newcommand{\utility}{U_{n,a,t}}
\newcommand{\incidence}[1][n]{K_{#1,\ell}}
\newcommand{\ptdf}[1][n]{H_{\ell,#1}}
\newcommand{\ptdfEqual}[1][n]{\ptdf[#1]^\circ}
\newcommand{\slackflow}{k_{\ell}}
\newcommand{\slack}[1][n]{k_{#1}}
\newcommand{\slackk}[1][n]{k^*_{#1}}
\newcommand{\Slack}{k_{m,n}}
\newcommand{\Slackk}{k^*_{m,n}}
\newcommand{\mulowergeneration}{\ubar{\mu}_{s,t}}
\newcommand{\muuppergeneration}{\bar{\mu}_{s,t}}
\newcommand{\mulowerflow}{\ubar{\mu}_{\ell,t}}
\newcommand{\muupperflow}{\bar{\mu}_{\ell,t}}
\newcommand{\lmp}[1][n]{\lambda_{#1,t}}
\newcommand{\flow}{f_{\ell,t}}
\newcommand{\cycle}{C_{\ell,c}}
\newcommand{\impedance}{x_\ell}
\newcommand{\cycleprice}{\lambda_{c,t}}
\newcommand{\injection}{p_{n,t}}
\newcommand{\netconsumption}[1][n]{p^{-}_{#1,t}}
\newcommand{\netproduction}[1][n]{p^{+}_{#1,t}}
\newcommand{\selfconsumption}[1][n]{p^{\circ}_{#1,t}}

% matrix notation
\newcommand{\incidenceM}{K}
\newcommand{\flowM}{f}
\newcommand{\injectionM}{p}
\newcommand{\slackM}{k}
\newcommand{\DirectedIncidence}{\mathcal{K}}
\newcommand{\InverseAPInjection}{\mathcal{J}}
\newcommand\diag[1]{\operatorname{diag}\left(#1\right)}

% totals
\newcommand{\totalnetconsumption}{p^{-}_{t}}
\newcommand{\totalnetproduction}{p^{+}_{t}}
\newcommand{\totalselfconsumption}{p^{\circ}_{t}}

\newcommand{\lagrangian}{\mathcal{L}}

% allocation quantities
\newcommand{\allocatePeer}[1][s \rightarrow n]{A_{#1,t}}
\newcommand{\allocateFlow}[1][n]{F_{#1,\ell,t}}
\newcommand{\allocateTransaction}[1][s \rightarrow n]{A_{#1,\ell,t}}
\newcommand{\allocateCapexGeneration}[1][n]{\mathcal{C}^{G}_{#1,t}}
\newcommand{\allocateCapexFlow}[1][n]{\mathcal{C}^{F}_{#1,t}}
\newcommand{\allocateOpex}[1][n]{\mathcal{O}_{#1,t}}
\newcommand{\allocateEmissionCost}[1][n]{\mathcal{E}_{#1,t}}


\newcommand{\emission}[1][n]{e_{#1,s}}
\newcommand{\emissionPrice}{\mu_{\text{CO2}}}
\newcommand{\megawatthour}{MWh$_\text{el}$}
\newcommand{\totalcost}{\mathcal{TC}}
\newcommand{\totalOpexGeneration}{\mathcal{O}}
\newcommand{\totalOpexFlow}{\mathcal{O}^F}
\newcommand{\totalCapexGeneration}{\mathcal{C}^G}
\newcommand{\totalCapexFlow}{\mathcal{C}^F}
\newcommand{\totalEmissionCost}{\mathcal{E}}
\newcommand{\totalRest}{\mathcal{R}}
\newcommand{\totalDemand}{\mathcal{D}}

%math 
\newcommand{\resultsin}[1]{\hspace{6pt} \bot  \hspace{6pt} #1}
\newcommand{\Forall}[1]{\hspace{10pt} \forall \,\, #1 }
\newcommand{\pdv}[2]{\dfrac{\partial #1}{\partial #2}}


\section{Optimization}


\def\l{\lambda}
\def\K{\kappa}
\def\m{\mu}
\def\G{\Gamma}
\def\d{\partial}
\def\cL{\mathcal{L}}


Start with a generic linear objective function over $N$ variables $x_i$
\begin{equation}
 \min_{x_i} f(x) =  \min_{x_i}  \sum_{i=1}^N c_i x_i
\end{equation}
such that they respect linear inequality constraints
\begin{equation}
  \sum_i A_{ji} x_i \leq d_j \hspace{1cm} \m_j \hspace{1cm} j=1,\dots M
\end{equation}
    [Linear equality constraints $\sum_i b_i x_i = c$ with $\lambda$ can be replaced by two inequalities $\leq c$, $\geq c$ with $\lambda = \bar{\m} - \ubar{\m}$.]

A subset $B$ of the inequality constraints will be binding at the
optimum point $x^*$, i.e. for $j\in B$, $\sum_i A_{ji} x^*_i = d_j$. We
write $A'_{ji}$ for the matrix that only runs over $j\in B$. For
non-degenerate solutions these will be enough binding constraints to
solve for a unique optimum $x^*$, i.e. $|B| = N \leq M$, so that $A'$ is an $N\times N$
square matrix.  Therefore we can invert the saturation equation to get
\begin{equation}
  x^*_i = \sum_{k\in B} A^{\prime -1}_{ik} d_k
\end{equation}
It might seem that we've now lost all information about the objective function in this expression for $x^*$, but remember that it is the objective function which determines which constraints are binding, i.e. which vertex of the feasible simplex is the optimum.

\section{Power system}


We consider a power system with minimized total cost $\totalcost$ being subject to supplying demand $\demand$ at node $n$ at time $t$. 
The change in total cost against variation of the nodal demand
\begin{align}
\lmp  =  \pdv{\totalcost}{\demand}
\end{align}
defines the Locational Marginal Price $\lmp$ that consumers at node $n$ and time $t$. 
The total cost $\totalcost$ splits into two cost terms
\begin{align}
\totalcost &= \totalDemand +  \totalRest
\label{eq:total_cost_terms}
\end{align}
where $\totalDemand = \sum_i \totalDemand_i $ combines cost all terms which are sensitive to changes in the nodal demand 
\begin{align}
 \pdv{\totalDemand_i}{\demand} \ne 0
\end{align}
and thus determine the LMP,
\begin{align}
\lmp = \pdv{\totalDemand}{\demand} 
\end{align}
and $\totalRest = \sum_i \totalRest_i$ combines cost terms which is not, thus 
\begin{align}
 0 = \pdv{\totalRest_i}{\demand} \Forall{i}
\end{align}
Due to strong duality the total cost $\totalcost$ minus the exogeous cost term $\totalRest$ is totally payed back by the consumers
\begin{align}
 \totalcost - \totalRest = \sum_{n,t} \lmp \demand
\end{align}


\subsubsection*{Allocate OPEX}

The total OPEX is given by 
\begin{align}
 \totalOpexGeneration& = \sum_{s,t} \opexGeneration\,  \generation 
\label{eq:opex_generation}
 \end{align}
 We want to disaggregate the totel OPEX into contribution of bus n at time 
 \begin{align}
  \totalOpexGeneration = \sum_{n,t} \allocateOpex
 \end{align}
which are defined as 
\begin{align}
\allocateOpex  = \pdv{\totalOpexGeneration}{\demand} \, \demand & =  \sum_{s,t}\opexGeneration \, \pdv{\generation}{\demand} \, \demand
\end{align}
% 
From feeding this back into \cref{eq:opex_generation} we see that the derivative $\partial \generation / \partial \demand$ has to fulfill
\begin{align}
\generation = \sum_n \pdv{\generation}{\demand} \, \demand
\end{align}
Where the term at the right hand side denote the power produced by generator $s$ and consumed at node $n$, which we define as
\begin{align}
 \allocatePeer = \pdv{\generation}{\demand} \, \demand
 \label{eq:allocate_peer}
\end{align}
% 
The direct payment of node $n$ to generator $s$ is then  
% 
\begin{align}
 \allocateOpex[n \rightarrow s] &= 
\opexGeneration \,  \allocatePeer
\label{eq:allocate_opexGeneration_detailed}\\
\end{align}


\subsubsection*{Allocate Generation CAPEX}
\begin{align}
 \totalCapexGeneration& = \sum_{s} \capexGeneration \capacityGeneration 
\end{align}
with 
\begin{align}
 \capexGeneration = \muuppergeneration \, \generation
\end{align}
and \cref{eq:allocate_peer}, we define 
\begin{align}
\allocateCapexGeneration = \pdv{\totalCapexGeneration}{\demand} \, \demand & = \sum_{s} \muuppergeneration \, \allocatePeer
\end{align}




\subsubsection*{Allocate Transmission CAPEX}


The CAPEX for the transmission system is given by 

\begin{align}
 \totalCapexFlow & = \sum_\ell \capexFlow \capacityFlow = \sum_{\ell,t} \left( \muupperflow - \mulowerflow \right)  \flow 
\end{align}
% 
The individual cost terms are 
\begin{align}
\allocateCapexFlow = \pdv{\totalCapexFlow}{\demand}\demand & =  \sum_{\ell,t} \left( \muupperflow - \mulowerflow \right) \pdv{\flow}{\demand} \, \demand
\end{align}
where we assume $\partial \muupperflow/\partial \demand \cong 0 $, as dual variables have a second order dependeny on variables. The marginal change of flow with repect to the nodal demand is again non-unique and requires assumption on where the power is resulting from. 

\begin{align}
 \pdv{\flow}{\demand}\, \demand = \pdv{\flow}{\generation} \pdv{\generation}{\demand} \, \demand  = \allocateTransaction
\end{align}
fulfilling 
\begin{align}
\sum_{s,n} \allocateTransaction = \flow 
\end{align}




 



\end{document}
