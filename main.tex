\documentclass[11pt]{article}
\usepackage{graphicx}
\usepackage[left=2.00cm, right=2.00cm, top=2.00cm, bottom=2.00cm]{geometry}
\usepackage{amsmath}
\usepackage[colorlinks]{hyperref}
\usepackage[backend=biber]{biblatex}
\usepackage{eurosym}
\usepackage[dvipsnames]{xcolor} 
\usepackage{subcaption}
\usepackage{accents}
\usepackage[capitalise]{cleveref}

\addbibresource{main.bib}
\graphicspath{{figures/}}

\crefname{relation}{Rel.}{Rels.}
\creflabelformat{relation}{(#2#1#3)}
\crefname{constraint}{Constr.}{Constrs.}
\creflabelformat{constraint}{(#2#1#3)}

\setlength\parindent{8pt}

\newcommand{\ie}{\textit{i.e.} }
\newcommand{\eg}{\textit{e.g.} }

\newcommand{\ubar}[1]{\underaccent{\bar}{#1}}
\newcommand{\note}[1]{\textcolor{Orange}{#1}}
\newcommand{\vpad}{\vspace{1mm}}

\newcommand{\generation}[1][n]{g_{#1,s,t}}
\newcommand{\generationpotential}{\bar{g}_{n,s,t}}
\newcommand{\generationshare}[1][n]{\omega_{#1,s,t}}
\newcommand{\nodalgeneration}[1][n]{g_{#1,t}}
\newcommand{\capacityGeneration}{G_{n,s}}
\newcommand{\capacityFlow}{F_{\ell}}
\newcommand{\capexGeneration}{c_{n,s}}
\newcommand{\capexFlow}{c_{\ell}}
\newcommand{\opexGeneration}[1][n]{o_{#1,s}}
\newcommand{\demand}[1][n]{d_{#1,a,t}}
\newcommand{\nodaldemand}[1][n]{d_{#1,t}}
\newcommand{\demandshare}[1][n]{\omega_{#1,a,t}}
\newcommand{\utility}{U_{n,a,t}}
\newcommand{\incidence}[1][n]{K_{#1,\ell}}
\newcommand{\ptdf}[1][n]{H_{\ell,#1}}
\newcommand{\ptdfEqual}[1][n]{\ptdf[#1]^\circ}
\newcommand{\slackflow}{k_{\ell}}
\newcommand{\slack}[1][n]{k_{#1}}
\newcommand{\slackk}[1][n]{k^*_{#1}}
\newcommand{\Slack}{k_{m,n}}
\newcommand{\Slackk}{k^*_{m,n}}
\newcommand{\mulowergeneration}[1][n]{\ubar{\mu}_{#1,s,t}}
\newcommand{\muuppergeneration}[1][n]{\bar{\mu}_{#1,s,t}}
\newcommand{\mulowerflow}{\ubar{\mu}_{\ell,t}}
\newcommand{\muupperflow}{\bar{\mu}_{\ell,t}}
\newcommand{\lmp}[1][n]{\lambda_{#1,t}}
\newcommand{\flow}{f_{\ell,t}}
\newcommand{\cycle}{C_{\ell,c}}
\newcommand{\impedance}{x_\ell}
\newcommand{\cycleprice}{\lambda_{c,t}}
\newcommand{\injection}{p_{n,t}}
\newcommand{\netconsumption}[1][n]{p^{-}_{#1,t}}
\newcommand{\netproduction}[1][n]{p^{+}_{#1,t}}
\newcommand{\selfconsumption}[1][n]{p^{\circ}_{#1,t}}

\newcommand{\totalnetconsumption}{p^{-}_{t}}
\newcommand{\totalnetproduction}{p^{+}_{t}}
\newcommand{\totalselfconsumption}{p^{\circ}_{t}}

\newcommand{\lagrangian}{\mathcal{L}}

\newcommand{\allocatePeer}[1][m \rightarrow n]{A_{#1,t}}
\newcommand{\allocateFlow}[1][n]{F_{#1,\ell,t}}
\newcommand{\allocateTransaction}[1][m \rightarrow n]{A_{#1,\ell,t}}
\newcommand{\allocateCapexGeneration}[1][n]{\mathcal{C}^{G}_{#1,t}}
\newcommand{\allocateCapexFlow}[1][n]{\mathcal{C}^{F}_{#1,t}}
\newcommand{\allocateOpex}[1][n]{\mathcal{O}_{#1,t}}
\newcommand{\allocateEmissionCost}[1][n]{\mathcal{E}_{#1,t}}

\newcommand{\emission}[1][n]{e_{#1,s}}
\newcommand{\emissionPrice}{\mu_{\text{CO2}}}
\newcommand{\megawatthour}{MWh$_\text{el}$}
\newcommand{\totalcost}{\mathcal{TC}}
\newcommand{\impactcapexgeneration}{\Phi_{n,s,t}}
\newcommand{\impactcapexflow}{\Phi_{\ell,t}}

%math 
\newcommand{\resultsin}[1]{\hspace{12pt} \bot  \hspace{12pt} #1}
\newcommand{\Forall}[1]{\hspace{20pt} \forall \,\, #1 }
\newcommand{\pdv}[2]{\frac{\partial #1}{\partial #2}}

\begin{document}


\title{From Linear Optimization to Transmission Cost Allocation}
\author{Fabian Hofmann}

\maketitle

% \begin{abstract}
% The abstract text goes here.
% \end{abstract}

In the following, we show how Flow Allocation (FA) can be used for decomposing Locational Marginal Prices (LMP) in a linear optimized power system. We begin by revising the fundamentals of the Power Transfer Distribition Factors (PTDF) and the role of the slack, which plays a key role for FA. This will lead us to a generalized form of peer-to-peer allocations and transmission usage allocations based on the choice of slack. In a second step, we breakdown the full Lagrangian of a cost-optimization for a network with transmission and generation capacity expansion. By taking into account sensitivities of the flow with respect to changes in the nodal generation, we are able to derive a cost allocation which directly links to FA.  

\subsubsection*{Nomenclature}

\begin{table}[h]
	\centering
	\begin{tabular}{ll}
        $\demand$ & Electric demand per bus $n$, demand type $a$, time step $t$ in MW  \vpad \\
        $\generation$ & Electric generation per bus $n$, carrier $s$, time step $t$  in MW \vpad \\
        $\flow$ & Active power flow on line $\ell$, time step $t$ in MW   \vpad \\
        $\opexGeneration$ & Operational cost (OPEX) in \euro/MW \vpad \\
        $\capexGeneration$ & Capital Expenditure (CAPEX) in \euro/MW \vpad \\
        $\capexFlow$  & CAPEX per transmission line $\ell$ in \euro/MW  \vpad \\
        $\capacityGeneration$ & Generation capacity in MW \vpad \\
        $\capacityFlow$ & Transmission capacity in MW \vpad \\
        $\incidence$ & Incidence matrix \vpad 
%         $\injection$ & Net nodal injection in MW, \ie $\left( \nodalgeneration - \nodaldemand \right)$  \vpad \\
%         $\netproduction$ & Net nodal production in MW, \ie $\text{min}\left( \injection, 0 \right)$ \vpad \\
%         $\netconsumption$ & Net nodal consumption in MW, \ie $\text{min}\left( - \injection, 0 \right)$ \vpad \\ 
%         $\selfconsumption$ & Nodal self-consumption in MW, \ie $\text{min}\left( \netproduction, \netconsumption \right)$
	\end{tabular}
\end{table}




\subsubsection*{Power Transfer Distribition Factors and Flow Allocation}


In linear power flow models, the Power Transfer Distribition Factors (PTDF) $\ptdf$ determine the changes in the flow on line $\ell$ for one unit (typically one MW) of net power production at bus $n$. Thus for a given gross production $\generation$ and gross demand $\demand$, they directly link to the the resulting flow on each line, 
\begin{align}
 \flow  = \sum_n \ptdf \left( \nodalgeneration- \nodaldemand \right)  
 \label{eq:flow_from_ptdf}
\end{align}
where $\nodalgeneration = \sum_s \generation$ is the production of all generators $s$ attached to $n$ and $\nodaldemand = \sum_a \demand$ the demand of all consumers $a$ attached to $n$.
The PTDF have a degree of freedom: The slack $\slack$ denotes the contribution of bus $n$ to balancing out total power excess or deficit in the system. It can be dedicated to one bus, a sinlge ``slackbus``, or to several or all buses. The choice of slack modifies the PTDF according to 
\begin{align}
 \ptdf\left( \slack[m]\right)  = \ptdfEqual - \sum_m \ptdfEqual[m]  \, \slack[m]
 \label{eq:ptdf_slacked}
\end{align}
where $\ptdfEqual$ denote the PTDF with equally distributed slack.
When bus $n$ injects excess power, it has to flow to the slack; when bus $n$ extract deficit power, it has to come from the slack. Summing over all ingoing and outgoing flow changes resulting from an positive injection at $n$ yields again the slack 
\begin{align}
\sum_\ell \incidence[m] \, \ptdf =  \delta_{m,n} - \slack[m] 
\label{eq:slack}
\end{align}
where $\delta_{m,n}$ on the right hand side represents the positive injection at $n$.
% As shown above, the choice of slack $\slack[m]$ decides on the generators and lines to which power imbalances are distributed to. 
Established flow allocation schemes [cite] haved used this degree of freedom in order to allocate power flows and exchanges to market participants. Under the assumption that consumers account for all power flows in the grid, the slack is set to $\slack^*$ such that 
\begin{align}
 \flow  = - \sum_n \ptdf\left( \slackk[m]\right) \, \nodaldemand  
 \label{eq:flow_from_demand}
\end{align}
Therefore, the flow can be reproduced from the demand-side of the system only. Each term in the sum on the right hand side stands for the individual contribution of consumers at node $n$ to the network flow $\flow$. In other words, each nodal demand $\nodaldemand$ induces a subflow originating from the slack $\slackk$ which all together add to $\flow$. The subflows, in turn, can be further broken down to contributions of each individual production-to-demand relation, $m \rightarrow n$, that is 
\begin{align}
 \allocateTransaction = \left(  \ptdfEqual[m] - \ptdfEqual \right) \slackk[m] \, \nodaldemand  
\end{align}
% 
It indicates the flow on line $\ell$ induced by the power flowing from generators at $m$ to consumers at $n$. When summing over all sources $m$ it yields the subflows induced by $\nodaldemand$, the same terms as on the right hand side in \cref{eq:flow_from_demand}; 
% \begin{align}
% \sum_m \allocateTransaction = -\ptdf\left(\slackk[m] \right) \,  \nodaldemand
% \end{align}
when summing over all sources and sinks, it yields again the power flow, thus
\begin{align}
\flow = \sum_{m,n} \allocateTransaction
\end{align}
% 
The consumed power at $n$ has to come from the slack $\slackk[m]$. As proofen in \cref{sec:proof_allocate_peer}, for each peer-to-peer relation $m \rightarrow n$, the ``traded`` power $\allocatePeer$  amounts to
\begin{align}
 \allocatePeer &= \slackk[m] \, \nodaldemand 
\label{eq:allocate_peer}
\end{align}
% 
Finally, when summing over all sinks the peer-to-peer trades yield the nodal generation 
\begin{align}
 \sum_n \allocatePeer = \nodalgeneration[m]
 \label{eq:sum_n_allocate_peer}
\end{align}
which we proof in \cref{sec:proof_sum_n_allocate_peer}. Summing over all sources yields the nodal demand 
\begin{align}
 \sum_m \allocatePeer = \nodaldemand
 \label{eq:sum_m_allocate_peer}
\end{align}
which straightforwardly follows from the fact that $\sum_n \slackk = 1$.
Both allocation quantities $\allocatePeer$ and $\allocateTransaction$ can be further broken down to generators $s$ or consumers $a$. The first comes through multiplying with the share that generator $s$ has in current nodal production $\generationshare = \generation/\sum_s \generation$, the second by multiplying with the nodal comsumer share $\demandshare = \demand/\sum_a \demand$. \\


One straightforward solution for $\slackk$ is the share in the total production $\slackk = c \, \generation$ with $c$ being defined as $c = 1/\sum_n \generation$. That leads to the nodal demand $\nodaldemand$ being covered by all nodal generators in the network proportional to their power production. In \cref{sec:solution_space_of_the_slack} we discuss the solution space of $\slackk$ and how this can be extended to an individual slack for each node $\Slackk$.



\subsubsection*{Network Optimisation}

% Maximise $f(x_l)$, with equality constraints $g_i(x_l)$ and inequality constraints $h_j(x_l)$
% 
% \begin{align}
%  \mathcal {L}(x_l,\lambda_i, \mu_j)=f(x_l)-\sum_i \lambda_i \, g_i(x_l) - \sum_j \mu_j \, h_j(x_l)
% \end{align}
% ...
% 

We linearly cost-optimize the capacity and dispatch of a simple power system. 

\begin{align}
    \underset{\demand, \generation, \capacityGeneration}{\text{max}}
    \left(\sum_{n,s} \capexGeneration \capacityGeneration - \sum_{n, s, t} \opexGeneration \generation - \sum_{\ell} \capexFlow \, \capacityFlow \right) \label{eq:minisation}
\end{align}
subject to following physical constraints. 
\\

The nodal balance constraint ensures that the amount of power that flows into a bus equals the power that flows out of a bus, thus reflects the Kirchhoff Current Law (KCL)
\begin{align}
    \sum_l \incidence \, \flow - \sum_s \generation + \sum_a \demand &= 0 \resultsin{\lmp} \Forall{n,t}
    \label[constraint]{eq:nodal_balance_lin}
\end{align}
Its shadow price mirrors the Locational Marginal Prizes (LMP) $\lmp$ per bus and time step. In a power market this is the \euro/\megawatthour-price which a consumer has to pay. Note that the flow $\flow$ in \cref{eq:nodal_balance_lin} is a passive variable only, given by \cref{eq:flow_from_ptdf}.\\

The generation $\generation$ is constraint to its nominal capacity
\begin{align}
 \generation - \generationpotential \capacityGeneration  &\le 0 \resultsin{\muuppergeneration} \Forall{n,s,t} 
 \label[constraint]{eq:upper_generation_capacity_constraint}\\ 
 - \generation &\le 0 \resultsin{\mulowergeneration} \Forall{n,s,t} 
 \label[constraint]{eq:lower_generation_capacity_constraint}
 \end{align}
where $\generationpotential \in \left[ 0,1\right]$ is the capacity factor for renewable generators. The constraints yield the KKT variables $\muuppergeneration$ and $\mulowergeneration$ which due to complementary slackness,
\begin{align}
\muuppergeneration \left( \generation - \generationpotential \, \capacityGeneration \right)  &= 0  \Forall{n,s,t} 
\label{eq:complementary_slackness_upper_generation} \\
\mulowergeneration  \, \generation &= 0 \Forall{n,s,t}
\label{eq:complementary_slackness_lower_generation} 
\end{align}
are only non-zero if the corresponding constraint is binding. \\


The transmission capacity $\capacityFlow$ limits the flow $\flow$ in both directions, such that 
\begin{align}
 \flow - \capacityFlow &\le 0 \resultsin{\muupperflow} \Forall{\ell,t} 
 \label[constraint]{eq:upper_flow_capacity_constraint} \\
 - \flow - \capacityFlow &\le 0 \resultsin{\mulowerflow} \Forall{\ell,t} 
 \label[constraint]{eq:lower_flow_capacity_constraint}
\end{align}
The yielding KKT variables $\muupperflow$ and $\mulowerflow$ are only non-zero if $\flow$ is limited by the trasmission capacity in positive or negative direction, i.e. \cref{eq:upper_flow_capacity_constraint} or \cref{eq:lower_flow_capacity_constraint} are binding. The complementary slackness 
\begin{align}
 \muupperflow \left( \flow - \capacityFlow \right)  &= 0 \Forall{\ell,t}
 \label{eq:complementary_slackness_upper_flow} \\
 \mulowerflow \left( \flow - \capacityFlow \right) &=  0 \Forall{\ell,t}
 \label{eq:complementary_slackness_lower_flow} 
\end{align}
set the respective KKT for flows staying below the thermal limit to zero. 
\\



\subsubsection*{Decomposing the Locational Marginal Prices and Allocating the Electricity Costs}
\begin{align}
 \lagrangian\left(\generation, \flow, \capacityGeneration, \capacityFlow, \boldsymbol{\lambda}, \boldsymbol{\mu} \right)   = &- \sum_{n,s} \capexGeneration \capacityGeneration - \sum_{n, s, t} \opexGeneration \generation - \sum_{\ell} \capexFlow \, \capacityFlow  \\
 &- \sum_{n,t} \lmp \left(\sum_\ell \incidence \, \flow  - \sum_s \generation + \sum_a \demand  \right)  \\ 
 &- \sum_{\ell,c,t} \cycleprice \, \cycle \, \impedance \, \flow  \label[constraint]{eq:langrange_cycle_constraint} \\ 
 &- \sum_{n,s,t} \muuppergeneration \left( \generation - \generationpotential \capacityGeneration \right)  + \sum_{n,s,t} \mulowergeneration \generation  \\
 &- \sum_{\ell,t} \muupperflow \left( \flow - \capacityFlow \right) + \sum_{\ell,t} \mulowerflow \left( \flow + \capacityFlow \right)     
\end{align}
% 
where $\boldsymbol{\lambda} = \left\lbrace \lmp, \cycleprice \right\rbrace $ and $\boldsymbol{\mu} = \left\lbrace \muuppergeneration, \mulowergeneration, \muupperflow, \mulowerflow \right\rbrace $ denote the set of related KKT variables. The global maximum of the Lagrangian requires stationarity with respect to all variables. The stationarity of the generation capacity variable leads to 
\begin{align}
 \pdv{\lagrangian}{\capacityGeneration}  = 0 \hspace{10pt} \rightarrow \hspace{10pt} \capexGeneration =  \sum_t \muuppergeneration \, \generationpotential  \Forall{n,s}
 \label{eq:capexGeneration_duality}
\end{align}

the stationarity of the transmission capacity to
\begin{align}
 \pdv{\lagrangian}{\capacityFlow} = 0 \hspace{10pt} \rightarrow \hspace{10pt} \capexFlow =  \sum_t \left( \muupperflow - \mulowerflow \right) \Forall{\ell}
 \label{eq:capexFlow_duality}
\end{align}


and the stationarity of the generation to 
\begin{align}
 \pdv{\lagrangian}{\generation} &= 0 
 \rightarrow \hspace{10pt}  \opexGeneration =  \lmp - \muuppergeneration + \mulowergeneration \Forall{n,s} \label{eq:opex_duality}
\end{align}
Solving \cref{eq:opex_duality} for the $\lmp$, leads to our first representation for Locational Market Price, which we will refer to as the ``Island Solution``,
\begin{align}
\lmp  =  \opexGeneration + \muuppergeneration - \mulowergeneration \Forall{n,s,t}
\label{eq:lmp1}
\end{align}
It connects the LMP directly with the local operational price and prices for the generation capacity constraint. However, we can derive a second representation for $\lmp$. Starting from the stationarity of the flow
\begin{align}
 0 &= \pdv{\lagrangian}{\flow}  \\
 &= - \sum_{m,\ell,t} \lmp[m] \, \incidence[m]  - \sum_{\ell} \left( \muupperflow - \mulowerflow \right)  - \sum_{\ell,c,t} \cycleprice \, \cycle \, \impedance &= 0
\end{align}
and multipying each term with the Power Transfer Distribution Factor $\ptdf$ leaves us with  

\begin{align}
  0 &= - \lmp + \sum_m \lmp[m] \, \slack[m]  - \sum_{\ell} \left( \muupperflow - \mulowerflow \right) \ptdf
  \label{eq:lmp_lmp1_reduced}
\end{align}
According to \cref{eq:slack}, the first term splits into the LMP at $n$ and the LMP weighted with the slack. The final term disappears as the $\cycle \, \impedance$ is the kernel of the PTDF $\ptdf$, so $\sum_l \cycle \, \impedance \ptdf = 0$. Solving \cref{eq:lmp_lmp1_reduced}, and replacing the LMP of the right hand side with the expression of the Island Solution in \cref{eq:lmp1} leads to 

\begin{align}
\lmp =  \sum_m \left( \opexGeneration[m] + \muuppergeneration[m] - \mulowergeneration[m] \right) \slack[m] - \sum_\ell \left( \muupperflow - \mulowerflow\right) \ptdf  \Forall{n,s,t} 
\label{eq:lmp2}
\end{align}
This representation  decomposes $\lmp$ to operational prices and prices for capacity constraints from all generators and transmission lines in the system. 
Further, \cref{eq:lmp2} reveals a crucial connection of the nodal electricity cost to the above presented flow allocation:
Multiplying both sides of \cref{eq:lmp2}  with the net nodal demand $\netconsumption$
breaks down the costs of power import to $n$. Then adding the price of the selfconsumption $\lmp \selfconsumption$ to both sides, finally breaks down the total gross demand cost at $n$,

\begin{align}
 \lmp \, \nodaldemand = \allocateCapexFlow + \allocateOpex + \allocateCapexGeneration \Forall{n,t}
 \label{eq:lmp_allocation}
\end{align}

with the allocated payments: \\
\begin{align}
 \allocateOpex &= 
 \sum_{m,s} \allocateOpex[n \rightarrow (m,s)]= 
 \sum_{m,s} \opexGeneration[m] \, \generationshare[m] \, \allocatePeer 
 &\text{OPEX for generators} 
\label{eq:allocate_opexGeneration}\\
 \allocateCapexGeneration &= 
 \sum_{m,s} \allocateCapexGeneration[n \rightarrow (m,s)] = 
 \sum_m \muuppergeneration[m] \, \generationshare[m] \, \allocatePeer
 &\text{CAPEX for the generators} 
\label{eq:allocate_capexGeneration} \\
 \allocateCapexFlow &=  
 \sum_{\ell} \allocateCapexFlow[n \rightarrow \ell] =  
 \sum_{m,\ell} \left( \muupperflow - \mulowerflow\right) \allocateTransaction  
 &\text{CAPEX for the transmission system} 
\label{eq:allocate_capexFlow}
\end{align}
Note that we used \cref{eq:allocate_peer,eq:allocate_transaction} which determine the peer-to-peer transactions between $m \rightarrow n$ and the corresponding usage of lines. \Cref{eq:lmp_allocation,eq:allocate_opexGeneration,eq:allocate_capexGeneration,eq:allocate_capexFlow} show how physically allocated flows on the basis of the slack directly tranlate into a cost allocation.\\ 

Let's have a look at the allocated OPEX first. Consumers at bus $n$ retrieve power from different generators $(m,s)$ and accordingly compensate their operational costs. The OPEX allocation behaves like P2P tradings between producers and consumers with fixed production prices. In this way, the generator $(m,s)$ retrieves the exact amount of money from consumers that it spends on the operation. In other words, all OPEX payments to generator (m,s) sum up to the total OPEX spent at (m,s), thus 
\begin{align}
\sum_{n} \allocateOpex[n \rightarrow (m,s)] = \opexGeneration \, \generation
\label{eq:no_profit_opex}
\end{align}


The CAPEX allocation for generators reveil a similar relation. According to the polluter pays principle, it differentiates between consumers who are `responsible` for investments and those who are not. If $\muuppergeneration > 0$, the upper Capacity \cref{eq:upper_generation_capacity_constraint} is binding. Thus it is these times steps which push investments in $\capacityGeneration$. If $\muuppergeneration = 0$, the generation $\generation$ is not bound and investments are not necessary. 
When summing over all CAPEX payments to generator $(m,s)$ each generator retrieves exactly the cost that were spent to build the capacity $\capacityGeneration$,

\begin{align}
 \sum_{n,t} \allocateCapexGeneration[n \rightarrow (m,s)] = \capexGeneration \, \capacityGeneration
\label{eq:no_profit_capex_generation}
\end{align}
where we used \cref{eq:complementary_slackness_upper_generation,eq:capexGeneration_duality}. So in total, throughout all time steps each generator $(m,s)$ receives the money it spends for invesments and operation (non-profit rule). 
% The non-profit rule can also be seen when looking at the the total revenue per generator, 
% \begin{align}
%  \sum_t \lmp \, \generation &= \sum_t \opexGeneration \generation + \muuppergeneration \generation - \mulowergeneration \, \generation &\Forall{n,s} \\
%  &= \sum_t \opexGeneration \, \generation + \capexGeneration \, \capacityGeneration &\Forall{n,s}
% \end{align}
%  where we multiplied both sides of \cref{eq:lmp1} by the generation and using the complementary slackness.  
\\ 

The allocation of CAPEX for the transmission system $\allocateCapexFlow$ builds up on the KKT variables $\muupperflow$ and $\mulowerflow$. Again the latter translate to the necessity of transmission invesments at $\ell$ at time $t$. Consumers which retrieve power flowing on congested lines, yielding a bound \cref{eq:upper_flow_capacity_constraint} or \eqref{eq:lower_flow_capacity_constraint}, pay compensations for the resulting investments. Again the sum of all CAPEX payments to line $\ell$ equal the total CAPEX spent, thus

\begin{align}
 \sum_{n,t} \allocateCapexFlow[n \rightarrow \ell] = \capexFlow \, \capacityFlow  
\end{align}
where we used the complementary slackness \cref{eq:complementary_slackness_upper_flow,eq:complementary_slackness_lower_flow} and the fact that summing over all sources $m$ and sinks $n$ the allocation equals the actual power flow as stated in \cref{eq:transaction_sum}. 
% A similar relation applies for the transmission CAPEX. Mulplying both side of \cref{eq:opex_duality} by the nodal injection $\sum_s \generation - \sum_a \demand$ and using the complementary slackness ...
% shows that the CAPEX per line amounts to the total congestion revenue
% \begin{align}
% \capexFlow \, \capacityFlow &= - \sum_{n,t} \lmp \, \incidence \, \flow \Forall{\ell} \label{eq:non_profit_branch}
% \end{align}
% The transmission lines carry power from nodes with lower LMP to nodes with higher LMP, the price difference times the transported power is their congestion revenue. \note{Sign is different to Tom's paper (there a minus is missing).}


\subsubsection*{Adding CO$_2$ Constraints}

Imposing an additional CO$_2$ constraint limiting the total emission to K,  
\begin{align}
 \sum_{n,s,t} \emission \, \generation \le \text{K} \resultsin{\emissionPrice} 
 \label[constraint]{eq:co2_constraint}
\end{align}
with $\emission$ being the emission factor in tonne-CO$_2$ per \megawatthour, returns an effective CO$_2$ price $\emissionPrice$ in \euro/tonne-CO$_2$. 
% The CO$_2$ price shifts the right hand side of the non-profit relation for generators \cref{eq:non_profit_generator} to
% 
% \begin{align}
% \capexGeneration \, \capacityGeneration + \sum_{t} \opexGeneration \, \generation &= \sum_{t} \left( \lmp - \emission \, \emissionPrice \right)  \, \generation \Forall{n,s} 
% \label{eq:non_profit_generator_emission}
% \end{align}
% This shows nicely the duality for exchanging the CO$_2$ \cref{eq:co2_constraint} for a shifted OPEX which includes the CO$_2$ costs
As shown in ... the constraint can be translated in a dual price which shift the operational price per generator
\begin{align}
\opexGeneration \rightarrow \opexGeneration + \emission \, \emissionPrice \label[relation]{eq:shift_opex_by_emission_cost}
\end{align}
This leads to allocated CO$_2$ cost compensation of node $n$ of
 \begin{align}
 \allocateEmissionCost &= \emissionPrice \, \sum_{m,s} \emission[m] \, \generationshare[m] \, \allocatePeer \Forall{n,t} \label{eq:allocate_emissionPrice}
\end{align}
which expands the allocation of the electricity cost in \cref{eq:lmp_allocation} to 
\begin{align}
 \lmp \, \nodaldemand = \allocateCapexFlow + \allocateOpex + \allocateCapexGeneration  + \allocateEmissionCost \Forall{n,t}
 \label{eq:lmp_allocation_with_emission}
\end{align}


% \newpage
% \section*{Showcase}
% 
% \subsection*{Network with CO$_2$ constraint}
% We illustrate the flow based cost allocation under use of the fictive network shown in \cref{fig:network}. It consists of nine buses and ten time steps. The solver optimizes the capacity of two generators, wind and gas, per bus. ...   
% 
% \begin{figure}[h]
%     \centering
%     \includegraphics[width=\textwidth]{compare_allocation.png}
%     \caption{Comparison between the flow based cost allocation and the LMP based cost per consumer. The left bars consist of the allocated OPEX $\allocateOpex$, the allocated CO$_2$ cost $\allocateEmissionCost$, the allocated generator CAPEX $\allocateCapexGeneration$ and transmission CAPEX $\allocateCapexFlow$, while the right bars show the of the nodal consumption times the LMP. }
%     \label{fig:cost_allocation}
% \end{figure}
% 
% \begin{figure}[h]
% \begin{subfigure}{.5\textwidth}
% \centering
%  \includegraphics[width=\textwidth]{network.png}
%  \caption{}
%  \label{fig:network}
% \end{subfigure}
% \begin{subfigure}{.5\textwidth}   
%     \centering
%     \includegraphics[width=\textwidth]{nodal_payments.png}
%     \caption{}
%     \label{fig:nodal_payments}
% \end{subfigure}
% \caption{Network used for showcasing. (a) shows the distributing of generation capacities $\capacityGeneration$, the widths of the transmission lines are proportional to their thermal limit $\capacityFlow$. (b) shows the total nodal payments according to the cost allocation.}
% \end{figure}
% 
% 
% 
% 
% 
% \newpage
% \subsubsection*{Relaxed CO$_2$ Constraint}
% 
% 
% \begin{figure}[h]
% \begin{subfigure}{.5\textwidth}
% \centering
%  \includegraphics[width=\textwidth]{network_relaxed_co2.png}
%  \caption{}
%  \label{fig:network_relaxed_co2}
% \end{subfigure}
% \begin{subfigure}{.5\textwidth}   
%     \centering
%     \includegraphics[width=\textwidth]{nodal_payments_relaxed_co2.png}
%     \caption{}
%     \label{fig:nodal_payments_relaxed_co2}
% \end{subfigure}
% \caption{Similar to \cref{fig:network} and \cref{fig:nodal_payments} but without CO$_2$ constraint.}
% \end{figure}
% 
% \begin{figure}[h]
%     \centering
%     \includegraphics[width=\textwidth]{compare_allocation_relaxed_co2.png}
%     \caption{Comparison of the stacked flow based cost allocation with the LMP based cost per consumer for each time step $t$ without CO$_2$ \cref{eq:co2_constraint}. Only one time-step $t=6$ determines the allocation of generator CAPEX $\allocateCapexGeneration$, as for all other time-steps \cref{eq:upper_generation_capacity_constraint} is not binding. Again note the cost scale difference between time step 6 and all others.}
%     \label{fig:cost_allocation_relaxed_co2}
% \end{figure}
% 
% \subsection*{How does the cost flow through the network}
% 
% \begin{figure}[h]
%     \begin{subfigure}{.5\textwidth}
%       \centering
%       \includegraphics[width=\textwidth]{opex_flow.png}
%       \label{fig:opex_flow}
%     \end{subfigure}%
%     \begin{subfigure}{.5\textwidth}
%       \centering
%       \includegraphics[width=\textwidth]{capex_flow.png}
%       \label{fig:capex_flow}
%     \end{subfigure}
%     \caption{}
%     \label{fig:fig}
% \end{figure}
    
\appendix

\section{Appendix}

\subsection{Proof \Cref{eq:allocate_peer}}
\label{sec:proof_allocate_peer}

\Cref{eq:allocate_peer} follows from summing $\allocateTransaction$ over all incoming flows to $n$ and taking into account the power that $n$ provides by itsself, $\slackk \, \nodaldemand$, which leads us to
\begin{align}
 \allocatePeer &= \slackk \, \nodaldemand - \sum_{\ell} \incidence \, \allocateTransaction \\
 &= \slackk \, \nodaldemand - \sum_\ell \incidence \left(  \ptdfEqual[m] - \ptdfEqual \right) \slackk[m] \, \nodaldemand  \\
 &= \slackk \, \nodaldemand - \left(  \delta_{n,m} - \dfrac{1}{N} - \delta_{n,n} + \dfrac{1}{N} \right) \slackk[m] \, \nodaldemand  \\
 &= \slackk \, \nodaldemand - \left(  \delta_{n,m} - 1 \right) \slackk[m] \, \nodaldemand  \\
 &= \slackk[m] \, \nodaldemand 
\label{eq:proof_allocate_peer}
\end{align}
where we used \cref{eq:slack} and the fact that the equally distributed slack amounts to $1/N$ for all N nodes in the network. 



\subsection{Proof of \Cref{eq:sum_n_allocate_peer}}
\label{sec:proof_sum_n_allocate_peer}
The relation follows from multiplying \cref{eq:flow_from_demand} with $\sum_m \incidence[m]$, and solving for $\allocatePeer$
\begin{align}
\sum_m \incidence[m] \, \flow &= - \sum_{m,n} \incidence[m] \, \ptdf \, \nodaldemand \\
\nodalgeneration[m] - \nodaldemand[m] &= - \delta_{m,n} \, \nodaldemand + \slackk[m] \, \nodaldemand    \\
\allocatePeer &= \nodalgeneration[m] - \nodaldemand[m] + \delta_{m,n} \nodaldemand \\
\sum_n \allocatePeer &= \nodalgeneration[m]
\end{align}



\subsection{Solution Space of $\boldsymbol{\slackk[m]}$}
\label{sec:solution_space_of_the_slack}

All choices of $\slackk[m]$ fulfilling \cref{eq:flow_from_demand} determine the solution space of $\slackk$. In the $N \times N$ nodal space this translates to the constraint given by \cref{eq:sum_n_allocate_peer} which denotes
\begin{align}
 \nodalgeneration[m] = \sum_n \slackk[m] \, \nodaldemand  
 \label{eq:solution_space_slack}
\end{align}
Solving for $\slackk[m]$ directly leads to 
\begin{align}
 \slackk[m] = c \, \generation[m] 
\end{align}
where $c$ is the inverse of the total consumption or production $c = 1 / \sum_n \nodaldemand   = 1 / \sum_n \nodalgeneration$. \\
However \cref{eq:solution_space_slack} has an inherent degree of freedom and can be reformulated as 
\begin{align}
 \nodalgeneration[m] = \sum_n \Slackk \, \nodaldemand  
 \label{eq:solution_space_slack_extended}
\end{align}
where $\Slackk$ denote the individual choice of slack for each bus $n$. For example this 





% \bibliographystyle{ieeetr}
% \printbibliography


\end{document}
