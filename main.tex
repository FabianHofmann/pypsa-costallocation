\documentclass[11pt]{article}
\usepackage{graphicx}
\usepackage[left=2.00cm, right=2.00cm, top=2.00cm, bottom=2.00cm]{geometry}
\usepackage{amsmath}
\usepackage[colorlinks]{hyperref}
\usepackage[backend=biber]{biblatex}
\usepackage{eurosym}
\usepackage[dvipsnames]{xcolor} 
\usepackage{subcaption}
\usepackage{accents}
\usepackage[capitalise]{cleveref}

\addbibresource{main.bib}
\graphicspath{{figures/}}

\crefname{relation}{Rel.}{Rels.}
\creflabelformat{relation}{(#2#1#3)}
\crefname{constraint}{Constr.}{Constrs.}
\creflabelformat{constraint}{(#2#1#3)}

\setlength\parindent{8pt}

\newcommand{\ubar}[1]{\underaccent{\bar}{#1}}
\newcommand{\note}[1]{\textcolor{Orange}{#1}}


\newcommand{\generation}[1][n]{g_{#1,s,t}}
\newcommand{\generationpotential}{\bar{g}_{n,s,t}}
\newcommand{\generationshare}[1][n]{\omega_{#1,s,t}}
\newcommand{\generationnodal}[1][n]{g_{#1,t}}
\newcommand{\capacityGeneration}{G_{n,s}}
\newcommand{\capacityFlow}{F_{\ell}}
\newcommand{\capexGeneration}{c_{n,s}}
\newcommand{\capexFlow}{c_{\ell}}
\newcommand{\opexGeneration}[1][n]{o_{#1,s}}
\newcommand{\demand}[1][n]{d_{#1,a,t}}
\newcommand{\demandnodal}[1][n]{d_{#1,t}}
\newcommand{\demandshare}[1][n]{\omega_{#1,a,t}}
\newcommand{\utility}{U_{n,a,t}}
\newcommand{\incidence}[1][n]{K_{#1,\ell}}
\newcommand{\ptdf}[1][n]{H_{\ell,#1}}
\newcommand{\ptdfEqual}[1][n]{\ptdf[#1]^\circ}
\newcommand{\slackflow}{k_{\ell}}
\newcommand{\slack}[1][n]{k_{#1}}
\newcommand{\Slack}{k_{m,n}}
\newcommand{\mulowergeneration}[1][n]{\ubar{\mu}_{#1,s,t}}
\newcommand{\muuppergeneration}[1][n]{\bar{\mu}_{#1,s,t}}
\newcommand{\mulowerflow}{\ubar{\mu}_{\ell,t}}
\newcommand{\muupperflow}{\bar{\mu}_{\ell,t}}
\newcommand{\lmp}[1][n]{\lambda_{#1,t}}
\newcommand{\flow}{f_{\ell,t}}
\newcommand{\cycle}{C_{\ell,c}}
\newcommand{\impedance}{x_\ell}
\newcommand{\cycleprice}{\lambda_{c,t}}
\newcommand{\injection}{p_{n,t}}
\newcommand{\netconsumption}[1][n]{p^{-}_{#1,t}}
\newcommand{\netproduction}[1][n]{p^{+}_{#1,t}}
\newcommand{\selfconsumption}[1][n]{p^{\circ}_{#1,t}}

\newcommand{\totalnetconsumption}{p^{-}_{t}}
\newcommand{\totalnetproduction}{p^{+}_{t}}
\newcommand{\totalselfconsumption}{p^{\circ}_{t}}

\newcommand{\lagrangian}{\mathcal{L}}

\newcommand{\allocatePeer}[1][m \rightarrow n]{A_{#1,t}}
\newcommand{\allocateFlow}[1][n]{F_{#1,\ell,t}}
\newcommand{\allocateTransaction}[1][m \rightarrow n]{A_{#1,\ell,t}}
\newcommand{\allocateCapexGeneration}[1][n]{\mathcal{C}^{G}_{#1,t}}
\newcommand{\allocateCapexFlow}[1][n]{\mathcal{C}^{F}_{#1,t}}
\newcommand{\allocateOpex}[1][n]{\mathcal{O}_{#1,t}}
\newcommand{\allocateEmissionCost}[1][n]{\mathcal{E}_{#1,t}}

\newcommand{\emission}[1][n]{e_{#1,s}}
\newcommand{\emissionPrice}{\mu_{\text{CO2}}}
\newcommand{\megawatthour}{MWh$_\text{el}$}
\newcommand{\totalcost}{\mathcal{TC}}
\newcommand{\impactcapexgeneration}{\Phi_{n,s,t}}
\newcommand{\impactcapexflow}{\Phi_{\ell,t}}

%math 
\newcommand{\resultsin}[1]{\hspace{12pt} \bot  \hspace{12pt} #1}
\newcommand{\Forall}[1]{\hspace{20pt} \forall \,\, #1 }
\newcommand{\pdv}[2]{\frac{\partial #1}{\partial #2}}

\begin{document}


\title{From Linear Optimization to Transmission Cost Allocation}
\author{Fabian Hofmann}

\maketitle

% \begin{abstract}
% The abstract text goes here.
% \end{abstract}


\subsubsection*{Lagrange Mutliplier}

Maximise $f(x_l)$, with equality constraints $g_i(x_l)$ and inequality constraints $h_j(x_l)$

\begin{align}
 \mathcal {L}(x_l,\lambda_i, \mu_j)=f(x_l)-\sum_i \lambda_i \, g_i(x_l) - \sum_j \mu_j \, h_j(x_l)
\end{align}
...


\section*{Linear Energy Modelling and LMP}



We linearly optimize the capacity and dispatch of a simple power system. 

\begin{align}
    \underset{\demand, \generation, \capacityGeneration}{\text{max}}
    \left(\utility(\demand) - \sum_{n,s} \capexGeneration \capacityGeneration - \sum_{n, s, t} \opexGeneration \generation - \sum_{\ell} \capexFlow \, \capacityFlow \right) \label{eq:minisation}
\end{align}

where 
\begin{itemize}
\item[] $\utility$ denotes the utility function per bus $n$, demand type $a$ time step $t$ 
 \item[] $\demand$ denotes the eletric demand 
 \item[] $\capexGeneration$ denotes the capital expenditure (CAPEX) per node $n$ and generator type $s$
 \item[] $\capacityGeneration$ denotes the generation capacity
 \item[] $\opexGeneration$ denotes the operational cost (OPEX)
 \item[] $\generation$ denotes the net generation in MW
 \item[] $\capexFlow$ denotes the CAPEX per transmission line 
 \item[] $\capacityFlow$ denotes the transmission capacity.
\end{itemize}
% 
\subsubsection*{Constraints}
In the following we neglect the utility $\utility$ of the nodal demand while fixing the demand $\demand$ to a predefined time-series. \\

The nodal balance constraint ensures that the amount of power that flows into a bus equals the power that flows out of a bus, thus reflects the Kirchhoff Current Law (KCL)
\begin{align}
    \sum_l \incidence \, \flow - \sum_s \generation + \sum_a \demand &= 0 \resultsin{\lmp} \Forall{n,t}
    \label[constraint]{eq:nodal_balance_lin}
\end{align}
Its shadow price mirrors the Locational Marginal Prizes (LMP) $\lmp$ per bus and time step. In a power market this is the \euro/\megawatthour-price which a consumer has to pay. \\

The generation $\generation$ is constraint to its nominal capacity
\begin{align}
 \generation - \generationpotential \capacityGeneration  &\le 0 \resultsin{\muuppergeneration} \Forall{n,s,t} 
 \label[constraint]{eq:upper_generation_capacity_constraint}\\ 
 - \generation &\le 0 \resultsin{\mulowergeneration} \Forall{n,s,t} 
 \label[constraint]{eq:lower_generation_capacity_constraint}
 \end{align}
where $\generationpotential \in \left[ 0,1\right]$ is the capacity factor for renewable generators. The constraints yield the KKT variables $\muuppergeneration$ and $\mulowergeneration$ which due to complementary slackness,
\begin{align}
\muuppergeneration \left( \generation - \generationpotential \, \capacityGeneration \right)  &= 0  \Forall{n,s,t} 
\label{eq:complementary_slackness_upper_generation} \\
\mulowergeneration  \, \generation &= 0 \Forall{n,s,t}
\label{eq:complementary_slackness_lower_generation} 
\end{align}
are only non-zero if the corresponding constraint is binding. \\


The transmission capacity $\capacityFlow$ limits the flow $\flow$ in both directions, such that 
\begin{align}
 \flow - \capacityFlow &\le 0 \resultsin{\muupperflow} \Forall{\ell,t} 
 \label[constraint]{eq:upper_flow_capacity_constraint} \\
 - \flow - \capacityFlow &\le 0 \resultsin{\mulowerflow} \Forall{\ell,t} 
 \label[constraint]{eq:lower_flow_capacity_constraint}
\end{align}
The yielding KKT variables $\muupperflow$ and $\mulowerflow$ are only non-zero if $\flow$ is limited by the trasmission capacity in positive or negative direction, i.e. \cref{eq:upper_flow_capacity_constraint} or \cref{eq:lower_flow_capacity_constraint} are binding. The complementary slackness 
\begin{align}
 \muupperflow \left( \flow - \capacityFlow \right)  &= 0 \Forall{\ell,t}
 \label{eq:complementary_slackness_upper_flow} \\
 \mulowerflow \left( \flow - \capacityFlow \right) &=  0 \Forall{\ell,t}
 \label{eq:complementary_slackness_lower_flow} 
\end{align}
set the respective KKT for flows staying below the thermal limit to zero. 
\\

\subsubsection*{Flow is a passive variable}
We treat the flow $\flow$ an a passive quantity only that is it is only influenced by the nodal injection.

The Power Transfer Distribition Factors (PTDF) $\ptdf$ determine the changes in the flow on line $\ell$ for one unit (typically one MW) of net power production at bus $n$. Thus with a fix demand $\demand$, they direclty link the generation $\generation$ to the flow on each line according to
\begin{align}
 \flow\left( \generation\right)  = \sum_n \ptdf \left( \sum_s \generation- \sum_a \demand \right)  
 \label{eq:flow_from_ptdf}
\end{align}
The PTDF have a degree of freedom: The slack $\slack$ denotes the contribution of bus $n$ to balancing out total power excess or deficit in the system. It can be dedicated to one bus, a sinlge ``slackbus``, or to several or all buses. The choice of slack modifies the PTDF accordingly 
\begin{align}
 \ptdf = \ptdfEqual - \sum_m \ptdfEqual[m]  \, \slack[m]
 \label{eq:ptdf_slacked}
\end{align}
where $\ptdfEqual$ denote the PTDF with equally distributed slack.
When bus $n$ injects excess power, it has to flow to the slack; when bus $n$ extract deficit power, it has to come from the slack. Summing over all ingoing and outgoing flow changes resulting from an positive injection at $n$ yields again the slack 
\begin{align}
\sum_\ell \incidence[m] \, \ptdf =  \delta_{m,n} - \slack[m] 
\label{eq:slack}
\end{align}
where $\delta_{m,n}$ on the right hand side represents the positive injection at $n$.
% Sensitivity of flow $\flow$ against changes of the power production $\generation$  
% \begin{align}
%  \pdv{\flow}{\generation}  = \ptdf \label{eq:flow_sensitity} 
% \end{align}


\subsubsection*{Breaking Down the Full Lagrangian}
\begin{align}
 \lagrangian\left(\generation, \capacityGeneration, \capacityFlow, \boldsymbol{\lambda}, \boldsymbol{\mu} \right)   = &- \sum_{n,s} \capexGeneration \capacityGeneration - \sum_{n, s, t} \opexGeneration \generation - \sum_{\ell} \capexFlow \, \capacityFlow  \\
 &- \sum_{n,t} \lmp \left(\sum_\ell \incidence \, \flow  - \sum_s \generation + \sum_a \demand  \right)  \\ 
%  &- \sum_{\ell,c,t} \cycleprice \, \cycle \, \impedance \, \flow  \label[constraint]{eq:langrange_cycle_constraint} \\ 
 &- \sum_{n,s,t} \muuppergeneration \left( \generation - \generationpotential \capacityGeneration \right)  + \sum_{n,s,t} \mulowergeneration \generation  \\
 &- \sum_{\ell,t} \muupperflow \left( \flow - \capacityFlow \right) + \sum_{\ell,t} \mulowerflow \left( \flow + \capacityFlow \right)     
\end{align}
% 
where $\boldsymbol{\lambda} = \left\lbrace \lmp \right\rbrace $ and $\boldsymbol{\mu} = \left\lbrace \muuppergeneration, \mulowergeneration, \muupperflow, \mulowerflow \right\rbrace $ denote the set of related KKT variables. The global maximum of the Lagrangian requires stationarity with respect to all variables. The stationarity of the generation capacity variable leads to 
\begin{align}
 \pdv{\lagrangian}{\capacityGeneration}  = 0 \hspace{10pt} \rightarrow \hspace{10pt} \capexGeneration =  \sum_t \muuppergeneration \, \generationpotential  \Forall{n,s}
 \label{eq:capexGeneration_duality}
\end{align}

the stationarity of the transmission capacity to
\begin{align}
 \pdv{\lagrangian}{\capacityFlow} = 0 \hspace{10pt} \rightarrow \hspace{10pt} \capexFlow =  \sum_t \left( \muupperflow - \mulowerflow \right) \Forall{\ell}
 \label{eq:capexFlow_duality}
\end{align}


and the stationarity of the generation to 
\begin{align}
 \pdv{\lagrangian}{\generation} &= 0  &\Forall{n,s} \\
 \rightarrow \hspace{10pt}  \opexGeneration &=  \lmp - \muuppergeneration + \mulowergeneration - \sum_{\ell} \left( \muupperflow - \mulowerflow\right)  \, \ptdf - \sum_{m,\ell}\lmp[m] \, \incidence[m] \, \ptdf   &\Forall{n,s} \label{eq:opex_duality}
\end{align}
% 
For the latter, we used \cref{eq:flow_from_ptdf} which sets the derivative of the flow with respect to the generation to 
\begin{align}
\pdv{\flow}{\generation} = \ptdf                                                                                                                                                   \end{align}
\Cref{eq:capexGeneration_duality,eq:capexFlow_duality,eq:opex_duality} show how the capital and operational prices translate into dual variables. \\

Note that \cref{eq:opex_duality} must hold for every choice of slack in the PTDF. According to \cref{eq:ptdf_slacked,eq:slack}, setting the slack to $\slack = \delta_{m,n}$ results in $\ptdf = \sum_\ell \incidence[m] \, \ptdf = 0$. This leads to our first represantion for Locational Market Price, which we will refer to as the ``Island Solution``,
\begin{align}
\lmp  =  \opexGeneration + \muuppergeneration - \mulowergeneration \Forall{n,s,t}
\label{eq:lmp1}
\end{align}
Accordingly, the LMP is directly determined by the local operational price and prices for the generation capacity constraint. 
However, from the Island Solution, we can derive a second representation for the LMP. Feeding \cref{eq:lmp1} back into \cref{eq:opex_duality} and applying \cref{eq:slack}, finally leads to 
\begin{align}
\lmp =  - \sum_\ell \left( \muupperflow - \mulowerflow\right) \ptdf + \sum_m \left( \opexGeneration[m] + \muuppergeneration[m] - \mulowergeneration[m] \right) \slack[m] \Forall{n,s,t} 
\label{eq:lmp2}
\end{align}
This equation generalizes the former Island Solution for the LMP which again can be reproduced by setting $\slack[m] = \delta_{m,n}$. It depicts the interdependence of the LMP, that is how $\lmp$ can be decomposed to operational prices and prices for capacity constraints from all generators and transmission lines in the system. 

\subsubsection*{Cost Allocation}

% As shown above, the choice of slack $\slack[m]$ in \cref{eq:lmp2} decides on the generators and lines to which the nodal price at $n$ is allocated to. 
As shown above, the choice of slack $\slack[m]$ decides on the generators and lines to which power imbalances are distributed to. Established flow allocation schemes haved used this connection and define the slack $\slack[m]$ for allocating power flows. According to the Equivalent Bilateral Exchanges (EBE) for example, the net demand of bus $n$ is covered by all other buses proportional to their net power production. The Marginal Participation (MP) on the other hand allows to allocate power also to buses with net demand and net production. In the Appendix, we revise the mathematical expressions of the slack $\slack[m]$ for the different allocation schemes. \\  

For a generic slack, the effective power which is produced at bus $m$ and assigned to consumers at bus $n$ is given by
\begin{align}
 \allocatePeer = \demandnodal \, \slack[m]   \Forall{n,t}
 \label{eq:allocate_peer}
\end{align}
which we will refer to as the peer-to-peer allocation. In other words, it indicates the part of demand $\demandnodal$ that is effectively covered by generators at bus $m$. Likewise, multipying the PTDF with the nodal demand will result in the flow induces by the net consumption at $n$ coming from the distributed slack 
\begin{align}
 \sum_m \allocateTransaction = - \demandnodal  \, \ptdf \Forall{n,\ell,t}  
 \label{eq:allocate_transaction}
\end{align}
where the PTDF include the slack $\slack[m]$ according to \cref{eq:ptdf_slacked}. The slack $\slack[m]$ in \cref{eq:allocate_peer,eq:allocate_transaction} should be chosen such that the sum over all recipients $n$ in $\allocatePeer$ results in the nodal production, thus
\begin{align}
 \sum_n \allocatePeer = \generationnodal[m] \Forall{m,s,t}
 \label{eq:generator_sum}
\end{align}
where $\generationnodal = \sum_s \generation$ is the nodal generation.
Then, weighted by the nodal production share $\generationshare[m] = \generation[m]/\generationnodal[m]$ the allocation distributes all power pruduced by generator $(m,s)$,
\begin{align}
  \sum_n \generationshare[m] \, \allocatePeer = \generation[m] \Forall{m,s,t}
  \label{eq:recipients_sum}
\end{align}
and the sum over all producers $m$ and recipients $n$ in $\allocateTransaction$ results in the power flow on line $\ell$,
\begin{align}
 \sum_{m,n} \allocateTransaction = \flow \Forall{\ell,t}
 \label{eq:transaction_sum}
\end{align}
\\

Once defined the quantities $\allocatePeer$ and $\allocateTransaction$ on a basis of a valid choice of slack, an intuitive decomposition of the nodal payments becomes vivid: Multiplying both sides of \Cref{eq:lmp2}  with the nodal demand $\demandnodal$ will breakdown the costs payed by consumers at node $n$ to the different expeditures

\begin{align}
 \lmp \, \demandnodal = \allocateCapexFlow + \allocateOpex + \allocateCapexGeneration \Forall{n,t}
 \label{eq:lmp_allocation}
\end{align}

with the allocated payments: \\
\begin{align}
 \allocateOpex &= 
 \sum_{m,s} \allocateOpex[n \rightarrow (m,s)]= 
 \sum_{m,s} \opexGeneration[m] \, \generationshare[m] \, \allocatePeer 
 &\text{OPEX for generators} 
\label{eq:allocate_opexGeneration}\\
 \allocateCapexGeneration &= 
 \sum_{m,s} \allocateCapexGeneration[n \rightarrow (m,s)] = 
 \sum_m \muuppergeneration[m] \, \generationshare[m] \, \allocatePeer
 &\text{CAPEX for the generators} 
\label{eq:allocate_capexGeneration} \\
 \allocateCapexFlow &=  
 \sum_{\ell} \allocateCapexFlow[n \rightarrow \ell] =  
 \sum_{m,\ell} \left( \muupperflow - \mulowerflow\right) \allocateTransaction  
 &\text{CAPEX for the transmission system} 
\label{eq:allocate_capexFlow}
\end{align}

\subsubsection*{Cost Terms}

Let's have a look at the allocated OPEX first. Consumers at bus $n$ retrieve power from different generators $(m,s)$ and accordingly pay for operational costs. The OPEX allocation can thus be seen as P2P tradings between producers and consumers with fixed production prices. In this way, the generator $(m,s)$ retrieves the exact amount of money from consumers that it spends on the operation. In other words, all OPEX payments to generator (m,s) sum up to the total OPEX spent at (m,s), thus 
\begin{align}
\sum_{n} \allocateOpex[n \rightarrow (m,s)] = \opexGeneration \, \generation
\label{eq:no_profit_opex}
\end{align}


The CAPEX allocation for generators reveil a similar relation. According to the polluter pays principle, it differentiates between consumers who are `responsible` for investments and those who are not. If $\muuppergeneration > 0$, the upper Capacity \cref{eq:upper_generation_capacity_constraint} is binding. Thus it is these times steps which push investments in $\capacityGeneration$. If in contrast $\muuppergeneration = 0$, the generation $\generation$ is not bound and investments are not necessary. 
When summing over all CAPEX payments to generator $(m,s)$ each generator retrieves exactly the cost that were spent to build the capacity $\capacityGeneration$,

\begin{align}
 \sum_{n,t} \allocateCapexGeneration[n \rightarrow (m,s)] = \capexGeneration \, \capacityGeneration
\label{eq:no_profit_capex_generation}
\end{align}
where we used \cref{eq:complementary_slackness_upper_generation,eq:capexGeneration_duality}. Hence, throughout all time steps each generator $(m,s)$ receives the money it spends for invesments and operation (non-profit rule). 
% The non-profit rule can also be seen when looking at the the total revenue per generator, 
% \begin{align}
%  \sum_t \lmp \, \generation &= \sum_t \opexGeneration \generation + \muuppergeneration \generation - \mulowergeneration \, \generation &\Forall{n,s} \\
%  &= \sum_t \opexGeneration \, \generation + \capexGeneration \, \capacityGeneration &\Forall{n,s}
% \end{align}
%  where we multiplied both sides of \cref{eq:lmp1} by the generation and using the complementary slackness.  
\\ 

The allocation of CAPEX for the transmission system $\allocateCapexFlow$ builds up on the KKT variables $\muupperflow$ and $\mulowerflow$. Again the latter translate to the necessity of transmission invesments at $\ell$ at time $t$. Consumers which retrieve power flowing on congested lines, yielding a bound \cref{eq:upper_flow_capacity_constraint} or \eqref{eq:lower_flow_capacity_constraint}, pay compensations for the resulting investments. Again the sum of all CAPEX payments to line $\ell$ equal the total CAPEX spent, thus

\begin{align}
 \sum_{n,t} \allocateCapexFlow[n \rightarrow \ell] = \capexFlow \, \capacityFlow  
\end{align}
where we used the complementary slackness \cref{eq:complementary_slackness_upper_flow,eq:complementary_slackness_lower_flow} and the fact that summing over all sources $m$ and sinks $n$ the allocation equals the actual power flow as stated in \cref{eq:transaction_sum}. 
% A similar relation applies for the transmission CAPEX. Mulplying both side of \cref{eq:opex_duality} by the nodal injection $\sum_s \generation - \sum_a \demand$ and using the complementary slackness ...
% shows that the CAPEX per line amounts to the total congestion revenue
% \begin{align}
% \capexFlow \, \capacityFlow &= - \sum_{n,t} \lmp \, \incidence \, \flow \Forall{\ell} \label{eq:non_profit_branch}
% \end{align}
% The transmission lines carry power from nodes with lower LMP to nodes with higher LMP, the price difference times the transported power is their congestion revenue. \note{Sign is different to Tom's paper (there a minus is missing).}


\subsubsection*{Adding CO$_2$ Constraints}

Imposing an additional CO$_2$ constraint limiting the total emission to K,  
\begin{align}
 \sum_{n,s,t} \emission \, \generation \le \text{K} \resultsin{\emissionPrice} 
 \label[constraint]{eq:co2_constraint}
\end{align}
with $\emission$ being the emission factor in tonne-CO$_2$ per \megawatthour, returns an effective CO$_2$ price $\emissionPrice$ in \euro/tonne-CO$_2$. 
% The CO$_2$ price shifts the right hand side of the non-profit relation for generators \cref{eq:non_profit_generator} to
% 
% \begin{align}
% \capexGeneration \, \capacityGeneration + \sum_{t} \opexGeneration \, \generation &= \sum_{t} \left( \lmp - \emission \, \emissionPrice \right)  \, \generation \Forall{n,s} 
% \label{eq:non_profit_generator_emission}
% \end{align}
% This shows nicely the duality for exchanging the CO$_2$ \cref{eq:co2_constraint} for a shifted OPEX which includes the CO$_2$ costs
As shown in ... the constraint can be translated in a dual price which shift the operational price per generator
\begin{align}
\opexGeneration \rightarrow \opexGeneration + \emission \, \emissionPrice \label[relation]{eq:shift_opex_by_emission_cost}
\end{align}
This leads to allocated CO$_2$ cost compensation of node $n$ of
 \begin{align}
 \allocateEmissionCost &= \emissionPrice \, \sum_{m,s} \emission[m] \, \generationshare[m] \, \allocatePeer \Forall{n,t} \label{eq:allocate_emissionPrice}
\end{align}
which expands the allocation of the electricity cost in \cref{eq:lmp_allocation} to 
\begin{align}
 \lmp \, \demandnodal = \allocateCapexFlow + \allocateOpex + \allocateCapexGeneration  + \allocateEmissionCost \Forall{n,t}
 \label{eq:lmp_allocation_with_emission}
\end{align}



\section*{Showcase}

\subsection*{Network with CO$_2$ constraint}
We illustrate the flow based cost allocation under use of the fictive network shown in \cref{fig:network}. It consists of nine buses and ten time steps. The solver optimizes the capacity of two generators, wind and gas, per bus. ...   

\begin{figure}[h]
    \centering
    \includegraphics[width=\textwidth]{compare_allocation.png}
    \caption{Comparison between the flow based cost allocation and the LMP based cost per consumer. The left bars consist of the allocated OPEX $\allocateOpex$, the allocated CO$_2$ cost $\allocateEmissionCost$, the allocated generator CAPEX $\allocateCapexGeneration$ and transmission CAPEX $\allocateCapexFlow$, while the right bars show the of the nodal consumption times the LMP. }
    \label{fig:cost_allocation}
\end{figure}

\begin{figure}[h]
\begin{subfigure}{.5\textwidth}
\centering
 \includegraphics[width=\textwidth]{network.png}
 \caption{}
 \label{fig:network}
\end{subfigure}
\begin{subfigure}{.5\textwidth}   
    \centering
    \includegraphics[width=\textwidth]{nodal_payments.png}
    \caption{}
    \label{fig:nodal_payments}
\end{subfigure}
\caption{Network used for showcasing. (a) shows the distributing of generation capacities $\capacityGeneration$, the widths of the transmission lines are proportional to their thermal limit $\capacityFlow$. (b) shows the total nodal payments according to the cost allocation.}
\end{figure}


% 
% 
% 
% \newpage
% \subsubsection*{Relaxed CO$_2$ Constraint}
% 
% 
% \begin{figure}[h]
% \begin{subfigure}{.5\textwidth}
% \centering
%  \includegraphics[width=\textwidth]{network_relaxed_co2.png}
%  \caption{}
%  \label{fig:network_relaxed_co2}
% \end{subfigure}
% \begin{subfigure}{.5\textwidth}   
%     \centering
%     \includegraphics[width=\textwidth]{nodal_payments_relaxed_co2.png}
%     \caption{}
%     \label{fig:nodal_payments_relaxed_co2}
% \end{subfigure}
% \caption{Similar to \cref{fig:network} and \cref{fig:nodal_payments} but without CO$_2$ constraint.}
% \end{figure}
% 
% \begin{figure}[h]
%     \centering
%     \includegraphics[width=\textwidth]{compare_allocation_relaxed_co2.png}
%     \caption{Comparison of the stacked flow based cost allocation with the LMP based cost per consumer for each time step $t$ without CO$_2$ \cref{eq:co2_constraint}. Only one time-step $t=6$ determines the allocation of generator CAPEX $\allocateCapexGeneration$, as for all other time-steps \cref{eq:upper_generation_capacity_constraint} is not binding. Again note the cost scale difference between time step 6 and all others.}
%     \label{fig:cost_allocation_relaxed_co2}
% \end{figure}
% 
% \subsection*{How does the cost flow through the network}
% 
% \begin{figure}[h]
%     \begin{subfigure}{.5\textwidth}
%       \centering
%       \includegraphics[width=\textwidth]{opex_flow.png}
%       \label{fig:opex_flow}
%     \end{subfigure}%
%     \begin{subfigure}{.5\textwidth}
%       \centering
%       \includegraphics[width=\textwidth]{capex_flow.png}
%       \label{fig:capex_flow}
%     \end{subfigure}
%     \caption{}
%     \label{fig:fig}
% \end{figure}
    
% \begin{appendix}
% Since the P2P comprises all net production and net consumption summing over all sources must result in the nodal demand  
% 
% \begin{align}
%  \sum_{n} \allocatePeer = \sum_a \demand[m]
% \end{align}
% as well as summing over all sinks results in the nodal generation
% \begin{align}
%  \sum_{m} \allocatePeer = \sum_s \generation
% \end{align}
% 
% The P2P relation can be further broke down to allocating generator specific $ \generationshare \, \allocatePeer $ where $\generationshare = \generation/\sum_s \generation$ denotes the share of generator $s$ of the nodel production. Similarly the allocation to consumers a are given by and consumer specific $ \demandshare[m] \allocatePeer$ with the nodal comsumer share given by $\demandshare = \demand/\sum_a \demand$.
% \end{appendix}

\newpage
\subsection*{Appendix: Allocation and Choices of Slack}

The presented cost allocation is sensitive to the choice of slack $\slack[m]$. The latter shift the shares of $\allocateOpex$, $\allocateCapexGeneration$, $\allocateCapexFlow$ and $\allocateEmissionCost$ for each node, whereas the total payment per node remains the same. In the following we show the connection of this cost allocation to established flow allocation methods. \\

\subsection*{Marginal Participation}


The peer-to-peer allocation for the MP scheme is given by 
\begin{align}
\allocatePeer &= \selfconsumption + \netconsumption  \, \dfrac{\netproduction[m]}{\totalnetproduction}
\end{align}

where 
\begin{itemize}
 \item $\injection = \left( \generationnodal - \demandnodal \right) $ denotes the nodal injection
 \item $\netproduction = \text{min}\left( \injection, 0 \right) $ the nodal net production 
 \item $\netconsumption = \text{min}\left( - \injection, 0 \right)$ the nodal net consumption and
 \item $\selfconsumption = \text{min}\left( \netproduction, \netconsumption \right)$ the nodal self-consumption. That is the power generated and at the same time consumed at node $n$. 
\end{itemize}



In order to allocate all gross demand $\demand$ to the slack, the latter is defined by 
\begin{align}
\slack[m,n] &= \dfrac{\selfconsumption + \netconsumption  \, \dfrac{\netproduction[m]}{\totalnetproduction}}{\demand}
\end{align}

...



% 
% % Direct Coupling:\\
% % \begin{align}
% %  \slack[m] = \dfrac{\generationnodal[m]}{\sum_{n} \generationnodal[n]} 
% % \end{align}
% % 



% \bibliographystyle{ieeetr}
% \printbibliography


\end{document}
